\Chapter{George Matthews and Thomas Slade}{Masters of the King’s Musick}
 
Exmoor, for most of its history, figured in foreign minds as a wild, barbarous place and the people who lived there as dull, unpolished rustics a hundred years behind the times; but in reality those rustics were perfectly capable of keeping up with the rest of England when they found it worth while, as for instance in the making of music. 
 Some time in the 1660s, the authorities of the Chyrch of England had felt the need to improve the music in their churches, and to achieve this they set about using local instrumental and vocal talent when it could be found; and so in due course many villages, if not all, came to have a small band (mainly wind instruments) and \textbf{male} chorus (“Let the women keep silent in church!”).

In the Exmoor area they flourished in Cutcombe, Winsford, Exford, Dunster, Bridgetown, Luxborough, Withycombe, Rodhuish and Leighland. Musicians’ galleries were erected in the west end of the church, and local singers and players were drawn in to form a ‘quire’ to lead and encourage the congregation to sing, though the first solid, four-square psalm tune cannot have appealed like the country dances. The innovations took root, performers improved, the music evolved and became livelier and more varied in style and metre. The quire – especially the players – became men of note in the neighbourhood and were called on to perform on week-day occasions as well.

Their reign lasted two hundred years, and while their music and manner of performing changed considerably over that time, it reflected, perhaps surprisingly, not the modal and freely rhythmic tradition of folk songs and dances but the style of the recent past among professional musicians. With this, in the late-eighteenth century, pehaps due to the rival Methodists with their love of vigorous congregational singing, their desire for greater refinement of thought and behaviour, and Charles Wesley’s incomparable outpouring of religious verse, a search for self-improvement in rural life, not least in music, became evident. (Quite apart from this, an Anglican quire could not let itself be outpaced by Dissenters...) 

But to savour the pleasure that the quires took in their music-making and to sppreciate their importance to their communities one must look away from Exmoor for a moment and into deep Wessex, and read of them in those passages of delight in which Thomas Hardy, recalling the traditions of his family, depicted the village musicians of his youth. There they stand, in Under the Greenwood Tree, Far From the Madding Crowd,Two On a Tower and The Mayor of Casterbridge, and they speak and reminisce and contend with a decent regard for the exigencies of their profession and a modest pride in their status and in the virtues of “men of note” who have gone before.

‘One Sunday I can well mind – a bass-viol day that time, and Yeobright had brought his own. ‘Twas the Hundred-and- thirty-third (Psalm) to Lydia,.
 
and when they’d come to “Ran down his beard and o’er his robes its costly moisture shed”, neighbour Yeobright, who had just warmed to his work, drove his bow into them strings that glorious grand that he e’en a’most sawed the bass-viol into two pieces. Every winder in church rattled as if ‘twere a thunderstorm.’
 
But even as Hardy wrote, these living ghosts were fading before his very eyes, dissolving into oblivion as at cock-crow along with the rest of the self-contained and self-sufficient life in which the English village had thriven since time out of mind. Town manners were filtering into country churches. Parsons influenced by the High Church Oxford Movement disapproved of the robust style of music and the vigorous, earthy performance, and being endowed with dictatorial powers they dismissed the bandsmen and replaced them with a barrel-organ or harmonium and choirs of surpliced men and boys.

The bandsmen felt the blow keenly, not least the manner in which it was delivered : 
 ‘”All we thought was, that for us old ancient singers to be choked off quiet at no time in particular, as now, in the Sundays after Easter, would seem rather mean in the eyes of other parishes, sir. But if we fell glorious with a bit of a flourish at Christmas, we should have a respectrable end, and not dwindle away at some nameless paltry second-Sunday-after or Sunday-next-before-something, that’s got no name of its own.”’ 
 The harm done to the self-respect of these servants of the church by the church itself cannot be measured, but it was the first step in the alienation of country folk from the church, and one might say that the “reforming” parsons had only themselves to blame. 
 Exmoor villages, despite their remoteness, inevitably suffered likewise at the hands of the improvers, and even this land of long memories could not preserve the names of the violists and clarionettists and serpent-players who created a rugged harmony and the men who led them, nor was there a Hardy to record – or invent? - the dialectal riches of their communing together. 
 But not all was lost, for it so happened that two of the local leaders, it seems, had talents of initiative, organisation and originality which established them as “characters” and won them a reputation in their neighbourhood which lasted till long after their death and almost into living memory. These two were George Matthews and Thomas Slade. 

George Mathews (ca 1790-1840), band and choir master and composer, lived at Leighland, on the eastern fringe of Exmoor and Brendon Hill, near Roadwater. His home, back in the days of King William IV, was Pitt Mill, in the valley of the stream which flows down from Brendon Hill to the sea at Watchet. He lived by milling, but lived for music, and if his quickness of temper might be counted a fault, he could have protested that it only burst forth when provoked by false notes and discord. 

Six days of the week he toiled in the mill or fields, and at night, by the flickering light of a tallow candle, he would sit at the kitchen table with his music manuscript book before him and write in arrangements of popular songs and dances, of anthems, psalms and carols (“pricking in the notes”, they called it). On the seventh day he would make his way across the fields or up a steep lane to the little whitewashed church on the hillside and lead his orchestra in the music he had arranged or composed : and there was more to this than met the ear. 

The music and musicians were well matched, for George had written specifically for his happy few : two clarionets, flute, key bugle, two horns, trombone and the leathern serpent he played himself. A country bandmaster did not send away to London for other men’s arrangements ; he wrote for whatever instruments were to hand, and if the Leighland Union Band was an eccentric combination by concert standards, the congregations did not cavil. Besides this, the psalms were not chanted, they were sung in the rhyming, metrical version of Tate and Brady, so that the band were always master and did not have to adjust to the irregular rhythms of a chant. 

Their duties were not burdensome. They sat in the west gallery, their privileged position, and at the start of the sermon they would slip out of the gallery and down the outside steps to a little ‘tib shop’ across the way, kept by a widow, and return – more or less steadily – only when a little “tacker”, told off to keep watch, warned them that “passon” was coming to the end of his sermon. But again, they would be called out for jollifications during the year and particularly in mid-September for the village revel at Roadwater, to play for the dancers. Here again George’s skill was evident, for the band played not only his arrangements of The Rising of the Lark and Rule Britannia but also his own compositions, the Roadwater Quick Step and the Somerset Waltz. 

Whatever the players’degree of skill, George was not the man to let them disgrace him, and he trained them rigorously. On summer evenings he would stand them in a field on the hillside above the mill and conduct them from the opposite slope. If all went well his face would light up with pleasure, but if a false note were heard or a entry mistimed he would career down to the valley bottom, leap the stream and go coursing up the other side like a “long-dog”, and the offending bandsman would feel the rough edge of George’s tongue. 

But with all this, the crown of the year was still Christmas, the only weekday holiday (apart from the revel) for working men and women; and then the band would play and the singers sing the carols proper to the neighbourhood, music springing from the eighteenth century, sonorous baroque or Handelian anthems with a vigorous beat that carried you irresistibly along. Such tunes were performed all over England, and their radiance was captured by Hardy even as it was fading away. It is worth while turning to his pages to recall, however dimly, the feeling of those days of simple, artless art, with their ‘ancient and time-worn hymns, embodying a quaint Christianity in words orally transmitted from father to son through several generations.’

With all that, however, George might seem no more than a run-of-the-mill musician, if an energetic one, but there was another, creative, side to him, for among the music he wrote on the old kitchen table were not only arrangements of popular songs but also dances of his own: the Somerset Waltz and the Roadwater Quick Step, which he and his band would play at the revel down in the valley. As well as that, he composed hymn tunes, some of which were included in a copious collection (some 800 in all meters)printed in Watchet and published by Thomas Hawkes, the land surveyor and agent in Williton, and intended as a musical vade mecum for all Wesleyan Methodist missionaries abroad; and so some of George’s music was heard even in far-off Canada.

In due time George died and it seems that the band ceased with him, for not only had he set an example that few could live up to, the church authorities demolished his chapel-of-ease and replaced it with a parish church in Victorian Gothic style, with room for a organ and choir but no musicians’ west gallery. Perhaps it was just as well for George’s peace of mind that he did not live to hear the declension from his group of sturdy tenors and basses to the high-piping trebles that had replaced them. 

\Chapter{Thomas Slade}{bandmaster and carol leader 1831 – 1907}

But again, not all was lost, for while the villages of these western vales might not show another band leader with all of George with his multifarious skills, rescue came at length, and the rescuer could well have served as a model for Samuel Smiles’s 'Self-Help'. 

Born in Roadwater in 1831 as one of ten children, Thomas Slade started with almost nothing but an inborn love of music, for his father, a farm labourer, had been so crippled by work in all weathers that he was reduced to cracking stone for a shilling a day. It was back-breaking and heart-breaking work, but one day a travelling Bible Christian minister stopped and said something which aroused his sense of human dignity and gave him hope. “Things” improved for him, and eventually he could afford to have young Tom apprenticed to a blacksmith. Young Tom slaved away for the seven long years and then bought the goodwill and set up as a smith himself. But all the time, music sang in his head, and he and his brother William determined to buy and learn the bass viol or ‘cello, for, said Tom, “’tis the queen of music, she can play the melodies and fill in the harmonies as well”. And so, one morning early, he left the forge with a few sovereigns in his pocket, walked the twenty up-and-down miles to Taunton, bought the viol complete with bow and case, strapped it on his back and walked the twenty miles home again, arriving in the early evening; and then, to show that he was not a weakling or an idle workman, he lit the fire and put in a couple of hours in the forge before going to bed. 

This must have been in his mid-thirties and the 1860s when the iron ore industry was beginning to bring a little unaccustomed prosperity to the neighborhood, and with prosperity a little leisure for enjoyment. With Tom to lead, the old Christmas tradition of the ‘waits’ going the rounds of the farms was maintained and continued long after it had ceased nearly everywhere else in Somerset; and as an spur for Tom and his men, some of the Cornish miners of Bredon Hill had brought one or two carol tunes unknown in George’s day. 

This carolling came at a price, however, as Tom, though not without humour, had a vein of sternness derived from the hard times of his boyhood and he kept a firm hand on his musicians. As Christmas approached he rehearsed them, and year after year, just before midnight on Christmas Eve, the musicians with their instruments would come along the village street to Tom’s door and into his ‘parlour’ for cordials and cakes to build them up for the long, cold round. (In folk memory there were no rainy or misty or thoroughly wretched Christmas Eves. . .) 

Then they would set out up the village street. Friendly talk and a chuckle made pleasant harmony in the night, but as they drew near the first stop, the quarry-owner’s residence, Tom would caution them with such words as, “Quiet now, friends. Number Six : Mortals awake! Take your places , but not a sound till I give the cue,” – for he knew how precious a part of music is a well-prepared silence, and he felt that while speech and laughter are the unique human privileges, music is divine, and the sleepers should not be awakened by idle chatter but by a noble harmony, “a concourse of sweet sounds.” As William Dewy phrased it, “Keep from making a great scuffle, but go gently, so as to strike up all of a sudden, like spirits.” And so, with the quire forming a semi-circle, Tom would beat four, would say “Mortals, awake!”, the players would sound the major chord, and in a stentorian voice he would announce:
“Mortals, awake! Rejoice and sing
 The gloaries of your Heavenly King”.
 Soon they would catch the glimmer of a candle in an upstairs window and then a female form, perhaps not the enchanting vision of Fancy glimpsed by the Mellstock quire but still promising a reward for their efforts. The lady of the house would open the door and invite them in for hot cider and cake and perhaps a half-sovereign to be shared among them all; then on to the next house for another carol, and on again, and so move on through the starry silence until at three in the morning they returned to the village to play in Christmas Day with,
 
 \begin{quote}
 	Once more, behold, the day is come,
	 The bright and glorious morn :
	Let every tongue on earth rejoice
	 For Christ the Lord is born ;
 \end{quote}

 
‘and what more entrancing, wrote Lewis Court, who heard it every year as a boy, ‘than to be awakened by the strains of Christmas music stealing in upon one through the silence of the night, or on the clean air of a frosty morning! – the deep, full tones of the bass viol, the celestial notes of the clarinet, the suave, appealing plaint of the flute, and the blend of good human voices.”

In the morning the musicians played again in the gallery of the chapel (the village had no Anglican church), but after Old Chrismas Day, 6th January, no more carols were played or sung until the Christmas season came round again, for Thomas, and many like him, held that everything had its due season, and a carol out of its proper time was an overturning of the universal order that he could not abide. 

He died on Boxing Day 1907 and the tradition of the ‘waits’ faded away. His village and others of Exmoor had their musicians, but the old rustic instruments had fallen out of favour, and as for the singers, rural depopulation carried them away .But in a few places – Cutcombe, Exford, Winsford, Roadwater, and Odcombe in south Somerset, the manuscripts survived together with some memories of the tradition, and the spread of the West Galley movement since its inception in 1991 has given the old carols a resurgence that no one a hundred years ago could possibly have foreseen. 
