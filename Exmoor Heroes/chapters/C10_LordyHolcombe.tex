\Chapter{"Lordy" Holcombe}{Prince of Poachers}

The old-time poachers on Exmoor and the Quantock Hills were sportsmen to their finger-tips, totally unlike the destructive, town based marauders of to day, and if landowners and keepers detested them as distributors of their private property among the lower orders, those poverty stricken working people made no complaint. One such agent of rural enterprise, looking back on his life, wrote, "It must be remembered that (in mid 19th century) hard work did not pay. Farmers were bent on making money, and if the labourer received seven shillings a week, it was considered as much as he was worth. The villages were bursting with folks. Nearly every cottage was clogged with boys and girls growing up. And they all had to live somehow. The price of bread was high. I have known it up to thirteen pence a loaf. As for meat, it was an unheard of luxury. What was more natural than that puzzled chaps, with no brains for making their way in the world, should help themselves now and again to wild creatures, both birds and beasts? Things are different now, but if you had lived fifty or sixty years ago, and seen with my eyes, you might not be so hard on the poor poacher." 

"Lordy" Holcombe was born and bred in Dulverton and was proud of it. His family lived in a low lying street just off the Little Bridge and known as Duck Paddle; (later generations, less susceptible to the picturesque, renamed it Chapel Street, but no matter). The father was a shoemaker by trade but, like a good many others, eked out a tight living by poaching. As young John grew up, he helped his father occasionally, but he never became a proper shoemaker: the thrill of poaching and pitting his wits against his natural and unnatural adversaries got into his veins and took over.

Dulverton at that time had a schoolmaster who made "miching" a pleasure to be paid for with a beating. Old Keen really was keen, and Holcombe remembered him as “a big, smart man, who shot, fished and played on the fiddle and cared nothing for our feelings". On winter mornings, with the thermometer degrees below zero, there was no fire in the schoolroom, and if the boys wanted warmth they had to bring sticks with them. The only other warming came from the hand of Keen, who also brought a stick and used it with indiscriminate zeal, making their hands tingle for hours.

On leaving school John tried broom making, followed by railway construction down at Starcross and Powderham Park, but, in his own disarming comment, "as a boy (he) was not partial to work, and it happened that a man employed on the works was of the same kidney." They both belonged to Dulverton, and between them they bought a dog, "a noble looking animal, a lurcher," which they reckoned would maintain them, at least on the way home. On nearing Dulverton they observed quite a dozen hares on Helverton Knap. Holcombe's partner marked one of them and held up the dog, who marked her too. Away flew the hare, but the dog was too fast, and caught her and brought her back. Holcombe flattered himself that he had a fortune between that dog's jaws, but the lurcher let him down with a peculiarity: nothing would induce him to enter the same field twice. Eventually the partners sold him to some gypsies for a sovereign   but John Holcombe had started on his adventurous road.

From early days he was known as "Lordy"; but if we imagine that he acquired his nickname for some unexpected stateliness of demeanour, he will quickly set us right. It was, he said, "all along of an old song."

He and several mates were in the New Inn, Dulverton, when one of them proposed that he should sing. He knew one ditty he had learnt of some old fellow, a highwayman's song, and two of the verses ran as follows: \\

 I took a kind and loving wife, \\
 I loved her as dear as I love my life:\\
 And to maintain her both fine and gay\\
 To all the world my life I'll pay.\\

 I robbed Lord Mansfield, I do declare,\\
 And left him on St James's Square.\\
 I bade him good night and the best of cheer\\
 While I ran to spoil with my comrades dear.\\
 
Lord Mansfield was probably the famous 18th century judge and friend of John Wesley, but it is unlikely that his fame persisted in remote Dulverton 60 or 70 years later. Still, the name caught the attention of one of Holcombe's boon companions, a tailor and bugler in the cavalry, who surnamed him "Lord Mansfield", and the title in shortened form stuck with him ever after.

Oddly enough, John was not the first of the family to be ennobled in this way. His great uncle was old "King" Holcombe of Zeal Farm, Hawkridge, though like many other farmers he dressed roughly, in breeches and stockings, with no gaiters. John solemnly maintained that with a king in his ancestry he had a vein or two of royal blood, and that explained his fondness for the "the royal game, the royal venison."

One evening soon after this Lordy, who had drunk more than a drop and become "a bit elevated", offered to bet that he would go and kill a hare and return within the hour. No sooner said than taken. He left the inn, made his way up the High Street and on to a hillside plantation. The hares, though, seemed disinclined to oblige him and he was turning for home, rather crestfallen, when it happened.

Unknown to Lordy, two keepers were waiting for him, and the moment he jumped over the gate on to the road they collared him and marched him off to give account of himself to 'Squire Bisset.

This Master of the Devon and Somerset Staghounds had no love for poachers, especially deer poachers who interfered with his own pleasure, and he rarely erred on the side of leniency. Neither then nor ever did he show Lordy the slightest favour, and the poacher had every reason to look on him as a personal foe: yet he could not refrain from saying that his opponent was "as fine a man as has been seen in our country: tremendously heavy   one of the heaviest men that ever rode hunting   but active and good all through."

Lordy walked along the high road a captive. The keepers, one on each side, held him by the collar, and the more he reflected on the situation the less he liked it.

Suddenly he sprang. broke free and ran off toward the town; but as Lordy's luck would have it, the keepers had a big mastiff with them and let him slip. He was muzzled, but Lordy could make no headway against him: the dog kept beating him down with his paws.

Presently they reached the squire's house and he was ushered in. The keepers went off to speak to their master, leaving Lordy alone in the kitchen. In the cupboard was a supply of beer, to which Lordy helped himself with a persistence which spoke of an urgent desire to lay in supplies against a season of drought in Taunton jail. All too soon the keepers returned with Bisset, and his tone was stern and his manner contemptuous.

"Ah!" said he to the keepers, "so you've got him, and a good job, too."

His manner and weight would have intimidated a common overnight poacher, but Lordy was not so easily cowed. Besides, the liberal refreshment was taking over and he fired off a salvo of names at the squire, but Bisset was in no mood for play.

"Let him go," he said to the keepers, "but watch him out of the park. He'll hear more of this."
And he did. Summoned for trespassing in pursuit, he was sentenced to one month's imprisonment.
An occasional spell in "chokey", as prison began to be known about this time, was a poacher's occupational hazard, and Lordy accepted it philosophically enough. His "old lady", in fact, rather welcomed it, on the ground that when he was behind bars he could not get into more trouble. Besides this, conditions inside, though harsh, were probably no worse than those endured by a farm labourer day in day out.
 The consequences, of course, fell punishingly on the wife and children left at home, in terms of the felt "disgrace" and the privation to be suffered, but a sturdy bachelor might get through his time not greatly the worse for it.
 
With Lordy, at any rate, the pleasure of poaching, of pitting his wits against landowner, keeper and game, outweighed the virtual certainty of being caught from time to time and packed off to Taunton for yet another spell.
 The regulations provided for prisoners to be transported the 20 miles to jail, and in winter a cold coming they had of it; but if they walked to prison   a comfortable day's stroll if they were not fettered   they were allowed seven shillings, the equivalent of trap hire. I suspect that officialdom found ways of deducting a "bob" here and a "tanner" there for breakages, clothing, caution money and what not, but seven shillings, as four days' wages on the land, was worth walking for,
 Think not, however, that the privilege of working for Her Majesty, as an unwilling guest at her presumed pleasure, came automatically. The prisoner had to sit an examination   though on reflection, "stand" would be a better word. He was stripped like a recruit, and if the doctor found him to be a healthy subject he would report, "This man's sound: he's fit for the wheel." Thereupon the prisoner donned his uniform, a short round jacket with no pockets, coarse dark breeches, shoes and a cap shaped not unlike an old nightcap, but knitted. He also wore an odd little necktie with two strings, a minute concession to elegance and human dignity.
 
The chief warder in Lordy's youth, Bill Dinner, was a hard hearted, sarcastic fellow, or so at least he seemed to his charges. When the time came for them to go on the treadmill he would enter and say, "Come, my little lads, be ready. There's a beautiful instrument, a pianner, out here for you to play with." But when they came to it they found an instrument not for delicate finger fingering but for laborious trampling, and so contrived as to benefit neither their souls nor their soles. Then Billy's tone would change and he would say sternly, "Go and tread that for seven hours," and tread it they did, apart from two and a half minutes' rest every twenty minutes. The wheel of the treadmill was of wood, and farmers, labourers and townspeople brought grain to be ground.

When the treadmill was running "leary", they called the process "grinding wind", and the lack of resistance offered by the wheel, with the knowledge of the futility of the exercise, made the work repugnant. They much preferred to grind corn, as the wheel then went more slowly. But all in all, the treadmill galled the spirit more than any other one of the purposeless conditions of prison life. Lordy deeply resented it, but the love of poaching was stronger than any fear of punishment; and even with the risk, it paid better than farm work, and the excitement of the thing suited his constitution.

For a short time he was impressed by a remark of the prison chaplain, that "poaching leads to other things: first a hare, next a fowl, next sheep stealing, next highway robbery, next murder!"; but it seemed a pretty unlikely succession of misfortunes. He felt he had to go through a good many rubs before he came to murder.

Quite apart from that, he saw himself as carrying on a tradition sanctioned by long usage. When he was a young man, deer, in a sense, were like rabbits, four-legged pensioners whose fate depended on the farmer on whose land they had taken up quarters. If the farmer's first thought was to please the gentry, he would preserve the deer; but if his corn and turnips loomed larger in his mind thoseother people's pleasure, he would send for his irregular Lordship or at least connive at what the poacher did. Add the fact that while the law protected the "rights" of the landowner, including game, those rights were the sum of several centuries' wrongs and the extinction of common rights by the power of the law, and one can appreciate that the poacher was only exercising the rights of which his forefathers had been cheated; and he ventured out always at considerable risk to limb, liberty and life.

Back to Dulverton came Lordy, and straightway set up in business again, this time in a strangely assorted trio with Harry Liscombe, "a sawyer and a fine fellow" but not over strong on dash, and tailor Marley, a little splinter shinned fellow who wore a box hat and swallow tail coat. Innocent he seemed to all the world: he might almost have been taken for a Bryanite preacher, but his sombre garb covered, if not a multitude of sins, at least a frequent brace of pheasant in his box hat.

One afternoon the three of them set out to go wiring for pheasants on Sir Thomas Acland's allotment close to the keeper's lodge on Winsford Hill. They had set up their wires round the fences when they began to suspect that a man was creeping around watching them; but whoever he was, he kept under cover, so they shrugged off their unease, and with evening coming on they made for home.

Next morning they were out again before daybreak, looking across the same field in the half light, and again it seemed a man was creeping about in the "vuzz". After a while the watcher got numb and was forced to stand up and reveal himself.

"There!" said Marley, "I told 'ee they was watchin' us."
 The game was up   for the moment   and the wires as good as lost. "Let's move on and get out of this," said Lordy, and they walked off up the hill.
 
They had not gone far when, looking back, they saw their pursuers, seven or eight of them. These were nimble fellows and soon one of them pounced on Lordy and triumphantly announced, "Breakfast for you down Winsford." Marley and Liscombe were told the same, but it soon became clear that this un looked for bonus was meant to be earned by whatever rigours the magistrates could eventually devise.

Still, a breakfast was a gift horse not to be scrutinised with too critical an eye, so off they went down to Winsford, and sure enough it was no barmecide feast the landlord of the Royal Oak set before them, and Lordy and friends did it quite as ample justice as any of the more painful variety they expected would follow.

But then, for the keepers, came the hitch. The magistrate, who was the Winsford parson, the Rev Mr Mitchell, was not at home, and in this pretty how d'ye do, one solution commended itself to both captives and captors. They stayed on drinking at the Royal Oak at Sir Thomas's expense till two o'clock, when they were treated to a excellent dinner of fried meat and beer, as much as they cared to have.   "Old Sir Tummus" was noted for his Liberal politics and liberal hospitality, but whether he envisaged anything on this scale is less than likely.

Soon after this they were told that the magistrate had come home, so off they trooped to the vicarage, and the magistrate began to question the keepers somewhat as follows   with discomfiting results: "
 "What have you seen?" 
   "These three fellows."
 "When?" 
   "Last night and this morning."
 "Can't you be more precise? At what time?"   
   " When 'twas gettin' dimpsey, sir."
 "Was it light enough to see clearly?"
   "We saw them clear enough."
 "What were they doing?"
   "Layin' wires, we think."
 "Can't you be sure?"
   "We know them all right."
 "Did you catch them in the act?"
   "Can't say as we did, exactly."
 "Did you see them touch any of the wires?"
   "Well, no."
 "Hum." (Turns to the three): "You're suspicious characters. You ought to be had up as rogues and vagabonds for wandering about the country." (To Marley): "You, what were you doing there?"
 (Marley): "If you please sir, I'm asthmatical, and I was gatherin' a little agrimony and wood betony for my complaint."
 "Humbug!" rapped the magistrate, but he was beaten, and he knew it, and so did the keepers. There was no real evidence and he had to dismiss the case.

The hospitality of the Royal Oak was now exhausted and they were sent about their business, but totting up the events of the day they reckoned they had come out pretty well, in credit to the tune of an excellent breakfast and their substantial share of the forty quarts of beer that they and the little army had battled their way through and that Sir Thomas would have the privilege of paying for.
 To cap this, on the way home they found a brace of pheasants caught up in the wires. These they commandeered, made a good price of them and spent it in carousing at the Nightingale.
 Finis, as the poet said, coronat opus.
 
\Flourish

Luck, however, was as capricious with Lordy as with anyone else, and as time went on he was sentenced to one, two or three months in jail so often that at last it became a joke   or so he assured the world. How he managed to enjoy the joke is something of a mystery, and his being careless of the consequences gave him a dash which carried him further still.

As for the right or wrong of his depredations, he strenuously maintained that he had as good an entitlement to game as anyone else, and taking the long view that the ancestors of such and such a landlord had probably acquired the estate by conquest or legalised robbery, it would take a tortuous and toadying moralist to prove Lordy wrong.

From pheasant he progressed to deer, and had his first lessons in the craft from a North Molton man, Tom Bell. Most folk knew Tom, or strongly suspected him, but he poached with an accent on "craft", calculating the risk and taking immediate action if an affair looked likely to go wrong.

He was a labouring man who had married a farmer's daughter and lived in. His father in law was one of the few who knew nothing of his poaching, or doubtless he would either have refused him permission to marry his daughter or laid down the law pretty forcefully.

Be that as it may, one day Bell was shooting on an adjoining farm when somebody saw him. Perceiving this, he pelted off home, took the pony out of the stable and galloped hell for leather to Winsford. Arriving in the village he walked into the Royal Oak with no appearance of haste and said to the landlord, "Mr Paull, do tell me what time 'tis. I want to know particular." The landlord obliged, and Bell downed a leisurely pint, mounted the pony and rode home again.

A day or two later a summons was served on him for trespass in pursuit of game. He appeared before the bench and called as witness to his innocence the landlord of the Royal Oak, who proved that near enough to the time in question he had been in Winsford.

"How could he be poaching when he was at my house?" asked the perplexed host., and the no less perplexed if suspicious magistrates could only yield to the evidence.

Lordy learnt much from Tom Bell, but not, unfortunately for him, the knack of keeping out of trouble, and only a certain native caution kept him from an activity which would have earned him not three months' in Taunton jail but transportation for seven years of his life. He sniffed the wool, so to speak, but left it on the sheep's back.

Once he was asked by a poor man living in a lonely cottage, with no land of his own, to go night hunting and stay a week or a fortnight. Lordy went with a friend, and as there were plenty of hares about, the first night provided some pretty good sport.

When they came back in the early hours, the friend observed that he could smell meat, but they took no further notice and all three went to bed. Their host, a labourer, had to go to work at seven, but the poachers were under no such constraint and slept on, and during the day the wife brought them breakfast and dinner. 

The man returned about five o'clock, and when it began to grow dark they had supper, smoked a pipe and held a discussion as to the best coverts to visit. About eight o'clock they sallied forth, again with success. They came home and went to bed, but the two poachers were struck, this time more than before, with the pervading smell of meat.

Something had to be done. They got up, lit a candle and began searching, and at last Lordy's friend found the meat, and plenty of it, too.

"Come over here," he said quietly. Lordy crossed the room, and there he saw a barrel half full of something salted in. He said, "This won't do, you know. This is mutton, "sheep's collar unbuttoned". They'll be searching the house, that'll be the next thing. I shan't stay here"   for he felt certain the man could not have bought the meat out of his mere seven shillings a week.

"We may as well stay one night more," said his mate, “we’ve had good sport." 
 "Just as you please," said Lordy, "but I don't like the look of it."
 The following night they went out again, and again enjoyed "very good sport", but Lordy felt increasingly uneasy. They passed no remark to the man or his wife about the contents of the barrel, but the thought of seven years' transportation loomed larger and large in Lordy's mind, and on the way back he said, "We're thinking it's about time we went home. If we bide round here too long we may get caught."
 They returned to the cottage, lit the fire, shared out the money, picked up their "ragged shirts" (their nets) and prepared to carry home the hares, accompanied by their faithful dog. On taking leave they promised to come again some day, but Lordy knew in his bones he never would. Quite apart from the risk, sheep-stealing, somehow, was not in his line . 
 Still, he posed a growing problem to the landowners and game "preservers" of the neighbourhood, and in the end they dealt with him by the method summed up in the phrase. "If you can't beat en, use en." Mr Barnett. of Morebath House, just over the border into Devon, offered him the post of gamekeeper   not at Morebath but on an estate near Honiton, well out of harm's way, they thought. But a law abiding, law enforcing existence made for anything but ease. and at the Three Tuns, in Honiton, Lordy was beaten up and could not use his hand or pull a trigger for weeks. After fourteen years with the Barnetts   "splendid masters, all three"   he returned to Dulverton and finished his working life as a water bailiff on the River Barle, and felt he could "say without fear that the post was never held by a better."
	 