\Chapter{Richard Slader}{Poet and Pedlar}

He was known in North Devon as a “proper character”, a phrase common in his day but now perhaps a little outdated.

At any rate, “characters” have this in common with rebels ; they go their way regardless of popularity or unpopularity: but with this difference: the rebel acts out of a sense of duty; the "character" is as he is because he cannot be any way else. Duty and conscience can make rebels by the hundred, but the true "character", the genuine eccentric, is a man or woman in a thousand. Eccentricity is a natural phenomenon: those who consciously play the eccentric make themselves laughing stocks: all the world knows that they are clamouring for recognition, even if it takes the form of disapproval. The genuine “character” is made in a superior mould. He cares for neither praise nor blame. Such a man was Richard Slader.

That, no doubt, was just as well for him, because by no means everyone in the Exmoor country early in the 20th century approved of Richard; but in his 69 years he outwore disapproval. His limitations, like his virtues, became accepted as parts of the man himself, as he became accepted as an almost timeless part of the Exmoor landscape.

That at least is how it seems eighty years on, and people with a clear memory of him must be few. If not everyone liked him, that is in itself a recommendation, for there are certain people whose approval would be an insult. The clever ones thought they saw through him, but they did not see half way. The wise knew that he hid far more than he revealed.

His reputation as a "character" was purely local until late in his life, when he appeared under the name of Richard Tiler   or "Urchard" in W. Joyce's Echoes of Exmoor. Some readers felt that the author was out of sympathy with his character and it must be confessed that he gave this impression when writing of men and women of a different religious persuasion from his own. "Urchard" was portrayed as an eccentric, certainly, but devious, lachrymose, avaricious, petty minded and in an indefinable way rather ridiculous. This caricature was corrected by Richard's great nephew, J.M. Slader, but his delightful little book, Dicky Slader:: Pedlar Poet, has been out of print since the mid 1960s. Owners of copies cling on to them, and not all libraries have them, but perhaps these words will encourage readers to seek them out.

\Flourish

Be as 'twill, as Dicky would have phrased it, he was Exmoor born and bred, and he came into the world at Hunnawins, near North Molton, the tenth child and sixth son of William and Mary Ann Slader. Exmoor in 1857 was still untouched by industry, and their yeoman family had been settled in the neighbourhood for centuries. William was churchwarden to North Molton. The six brothers were all different in character, from John, the upright, hard working eldest who became a Methodist minister, to Thomas, intelligent, sensitive and the inseparable companion of Richard, youngest son of all.

Exmoor life was hard, and grew still in the 1860s, and over the next 20 years   over the next hundred, even   men and women drifted away to easier and better paid work in the towns and industry. But Dicky would never be drawn. From childhood he loved country life, and country people fascinated him. His great-nephew said that as soon as Dicky could walk he was "anxious to wander, and it became increasingly difficult for his mother to keep him within the bounds of the farm, let alone the house."

A happy child was Richard, and keen of mind with a country brightness. He always had an answer for anyone trying to stop him, and often it was a humorous one. By the time he was ten he had begun to gain the reputation of a character in embryo. Every cottager and farmer in the parish knew of young Dicky, and at school in North Molton "it was known, and almost expected, that he would stroll in just when it took his fancy". How this free and easy attitude squared with Victorian school discipline is hard to explain, but maybe the status of churchwarden's son conferred a degree of immunity from the cane; or maybe a ready tongue and disarming innocence saw him through.

In his early teens he went with his brother Thomas to the Wesleyan chapel at Molland Cross, which was nearer home than the church, and the habit stayed with him ever after. Quite apart from their religious content, Sundays offered a respite from the otherwise endless routine of the farm; but other breaks in monotony were market days, to which his father often took him, and the bustle and excitement sparked off his comments.

One day in South Molton he was introduced to Parson Jekyll, from Lynton, who did not take over kindly to yeomen in general and had little time for their sons, either; but the twelve year old Dicky was not in the least disconcerted.

Jekyll was a "sporting" parson almost in the class of his colleague Jack Russell for devotion to hunting, and he was correspondingly well mounted, but he did not overawe Dicky. "That be a fine lookin' 'oss you ‘ve got there, passon," he exclaimed in his best Exmoor.

"The finest horse about these parts," replied Jekyll incautiously, taken off his guard by this unlooked for praise. "From a gentleman's stables up Somerset way. Catch the fastest fox west of Dunster."
 "Really now," commented young Dicky. "Well, me veyther has a gurt 'oss, bred vrom yeoman stock, get 'ee to Barum (Barnstaple) quicker than it d' take your 'oss to catch thik fox."
 And Jekyll could find no more convincing repartee than advise him to run and look for his father in case it were time to make for home.

\Flourish 

Young Richard delighted in country life, in its colour, beauty, variety, the stir and excitement of social life and markets and, as he grew to manhood, the interplay of conversation and bargaining in petty trade; but farming made no appeal to him. He often "michied off" from the farm and his parents could never depend on finding him. "Just like the other Richard,"I hear someone say, "Dream a day Dick Jefferies on his farm near Swindon only a few years before." How like   but how unlike in the end. Richard Jefferies, living in a world of his own imagination, derided by his family, but filling his mind with the sights and sounds of Nature and dreaming dreams of what he might do, then fulfilling those dreams, becoming our greatest writer on Nature, but at the cost of exile from his native countryside, illness and an early death. And Richard Slader, happier in the absence of derision, dreaming his dreams also for a while, and finding some little talent as a writer, but kept from discontent and the penalties of fame by the call of his open air life, his parish, his chapel and his Exmoor home.

The brothers moved away one by one till only he and David remained. They did not get one very well. David took over the farm on the death of his father, and Dicky lived there more as a labourer than a partner. But maybe David had some reason and saw himself as having been deprived of at least part of his birthright without a compensatory mess of pottage.

He had always been a loyal son, even if the attractions of the Poltimore Arms had sometimes been too great for him, but his father’s will had been, well, tampered with. Once when the father was ill a lawyer was sent for, and Richard was worried lest David might get a larger inheritance than he did; and as David was in the house when the lawyer came he might influence the father at the critical moment. 

"Never say die," thought Dicky, and took action. He ran to a field in another part of the farm where sheep were grazing, and drove some of them into a field of corn. Then he hurried back to the farmhouse to raise the alarm, and David had no choice but to come out and round them up.

In his absence Dicky put in some powerful persuasion, and when after the father's death the family gathered for the reading of the will they heard with astonishment that Elizabeth had been left £50, Susanna £50, John £25 (if he came from America to claim it), William and Michael £100 each, but Richard £300. David had all the rest and was named sole executor, but evidently in the eyes of both brothers 300 greenbacks in the hand were worth any number of beasts in the shippon.

\Flourish

Dicky stayed on at the farm for some ten years until one night when it caught fire. David was away at the Poltimore Arms and his wife and children had little time to save anything. As for Dicky, he brought away, as he said, only "me top hat, me Bible and me verses".

Verses! Can one imagine that as a product of an Exmoor farm in the 1880s? In fact, it was no more improbable then than now. A French visitor to England in the 1850s, Taine, I think, noted with appreciation the culture of the wives and daughters of many farmers and ruefully contrasted that to the state in his own country. This was twenty or thirty years later, and by then North Devon, thanks largely to such schools as Shebbear College, Edgehill and West Buckland   and of course the village schools from 1879 onward   shared fully in this popular culture. Dicky had been composing his "verses" for quite a few years, a collection of 50 poems which he set to well known hymn tunes. He had a pleasant light tenor voice and could sing without prompting evert tune in his hymnal. In 1892 he had his collection published, "printed by W. C. Coles at his machine printing works, Grenville Street, Bideford". He dedicated the verses to the various churches which abounded in North Devon,   for in those days most people in the Devon countryside, unlike the great cities, went either to chapel or to church.

Dicky did likewise, but not too docilely or uncritically. While still a lad he took one of his ministers to task for what he thought an unnecessarily long sermon.

At the close of the service he complimented him “That was a right good 'un you preached, sir. 'Bout the best I've a yeard in this yere chapel."

"I'm glad you think so, Dicky," replied the minister, who had noticed him taking in every word,
 "Only one trouble," returned Dicky, after a suitable pause, "'twas too long. I could ha' said what you said in half the time. Me backside's proper sore now, jus' as if I'd a rode a 'oss from yere to Bristol."

Whatever the cause of it, his way of life changed around his fortieth year. He and David had little in common and although the farm was being rebuilt, Dicky would often wander off. He travelled around the neighbourhood, stopping for a chat with the farmers' wives and finding that he was welcome. This gave him the incentive to earn his own living in the way that most appealed to him: travelling his beloved Exmoor and North Devon. He took a little roadside cottage near Molland - it still stands to recall his memory - and set up in trade as a pedlar.

That designation, ‘pedlar, does him scant justice. Something in him kept him raised above a lowly, pedestrian status, some sense of innate worth which surmounted the petty shifts of his trade and gave him an essential dignity.

As a Slader he had a proper pride in his yeoman ancestry and in the stories of their activities on the Moor long ago. "I should be up to Lunnon wi' a crown 'pon me 'aid," he remarked; and although certain honoured friends might address him as "Urchard" or "Dicky", to the rest of the world, and to everyone on Sundays, he was, and insisted on, "Mr Slader". He wore rough clothes for the rough work of tramping from door to door, but for the markets and fairs in South Molton, Bampton and as far away as Braunton he would don a faded, old fashioned, high crowned bowler hat and a tattered long tail coat, and he would announce his coming by fanfares on a hunting horn given him by the Master of the Exmoor Foxhounds.

His first stock in trade was produced from his cottage garden and nuts   "nits" to the initiated   and as will be seen, to appreciate the true value you needed to be conversant with the uses of the peddling world.

A hundred years ago nuts formed a staple food, and as they grew plentifully they cost Dicky time and patience but no pence. He carried them around in a sack slung over his shoulder and in a large ‘maund’, together with a pair of scales which were always on the ground beside him by the time the housewife came to the door.

"Fine nits, missis," he might say. "Tull 'ee what, now, you 'on't find wan nit in a poun' that's deeve (empty or withered inside)"   and if you paid the right price of ten pence a pound you could be sure of good value. But if you took a chance and bought the eightpenny sort, well, chance didn't enter into it. You were sure to be "had". One day at Braunton Fair a Barnstaple man   a townie, say no more   bought a pound of the ‘eightpennies’, cracked them open one after another and found every single one ‘deeve’.
 He complained angrily.
 "Whatever did 'ee 'spec?" chuckled Dicky. "They 'm all the swimmin' kind." He sorted his "nits" by the simplest of meththod: dropping them in a bowl of water. Those which floated were "deeve", and these he put aside for selling to bargain hunters. His "regulars" knew better.

To help him on the long round he had a donkey, bought from the rector of Mariansleigh   "bred by passon, so he mus' be proper," said tongue in cheek Richard. He grew fond of the donkey, which served him faithfully for twenty years and became almost as well known as Dicky himself. He would extol its virtues. Such a donkey, he declared, came of a royal line in Jerusalem and had to have seven names   in this case Eva Minnie Mona Frances Adelaide Hamilton Jessie, and therefore referred to as "he". The small boys of South Molton plagued both master and beast unmercifully, but other people treated them with decent respect. At one South Molton fair day Dicky asked the bank clerk if he might bring Eva Minnie inside while the transacted his business. The clerk called the manager, who amiably agreed, "Certainly, bring him in, Mr Slader." Banks in those days, need I say, selected and accepted their clientele with care from a narrow stratum of society, and in South Molton they all thought they knew Dicky and his ways, but the nuisance created made the manager and staff wish they had never set eyes on the animal. But having won the point, thereafter Dicky often led Eva Minnie into the bank and shops on the market square, laden with baskets and all the accoutrements to "his" master's trade.

He had little success with the troublsome boys, however : they were too nimble to be caught. He could only protest plaintively and to no avail whatever: "Yere! Yere now! Tull 'ee what now! Don't 'ee go gwain doin' thik there now! You'll frighten me ole donkaay. You bess wy picky yer wy home along now! You bess wy michie yer wy homeward!"

Generally Eva Minnie could fend well enough for himself, feeding on the roadside hedges except in winter when food was scarce. Then, he would sometimes climb through a ‘shord’ in the hedge and feast on the better grass beyond. One farmer, annoyed by these irruptions, warned Dicky, "Nex' time I see your ole donkey in my fields I'll zhut en daid on the spot."

"Aw, don't ee go doin' thik there, maister," exclaimed Dicky fearfully. "Don't ee zhut me ole donkaay, zir! I' d zo zoone vor ee zhut me a braace o' rabberts an' zend 'em down!"
Or maybe: "Aw yere, maister, you 'ouldn' grudge me poor ole donkaay a bit ' o’ grass, would ee? Tull ee what now, I"ll bring missis up zome blackberries zo's er kin make some jam. Sure 'nough I will. But don't ee carr' on zo about me pore ole donkaay. He 'on't do it again, I promise 'ee."

He made his home in the wayside cottage near Molland Cross, and there, as his reputation grew, he received many visitors. The enterprising photographer of South Molton, Mrs Elizabeth Askew, suggested to Dicky that such a smart man as he should have his ‘likeness’ taken for all the world to see. He liked the idea, and a few days afterwards Mrs Askew and her family came up and took a series of photographs showing him at his fireside or out feeding his chickens or working in the garden. "All that for a pedlar!", some will have thought; but Dicky possessed that quality, so hard to identify or define, which commands attention. He was a "character", and the portraits, made up into postcards, sold in their hundreds over the next few years. Visitors took them home as mementos of a Devon countryman at home; but more significantly, local people bought them too, and although they may not always have consciously realised what they were doing, they kept and valued them, in the word of his grand nephew, "as something to the talked about, something to be preserved from the days when Victorian England was being quickly left behind." 

\Flourish 

Richard's cottage was plain and unadorned, but it was all he needed   though as to simplicity, this soon gave way to all the multifarious stock in trade of pedlardom. He may have started out with "nits" which cost him only the time and labour of stripping them from the hedges, but he soon moved on to articles of necessity which he could buy cheaply and re sell at a small but dependable profit, and so his bedroom became his store for candles, lanterns, soap, chamber pots and pitchenware, together with country brews such as nettle pop and potato or parsnip wine. In season there would also be blackberries and sloes, or potatoes from his own garden, and even a basket of eggs: "I know they 'm fresh, me dear, I laid 'em mezelf."

"In his early peddling days," his great nephew wrote, "he would return early to his windswept cottage, and there he would sit by his wood and peat fire surrounded by his wares, penning his verses, cleaning his nuts for his trek on the morrow, and counting his taking for the day. Before retiring up the stairs he would hang up his hat behind the door, kneel down in front of the stone hearth and say his prayers. His bed was of iron, with two heavy blankets and a pillow covered in serge: "Devonshire serge," he would say, "Vicary serge of South Molton, dyed with Exmoor lichen."

So simple a mode of life accorded completely with the remoteness and quietness of Exmoor a hundred years ago, and although the portrait of the bachelor pedlar, carefully counting over his small daily takings, had little that can be recognised as romantically attractive, there is an element of wistful melancholy that some will find appealing. Richard was a Victorian, and he carried the quietness and unquestioning certainty of the Victorian countryside into the frenzied new century   though in truth the frenzy made no impact on Exmoor for many years.

The pattern he established served him for the rest of his life. For year after year, six days of the week, he followed the simple round and unchanging task (never "trivial" or "common", as the poet incautiously dubbed them, for they satisfied Dicky). Season merged into season, and the fresh green, glowing gold or russet brown of the beechen hedges were his calendar; and although the world outside might be careering toward the precipice of 1914, Exmoor seemed to keep herself inviolate; and if ever Richard had given a thought to it, he would have felt secure from every danger and alarm. But seclusion, even in those days, could not guarantee complete safety, and in time it betrayed him and left him defenceless.

He never married, and as he clearly spent little on his cottage and even less on his clothes, then in the minds of the village gossips he "must be worth a mint o' money", quite regardless of the fact that he earned a hard and scanty living in shillings and ha'pence.   But for gossips and the ill intentioned generally, truth is the least concern and ultimate disappointment; so to their envious minds Richard was a millionaire: his furniture was stuffed with bags of sovereigns, the lonely cottage was lined with coins, and the precious metals it contained weighed more than the fabric itself! However preposterous the rumours, they came to the ears of two tramps who believed enough to determine to get the wealth for themselves..

One morning in 1912 they set out from South Molton and Lynton respectively and met somewhere near Molland Cross, and that night Richard's faithful old watchdog died  "poisoned by them blasted ruffians," he said later, though at the time he put Sammy's death down to old age, and its suddenness, perhaps a reminder that he himself at 55 was no longer in his prime, depressed him. He buried Sammy, "best pal ole Richard ever had," in the garden and consoled himself with the thought that his donkey, Eva Minnie, would never let him own.

Two nights later, as he sat alone in the cottage, the tramps broke in. They bound and gagged him, threatened him with an iron bar and strapped him to the settle, with the warning that his only chance of staying unharmed was to help them. He of course had nothing to say, and although they ransacked the place and made off with all the money they could find, it came to only about £75, which even in those days amounted to no more than a year's wages, certainly not a fortune. The police caught them, and Dicky had his word about the family of one of them: "They was all a rough lot. His brother could eat whit pot till he could touch it wi' his finger, an' drink zider till it urned out both zides o' his mouth", but this gave small comfort for the loss of £75.

Loneliness came over him as he grew older, and as he took less care over his appearance, strangers hesitated to engage him in conversation   to their loss. But he was not deserted: Lord and Lady Poltimore often called when passing this way, for they had great respect for the survivor of a historic Exmoor family. Sometimes Lady Poltimore would tidy up the cottage or mend or patch Dicky's clothes, and when the mood took them, she and Richard would sit and talk for hours together.

Occasionally, moved by some secret emotion, he would reveal an inner, unsuspected Richard to others. A stranger at Heasley Mill once asked him whether his "poor donkey" could pull his little cart up the steep hill. Richard replied with dignity, and in "stranger's English" (for he had two languages, "My donkey's not so poor as you think, and you can tell by his eyes he's not as old as I am. Anyhow, even if he is poor, old and blind, he'll still follow his master home."

In 1925, his 68th year, ill health curtailed his travels, but sometimes in the hot summer he would tramp across the moor to Lynton or down to South Molton market; but his strength was giving way to diabetes, and sensing that this would be his last summer, he mellowed into tranquility, looking forward with "complete happiness and satisfaction" to his last days on earth and his first beyond.

He died in April 1926, and with him there went a vital link with a simpler, rougher but surer world. Since then a few critics have seized on petty details such as his carelessness of dress and appearance or his "keenness" in trade, but who of us would like our petty weaknesses proclaimed to the world? Truer words were written in the obituary written by one who knew and prized him not for the virtues he might have possessed but for the sterling qualities which underlay his quaint characteristics and garb:

 "There passed away on Tuesday a figure long familiar throughout this district in the person of Mr Richard Slader of Molland Cross. His originality and exceptionally retentive memory   he was never at a loss for a birthday or wedding day of any member of the royal family, and could recite the chief events in the history of many prominent people with remarkable accuracy   made him an outstanding personality..... He followed current events carefully, and his opinions, though often quaintly phrased, showed much keenness of perception. As a young man he sang tenor tolerably well, and in 1892 he published a readable booklet of verses.

In order to hear visiting speakers, whether preachers or politicians such as the Hon. George Lambert, "he tramped all over the district at one time or another, and would give an outline of a speech years after he had heard it."

Such qualities, in a wider sphere, might have brought him somewhat patronising recognition. But to what purpose? Exmoor gave him all the life he wished for: the chance encounters of farm, cottage and market place, the sun and storm on the heathland, and the speaking silences of the lonely moor and the quiet companionship of man and beast.

How capricious are celebrity and fame! Thousands labour lifelong to "make a name" by creating works of art, composing music, covering ream upon ream with words, all for the uncertain end of transmitting that "name" to following generations. Richard Slader achieved it without even trying.

\Chapter{General John Mole}{Soldier - Sailor – Grand Old Gentleman}

The Good Soldier, said Paul, must be “able to endure hardness”, and to many young men caught up in the conflicts of the twentieth century, hardness was an unfamiliar companion and endurance a virtue to be learnt. Not so for John Moles: he knew poverty and hard living from the cradle onward. He was born in September 1852, the son of John and Elizabeth in the parish of Exton, on the western slope of Brendon Hill overlooking the Exe Valley. He started out in the traditional hard way for the son of a labourer, bird-scaring at the age of eight for three-ha’pence a day or ninepence a week; and since any crow unable to down nine pennyworth of seed in a week must have been suffering from avian beri-beri, clearly some farmer was doing very well out of John.

A few years’ work drove the lesson home pretty thoroughly, and in 1870 he joined the Army at Shorncliffe. Why so far away from his West Country home I do not know, and the 34th Regiment of Foot, into which he was apparently drafted, recruited largely in East Anglia at this time, but was shortly to become the Border Regiment – that regiment whose valour in the Burma Campaign 75 years later has been so vividly recalled in George Macdonald Fraser’s 'Quartered Safe Out Here'. But perhaps the ‘34’ was merely the reporter’s mis-remembering when he came to write up his interview back in the office; and neither of the two Border battalions served in Ireland at this time. Much more likely – for reasons that will appear later – his regiment was the (Royal) Berkshire, the 66th of foot, who in that same campaign in 1944 shared, with the Royal West Kents and the Queen’s Royal Regiment, the defence of Kohima and at dreadful cost, turned the tide of war towards victory.

But those men, of whom the enemy commander said, “We could not break the British soldiers” were mostly not professional or volunteers, but enlisted “for the duration” of the war, and not beyond; and service for the soldier of the Victorian empire was very different.

For John Moles in 1870, India lay years ahead. First he was drafted to Ireland and stationed in several towns, and then, somewhat later, they sent him with a draft to India, probably for the usual tour of seven years. The regiment was there already: it had been sent in 1870 for an eleven-year stint.

Eleven years! Imagine all the changes back home during that time: year after year of wet summers and poor harvests hastening the decline of agriculture from 1877 onward; set against this the coming of electric light in 1878, the telephone in 1879, electric tramcars in 1883, agitation for Home Rule in Ireland, and the replacement of the old home-based patriotism by an unattractive jingoism as Great Britain became involved in one war after another . The England of the mid-1880s may have worn an air of greater pride and prosperity than ten years earlier, but not all the changes were for the better, nor would John Moles have necessarily found himself straightway at ease.

Besides, in foreign service he had other questions to occupy him, and the most urgent were to keep his vital spark undimmed, his “pecker” up and his skin in one piece. On two of these counts he scored full marks; on the third, well, he was a soldier of the Queen. He fought with the regiment in the Afghan War of 1879, taking part in Roberts’s celebrated march from Kabul to Kandahar; and later, he said, in a two-month war with Tibet he went with a detachment that forced its way to the Forbidden City, Lhasa.

Even to-day with all the evidence of the courage and suffering of British soldiers in Afghanistan, there are certain critics to whom one has only to breathe the words “imperial soldier” to raise a cloud of condemnation. “It was such men as Moles,” they cry, “who by making war on less advanced nations helped to create the spirit of imperialism we now deplore.” 

But much more should be said. Imperial expansion undeniably took place to provide cheap raw materials and new markets for developing industries, and the Opium War inflicted on China was the most flagrant and shocking of these, but that is only part of the story. The Victorians, in the main, were convinced of the superiority of Western culture, art or morality over those of Africa and the East, and while they were slow to recognise child labour in their own factories as an evil, when convinced they put a stop to it. To their simple minds, then, ritual mass slaughter in Ashanti, suttee and thuggee in India and cannibalism in Africa and Melanesia were also evil, and they saw good reason to put a stop to them likewise; and if, as they believed, Zulu kings had compassed the death of a million people in Africa, that also seemed an evil that should be ended.

As always, the right and the wrong were inextricably mixed; but whichever predominated, the Army had, regardless, to carry out the decisions of men who sat at home and enjoyed the comfort of an immaculate conscience or a maculate but pleasurable bank-balance. And it was men like John Moles in the Old Contemptibles who, thirty years later, in the last hours of the old familiar world, “held the sky suspended and saved the sum of things, for pay.”

In The Road to Mandalay the seven- or ten-year soldier who had returned to Blighty from peace-time Burma, remembered inconsolably “the sunshine and the palm trees and the tinklin’ temple bells”, but many British private soldiers in India had other memories, other tales to tell. Their existence in overwhelming heat in a garrison or cantonment , as summed up by Kipling, made for misery. “All their work was over by eight in the morning, and for the rest of the day they could lie on their backs and smoke Canteen-plug and swear at the punkah-coolie.” (He might have added the torture of prickly heat). “. There was the Canteen, of course, and the Temperance Room with the month-old papers in it; but a man of any profession cannot read for eight hours a day in a temperature of 96 degrees or 98 in the shade, running up sometimes to 103 by midnight. Very few men, even with a pannikin of flat, stale, muddy beer hidden under their cots, can continue drinking for six hours a day. One man tried, but he died, and nearly the whole Regiment went to his funeral because it gave them something to do... It was too early for the excitement of fever or cholera. The men could only wait and wait and wait, and watch the shadow of the barrack creeping across the blinding white dust.” – and endure, as Other Ranks, the lack of any contact with a woman of their own nation or race.

In the other kind of station, guarding the North-West Frontier and the Khyber and Bolan Passes leading to Afghanistan, there was no time for indolence, and inattention could mean death. The regiment normally occupied quarters in a border town such as Peshawar and only went up into the mountains on punitive expeditions, to “persuade” the resident tribesmen to keep out of the plains. But even the town gave no safety, for by night a tribesman might creep in to steal a superior British rifle and cut a sentry’s throat on the way. Fighting in the mountains was a skill quickly acquired or else, and standing up to the flashing knives, holding the line, and then driving back the attackers, that came from steadfastness and trust in the mate standing next to you. But lastly, from knowing that surrender could not be thought of. 

British soldiers in Afghanistan in recent years have had to face not only the age-old hazards of shellfire and bullets, but also the new peril of hidden mines detonated by enemies at a distance but inflicting blindness, scarring and loss of limbs. But a wounded soldier of the Empire was also disabled, and frightfully and permanently. “They cut ‘em up ‘orrid,” snapped the drill-sergeant in 'Stalky \& Co'. There was only one way out, and Kipling, again, told it:

“If your officer’s dead and the sergeants look white, Remember it’s ruin to run from a fight; 	 So take open order, lie down, and sit tight, 	And wait for supports like a soldier. 		 Wait, wait, wait like a soldier, Soldier of the Queen.

“When you’re wounded and left Afghanistan’s plains\\
And the women come out to cut up what remains, \\
Jes’ roll to your rifle an’ blow out your brains \\
An’ go to your Gawd like a soldier,\\
Go – go – go like a soldier, Soldier of the Queen.’ 

But far from demoralising the British infantry, the sight of their slaughtered wounded when they were brought to close quarters with Afghan knives man-to-man, gave them a desperate strength to drive back the attackers, so that discipline and years of training prevailed over mad hatred and fanaticism. And after a very few years the action moved the 66th from the Frontier into Afghanistan itself. 

In 1879 the British army in South Africa had begun the year with an unforgettable disaster, when a whole a whole battalion of the 24th Foot was wiped out by a Zulu army of some 10,000 at Isandlhwana. But courageous and highly trained though the Zulus were, and merciless in battle, they were seen by the British as Men, not as devils, and even the official report on the disaster admitted that the Zulus were not guilty of torture.

Later in the year, the British-Indian government, alarmed at the growth of the Russian empire and its nearness to the North West Frontier, sent an expeditionary force into Afghanistan, displaced the Emir, installed their own nominee, signed a treaty, left a very visible presence in the capital, Kabul, and then withdrew to India. Soon, however, the representative, Sir Louis Cavagnari, and his escort were murdered. A rewvolt broke out and spread wide, and the British sent in the largest force yet, some 90,000. commanded by General Frederick Roberts, VC, a brilliant general, bold, and confident in his soldiers, who in their turn trusted him and saw him, unlike almost any general up to Slim and the Fourteenth Army, as their trustworthy leader and friend.

Roberts marched to Kabul, defeated an Afghan army and occupied Kabul early in October 1879. Rebellion broke out and some 10,000 besieged British troops in Sherpur, near Kabul, attacking with ferocious courage but failing in the end. Another Emir was installed, and again rebellion headed by the governor of Herat, Ayub Khan, who felt he had a stronger claim.

In June 1880, when he headed for Kandahar with a force of 30,000 and another 4,000 Ghazi fanatics, a small brigade of 2,500 men commanded by Brigadier Burrows, with the Berkshires as British regiment, was detached from Kandahar to oppose him, and advanced to Helmand. There they were deserted by Afghan levies These were defeated, their guns were captured, and Burrows fell back to a position where Ayub Khan could be intercepted wherever he was heading. And here at Maiwand the little contingent waited in the stifling heat of late July till the 27th brought in a day written in blood and horror, and ever to be remembered in the British army as a disaster redeemed only by steadiness under devastating fire and, at the last, the self-sacrifice of eleven men.

Early in the morning Afghan horsemen were sighted, and the brigade deployed for battle with the 66th and an Indian infantry forming the front line.. A shelling duel went on until the British ran out of ammunition, and during this time, unknown to the British, the whole Afghan army was approaching under cover of deep nullahs and then, with the Ghazis at the head, launched a massive attack on the Indian regiment, rolling it in a great, irresistible wave into the Berkshires. So tightly were they crammed that they could not defend themselves or use their weapons, and their involuntary retreat became a bloody rout. Later, Kipling’s Tommy remembered: ‘I ‘eard the knives be’ind me, but I dursn’t face my man, Nor I don’t know where I went to, ‘cause I didn’t wait to see, Till I ‘eard a beggar squealin’ out for quarter as ‘e ran, An’ I thought I knew the voice an’ – it was me!’ But when the pressure of bodies eased, they rallied around the regimental Colours and fought with their customary professional discipline under heavy fire. Then 56 of them fought their way down to an enclosure in a garden to make a last stand. The Afghans shot them down one by one, but they fired steadily till only eleven of them were left. 

Then these eleven came out to fight their last fight man to man, and all died. An Afghan artillery officer paid them this tribute: ‘These men charged out of the garden and died with their faces to the foe. Their conduct won the admiration of all who witnessed it’.

The regiment had lost 286 dead (including 14 officers) and 32 wounded, but the scattered survivors, no doubt with John Moles among them, had re-assembled by the nest morning. They set out on the exhausting 35(?)-mile trek to Kandahar, harried by snipers, until they met a relief force sent from Kandahar. Ayub Khan’s confidence was somewhat shaken by the loss of 3,000 men in the battle, but he besieged the town until General Roberts brought a force from Kabul in an epic march and defeated him on 1st September. The Berkshires stayed another month in Kandahar and then returned to India; and in 1881, with their eleven years completed, they were brought home. And the war ended with 

The returning soldier with a full tour of India and Afghanistan behind him could fairly hope for as long a home posting, probably in the regimental town, in this case Reading, and no doubt this was the base for John Moules, though in this same year the sadly depleted 66th was paired with the 49th (Hertford), still under the old title but now forming two battalions. But the old Berkshires had little rest, for in the next year they were deployed to Egypt to help put down a rebellion against the Khedive, but also to override French influence and make the country, with the Sudan added, a virtual protectorate. 	 It seems that after the brief rebellion the regiment settled down to the usual peace-time role, but down in the Sudan, smarting under Anglo-Egyptian rule and resentful of reforms, thousands led by the Mahdi, a ‘prophet’ with mesmeric power, rose in powerful revolt, besieging Khartoum, where the English General Gordon governed the Sudan. The British government ordered him to leave, but he refused, and after long delay the Government ordered an expedition to rescue their recalcitrant servant; and the bugles blew for the Berkshires again. 

This Field Force of 13,000 men started in early March 1885 from the port of Suakin, but unlike the little army at Maiwand, the Berkshires were not the only British regiment, for they had good company from the East Surrey Regment and those Shropshire Light Infantryman (to whom A. E. Housman paid such moving tribute: ‘It dawns in Asia – headstones show, And Shropshire names are read: And Nile spills his overflow Beside our Severn’s dead.’) Their trekking should have been straightforward, even if exhausting, because of the knowledge of the Mahdist moods and intentions won, at great personal risk, by the Intelligence Officer assigned to them – none other than Horatio Herbert Kitchener. 

In 1884 the future field-marshal was a middle-rank officer of the Royal Engineers, but his disciplined energy had already marked him for promotion. His first independent command had taken him to Palestine with an archaeological and surveying team, and he had worked his men twelve hours a day, seven days a week, and they responded to his demands and completed the survey several weeks in advance. His total dedication to his profession and his refusal to relax in the company of his fellow-officers distanced him from them but made him the favourite of the higher command, and they had new and important work for him. An officer of his ability was urgently needed down in the Sudan, and he was instructed to move among the Mahdists and find whatever was known about their intended movements – for Kitchener, with his other military accomplishments, had mastered Arabic, spoke it “like a native ”and excelled in negotiations with the Arabs. In the summer of 1884 he was at Korosko, negotiating with certain sheiks with a view to an advance across the desert to Abu Hamed. Both of these towns on the Nile were traditional points of departure for caravans crossing the desert to the south.

A picturesque little legend in Moles’s account was that Kitchener ventured among the dervishes disguised, it was said, as a date-seller to learn their plans, and Moles went along with him, presumably as his servant, his Englishness obscured by ten years of tropical sun and half a gallon of walnut juice. Maybe so. If true, he must have shown the qualities of a senior N C O to have been selected as a companion for such a man on such a task. But perhaps a tall tale was worth telling for its own sake. 

On 21st March the Berkshires, as part of a contingent 3.000 strong, with 1,500 camels and their drivers, marched out, heading for the rebel headquarters twelve miles away; but ‘they quickly found themselves struggling through a dense jungle of mimosa bushes whose low-level branches covered in sharp thorns slowed progress and created havoc among both troops and transport animals.’

This trek needed two days, and after eight miles they halted to build a ‘zariba’, a laager formed from thorny mimosa bushes. ‘Mimosa trees had to be cut down, arranged in line to form the walls of the enclosure, their trunks tied together to prevent them from being dragged away by the enemy. . . Construction was the responsibility of the Engineers, with British and Indian troops cutting the trees and dragging them into position. . . Others were tasked with unloading water and stores and with protecting the working parties from attack.’ Half the Berkshires were guarding the camels and mules, while the other half were defending a redoubt. There was still work to be done, but just before three o’clock the Mahdists – 2,000 of them – launched their frenzied attack, and the Berkshires learnt that a dervish with spear or knife was as formidable and fearless of death as any Ghazi.

A horde of them forced their way into the south redoubt, and the half-battalion of Berkshires formed a square; but their devastating massed volleys were not enough; they came to the cut and thrust of close quarters, and Moles had a triangular piece of flesh knocked out of his head by the butt of a spear. But discipline told in the end against fury, and the dervishes were driven out, leaving more than a hundred dead and wounded.

But in the meantime the main body of Arabs ‘succeeded in stampeding the transport animals into the central square, gaining cover for themselves as they attacked from the east. A British officer recalled, “Everything seemed to come at once: camels, transport, ammunition mules, Indian infantry and Sappers, sick-bearers, Cavalry and Arabs fighting in the midst. The dust raised by this crowd was so great that . it was impossible to see who was standing or what was likely to happen. The men behaved splendidly and stood quite still. It was the highest test of discipline I shall ever see” In fact, of the British and Indian, 70 soldiers and 34 “followers”died, and 130 soldiers were wounded – but of the 2,000 Arabs, as many as probably 1,000 died. In fact this “orgy of destruction” took only 25 minutes, but the steadiness of the Berkshires in the Battle of Tofrek won them the title ‘Royal’. Next day they buried their dead, and then returned to Suakin and then to Egypt.

The Sudan had not finished with them, however, for the victory at Tofrek had not affected the Madhi’s power, and when the British began to build a railway, they had to be guarded all the time. The Berkshires were sent back in Grenfell’s Field Force, and on the last day of the year they took part in the capture of Ginnis, a fortress in northern Sudan occupied by the Cameron Highlanders and besieged by the Mahdists, and won the battle after considerable street-fighting. It was the last battle in which scarlet uniform was worn; then khaki took over. 

John Mole had endured many hardships marching and fighting in the wilderness and the scorching heat. Forty years later he still had a rash from marching through the desert sand, and by the irritation he could tell if the weather was going to change. 

He completed his twenty years and was discharged in Italy. His rank is not known, but an event later on showed him as probably a senior N.C.O, a man trusted for his capability and command. But now he took up a seafaring life, which would give rest for his marching feet if for little else. He shipped before the mast in a barque and sailed to Brazil, Australia, Spain and wherever the winds took him. He went to South America, worked on plantations and became a trader, and then mixed in politics and took part in several revolutions - always on the winning side, apparently, which argues an exceptional eye for the main chance. In Haiti he helped the revolutionaries in a successful coup d’etat -probably that of 1908, and was made a general, and for a time he sat near the top of his particular military tree. But three years later another revolution shook the branch and deposited General Moles at the bottom again.

The time came, however, when a quieter life beckoned. He returned to England, settled down near his birthplace and took up the trade of mole-trapping with a brother from Dublin. For a time it proved quite lucrative, but one by one his relatives died off and he fell on hard times, though he earned a shilling or two picking and mixing herbs to cure his neighbours’ ills, thereby gaining a reputation as a “wise man” of proven ability.
Well into his seventies he kept his upright, soldierly bearing, and his bronzed complexion, perfect hearing and eyesight bore testimony to his healthy open-air living. 

In his working retirement he acquired a new if modest fame, for he became not only one of Exmoor’s acknowledged “characters” but also that rarity of rarities, a sociable hermit. He took up residence in a cave in a disused quarry near Slade Corner, Timberscombe, in a steep hillside facing the main road from Minehead to Exeter. There in his 78th year he was visited by a reporter from the Daily Chronicle, and the visitor found a cheerful old countryman contented with his lot.

His “mansion”, as he called it, was the size of a large room, invisible from the main road and dry even in the worst weather. He had furnished it with a wooden box as seat, a teapot, tin mug and plate, a few cooking pots, a candle and a calendar and several mouse-traps (or were they mole-traps?). A screen of bushes kept the draught from his bed of bracken, straw and sacks. He fetched water from the stream down in the valley and bathed there even in wintry weather.

Such elemental simplicity had its advantages. As he remarked cheerfully, “I pay no rent or taxes.” Life had taught him a working philosophy: “There is nothing like roughing it. I would not exchange my mansion for any house in the country. I never worry about anything, and I am as happy as the day is long. It’s worrying that kills most people. Here I am with a real dry home, good bed, no rent, rates or taxes to pay. Occasionally I get a little job of work from the neighbours, who are very good to me.”

Only a few weeks after those words, alas, he was evicted from his ‘mansion’ by a new landowner, and the ‘union’ eight miles away in Williton seemed inevitable. But John was still a fighter, and in the spirit of Browning’s “one fight more, the best and the last,” he found a shed in an old quarry at Timberscombe. A Mr J Jeffery looked after him as far as possible, and thus, eking out his old age pension in various ways, he struggled on for two years more.

At last nature, with whom he had so long lived in harmony, took him out of the struggle. A “seizure” paralysed him from the waist down. They took him to Williton Infirmary and there in November 1930 he died. He was buried in St Decuman’s churchyard, Watchet, and no stone seems to have been raised to record his name or mark his grave. But here and there in a few minds, even to day, the childhood memory of John Moles the Exmoor Hermit lives on.	