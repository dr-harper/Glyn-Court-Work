\chapter{Susanna Musgrave}
	
History as an art has, with a few dishonourable exceptions, outlived Gibbon’s censure of it as “the register of the crimes, follies and misfortunes of mankind”, and to say that it is a record written by the winners strikes nearer the mark.

But if the winners have the history, the losers have the legends, and no single year in the West Country’s past from the days of King Alfred to the Second World War has given birth to more stories - part true, part imagined - than the Monmouth Rebellion: It may be a mere episode in the history of England, but in the lore of the West Country, and of Somerset, Dorset and East Devon in particular, it occupies the central place. Those three counties gave most of the volunteers to Monmouth’s army and bore most of the suffering and agony of that foredoomed enterprise. The slaughter and misery of that July night 300 years ago still speak to local people with meaning, and the tales of heroism and tragedy, of the Maids of Taunton, the “push of pike” at Philips Norton, Jan Swaine’s Leaps, Dame Alice Lisle, Squire Plumley’s little dog, young Mary Bridge and her sword, still strike a reverberant chord.

The men who marched with Monmouth were not, as used to be thought, merely a horde of ignorant ploughboys and clodhoppers. True, the countrymen were there in some strength, but many others were craftsmen, artisans and weavers - townsmen accustomed to hearing and discussing the news, keeping abreast of the rapidly changing political situation at the court of King James and keenly aware of the threat to religious freedom looming from the persecution of Protestants in France. They were, echoing Cromwell’s phrase, “men who knew what they fought for and loved what they knew”, and if the Duke amounted to less than they took him for, that was no discredit to them.

As to their wives and daughters, they did not, of course, enlist as a brigade or even a platoon of Amazons to fight alongside the men. There was no Joan of Arc, no Phillis de la Charce in Monmouth’s command, nor even a Polly Oliver, and if there had been, most likely his Puritan army would soon have packed her off home again. Nor did he have the customary complement of camp followers, and it is striking evidence of the unique sexual discipline obtaining among the rebels that although they lodged in Taunton 4,000 strong for ten days in June 1685, the parish registers for nine months later record no increase in illegitimate births.

Still, the women of Somerset and Dorset did not lag behind their menfolk in zeal for the Cause, but happily none of them except Dame Alice Lisle suffered the extreme penalty of the law, and two, Susanna Musgrave and Margery Parminter, were heroines of story with unexpected “light relief” and happy endings.

In all the saga of the rebellion there is nothing more winning than the episode of the Maids of Taunton, which brought Susanna Musgrave into trouble but gave her a claim on the memory of posterity.

With Miss Mary Blake (a name revered in Taunton since the siege of forty years before) and Sarah Longford she ran a young ladies’ academy whose pupils gave the Duke an unforgettable welcome. They formed a little procession, about forty in all, Miss Blake leading them in Cromwellian fashion with a Bible and a drawn sword. The girls carried colours, twenty-seven of them, and a girl named Mary Mead proudly held the best, with a crown and fringes and letters in gold: J R - Jacobus Rex . Each of the girls, when she handed over her flag, was saluted by the hero of the hour with a kiss - and given Monmouth’s charm, courtesy and good looks, it would have been a very untypical Somerset maid who hung back.

Six months after the slaughter of Sedgemoor and the butchery of the Bloody Assizes, the king proclaimed a general pardon, but he excluded 178 rebels, among whom were George Speke of Whitelackington, one of Monmouth’s colonels, and his wife Mary, their daughter Mary Jennings, of Curry Rivel, Mary Ball of Wrington, and the 39 Maids of Taunton, whose ransom had not yet been paid, and of course their mistresses, Sarah Langford and Susanna.

Mary Blake’s name was missing, but she had already passed far beyond the king’s vengeance. She had escaped from Taunton, but had been caught and imprisoned in Dorchester jail, and there she died. What became of Sarah Longford is not known, but Susanna, at least, escaped the worst rigours and even found friends to take her part. She fled westward from Taunton, apparently beyond Dulverton and into the recesses of Exmoor, though where she can have eventually found shelter is a mystery. Perhaps her magistrate namesake, George Musgrave of Combe Sydenham, contracted a merciful case of legal blindness and took her in - but the location makes it unlikely.

After a few months, however, as Spring came to Exmoor, she felt she must bring her exile to an end. She had no means of knowing whether the proscription had been lifted, but optimism asserted itself and she set off for home.

She came into Dulverton on 11th May 1686, and evidently her presence in the neighbourhood was no secret. She dined at the Red Lion with Joseph Chadwick, a Nonconformist minister, and with some dozen Dulverton and Taunton friends, and then went on to spend the afternoon at the house of Arthur Case. But George Sydenham, a leading gentleman of Dulverton, was on the trail, and together with the vicar, Parson Lloyd, and the parish constable, Thomas Wilson, he tracked her down and called about five o’clock to arrest her. They found the party “making merry with good liquor and sweetmeats” and in no mood to go quietly.	
The company led the intruders along: “Seize the person if you know her,” they said, and Chadwick, whose calling as a minister must have frequently subjected him to harassing by the law, moved to the attack and told them that if they arrested her they would have to carry her away on their backs - Susanna would not stir a step to oblige them. Then the others joined in with what the vicar called “impertinent discourse”. Arthur Case challenged him with “It were fitter you were in your study” and Chadwick in a heated argument called him a Papist.

But the bullseye for this verbal fusillade was a hapless constable Wilson. Chadwick told him he was “old enough to have had more wit than to meddle in such proceedings,” and then, to preserve Susanna from this now weakened arm of the law, he put her on a settle beside Mrs Chadwick, sat down by Susanna’s other side and called out that the constable should not touch her. The intruders retired discomfited and made a report to the magistrates, but nothing more seems to have been done to bring Susanna within the iron grip of the law - a happier outcome than could ever have been expected.