\Chapter{Dan'l Nethercott}{Scenes of a vanished England}

Only very recently, as the pace of change and built vandalism has quickened to headlong speed, have we come to value the work of the commercial photographers who in thousands of pictures recorded the scenes and people of Victorian and Edwardian England.

As to the local photographers, professional or amateur, who worked alongside the great firms and in healthy competition with them, I cannot say whether Somerset was more generously provided than other counties; probably not, but in the western corner of it we may look over a century of their work, naming almost at random the Hole family for three generations in Watchet and Williton, Alfred Vowles in Minehead and over most of Exmoor, and before them,William Vickery in Luxborough, John Palmer in Skilgate, William Bryant in Brompton Ralph, James Date in Watchet, and here, there and a dozen places else, Daniel Nethercott. 

"Dan'l" came very early on the scene, but not only for that reason is he taking the centre of the stage. He was reckoned, in the classic phrase, "a good workman that needed not to be ashamed", but his life also had more than a touch of romance, and that is an element which may not guarantee an immortal memory but offers a better hope of it than any amount of dull duty dully done. He came into the world in the 1830s at Drucombe, (nowadays most often, though for no valid reason, written and pronounced "Druids' combe", as if a hang out for those hierarchic hatchetmen rather than simply a "deep or narrow valley"). Anyway, Drucombe, halfway between Roadwater and Luxborough, lies in the valley of that musical stream which rises above Luxborough and flows north through the two villages and Washford to the sea at Watchet. Daniel's family had long been connected with Luxborough and enjoyed much respect, but by one of those quirks of parish boundaries so frequent in these parts   irrational at first sight but easily explained   Drucombe Farm stands in the furthest corner of Carhampton parish, and it was then an outpost of the Dunster Estate. 

Daniel's father seems to have combined the roles of tenant farmer and keeper, and he was a man to be trusted, hardworking and thrifty, and therefore better able than most to give his son a "start in life"; and so, when Daniel's working life began he was not sent to the endless work of the ploughlands and fields. He ran errands, took charge of small commissions and so on, at a ha'penny or a penny a time, and soon he had saved enough to buy a donkey. With this working partner he went further afield, increased his takings, and bought a little cart with which he set up as a carrier of small goods. 

So far, nothing marked him as greatly different from other young fellows of his class; but Daniel kept his eyes open and his mind alert, and as he drove over the hill country and down the long lanes to the farms he was struck by their isolation and the solitary lives of the people who lived in them. "Suppose," he seems to have thought, "suppose I can think of something to give them interest, to take them out of themselves. Let me see what I can do." He was a handy lad with something of the artist in his make up, and so he painted a backcloth and set it up in his cart. Then he made puppets or maybe Punch and Judy and their partners in crime, and with these he treated his farming public to a raree show, and the pennies clinked pleasantly in the tin as he passed it round. 

At the same time he trained as a mason, for so he described himself, but at some time probably in his early twenties he hit on the great idea. He had heard of the new method of "taking likenesses" of the human face by the power of light and imprinting them on sensitised paper, and the magic of it appealed to his imagination. There was a practical side, too. These "photographers" practised in the towns, where there were enough people to ensure steady patronage. Why should not he, Daniel, set up a practice in the country? His patrons might be fewer, but they would come to him where they could not have afforded an expensive excursion to town, and the takings would supplement the income from his other work.

Still, he could hardly expect another Somerset photographer to coach a potential rival in the art. He must go to London, where he could learn the latest methods and stay with relatives for the short time needed. 

Imagine him, then, striding down the valley and over the hills on the 20 mile trek to Taunton with a £5 note from his father for the expenses of tuition folded in his pocket book, and see him again a day or so later, a rustic but self confident figure in the West End, bearding a photographer in his den.

The photographer was evidently a decent, helpful man, by no means inclined to take Daniel down a peg or foist him off on an assistant. I cannot quote their exchange word for word, but it went something like this:

Daniel: "I should like for you to learn me to take likenesses, sir. Do 'ee think you could do it?"
 Photographer: "Certainly, but" (with a smile) "I'm afraid it might cost rather more than you would expect."
 Daniel (with an air of victory won and laying his £5 note triumphantly on the counter): "More 'n that, then, sir?"
 Of anyone else the photographer would have asked four or five times as much, but he was "tickled" by Daniel, as they would have said, for the young man appealed to his sense of humour. He chuckled to himself and said, "Well, I expect we could manage something for that, but," as he looked more closely at the banknote, "I'm afraid that note is of no use to me. I don't know the bank"   for the note had been issued by one of the Somerset banks, either Stuckey's or Fox and Fowler's, as safe in their local sphere as the Bank of England, but not known to the benighted metropolitans.

Daniel, crestfallen, left the shop and went to look for a policeman to direct him to the address of his relatives. Within a few minutes he found a helpful one   how London has changed since those days!   and his question was answered in the unadulterated accents of West Somerset. Daniel's rescuer came from Cutcombe! He took him back to the studio, guaranteed the note as genuine, and the lessons went ahead.

The photographer kept his word. He gave his pupil a thorough grounding in the art (or craft? or trade?   something of all three, perhaps, though early photographers such as Julia Cameron and Charles Lutwidge Dodgson were firmly convinced of the first definition and acted upon it). At the end of the course Dan'l bought a camera and equipment with another draft and came home. In the wood near his home he built a little studio and fitted it up with a few pieces of furniture and a backcloth. There he practised his art   and I use the word after due thought, for though Daniel would never have constructed theories of art for himself or paid attention to the aesthetic word spinning of others, he was an artist. He would not have recognised himself in the role, but he was no less genuinely an artist for being unconscious of it. 

Daniel, for sure, only set out with the aim of "taking likenesses", of starting a side line which would "tap" a market and attract the custom of neighbours who, in the few years of comparative prosperity, 1855 – 1875, had a few more pence than usual to spare; but it turned out differently.

Over thirty years there came to him men and women of almost every calling that made up the society of rural Somerset: the farmer, the shepherd, the inkeeper, the labourer, the thatcher and his family, the carpenter, the tailor, the smith and innumerable others, and Daniel "took" them all; and perhaps because he lived so near to them, both physically and mentally or spiritually, he succeeded as no stranger could have done, in rendering their inner selves in visible form.

Daniel, in fact, was a thorough countryman and a country craftsman. Art as practised by the professionals was a world outside the experience and understanding of our Victorian labouring folk. The word itself was too "big" to find a place in their speech, and artists   such as the distinguished trio of painters drawn to West Somerset by the beauty they saw all around   appeared almost as extra terrestrial beings without visible means of support and noticeably casual in their Sabbath observance. 

But they did not need to talk about art. Without knowing it, the craftsmen   the carpenters, wood carvers, blacksmiths and bladesmiths, the cordwainers and saddlers, coopers and wainwrights   followed in the steps of the ancient Greeks, for whom "techne" was both art and craft.

His art, though, could not keep him fully employed, because the few years of prosperity locally (1855 – 1875) were followed by decades of rural depression and whole families left the district for the industrial areas or oversea, and so he also worked at his trade as a mason. But he photographed every scene and event that appealed to him, sometimes regardless of whether it would readily sell, and he created a veritable gallery of portraits of his neighbours, almost as if he were unconsciously laying up for posterity a treasure of memories of a way of life which, though harsh and unjust, bred men and women who would work and strive and endure to the uttermost. .

In his late middle years he moved down the valley, settled in Roadwater and he built his Model House overlooking the meadow where the village revel and sports were held; and there he lived on into a green old age.

He grew more and more patriarchal and dignified in appearance, and this dignity, far from unnerving his rustic clients, struck an answering chord. As we look back across the gulf of a hundred years littered with the wrecks of dreams and innocence, we may at first sense a humorous incongruity in the stolid, unpretentious figures posed against a Renaissance balustrade or Italianate arches; but if we ponder them they take on new, transfiguring lineaments, and we see them not as simple, perhaps quaint figures from the past but as men and women endowed with a dignity which we in our centuries have lost. `

Nowhere is this seen to better advantage than in his portrait of Robert Rowe, the thatcher from Hayne, "In appearance," as was written of him long ago, "a simple, rugged child of Nature:his massive head well poised on a body of moderate height, and a fine forehead arched into a high dome   and with what powers of thought had be but received tuition! There was a full, clear depth in the eyes that reminded you of the one in whom was no guile. You looked into them and saw something that was not of this world, "a light that never was on sea or land." The long furrows on either side of the nose and mouth had the curve of kindliness. The mouth itself was expressive of a strong will, and the lips were wonderfully mobile and sensitive. It was a rugged, weather stained face, but illumined with a gracious light, like a rough granite crag seen in the splendour of the morning sun." 

That physiognomic study, to my mind, was captured and rendered almost in its entirety by Daniel's camera, but I have noticed something else in his seated portraits: unlike us who perch on an upright chair as if impatient to hop and flutter off, these men and women sit in it as if they belong. They almost sit into it, as if nothing can disturb them until the work of posing for the likeness is done. 

 Right folk in the right places

There is something deeply indicative in this, and deeply moving. This solidity and stability transcend the classes of the sitters: the smiths and carpenters and labourers have it as well as the farmers. Poor and oppressed though they often were, and ill rewarded their work, they knew their value to the world's affairs, they knew they had a place in the world that only they could fill, and a task that their brain and hands alone could carry through. Nothing can excuse the wage slavery and exploitation they endured for so long, but they retained a pride in their work and sense of being needed individually, and these were qualities that the countrymen who went into the mills and factories all too quickly lost. The outside world, then as now, brimmed with violence, bloodshed and oppression, but they heard little about it. Life moved with the deliberation of eternity, and by the same rhythm they governed their daily lives. Whether prosperous or poor, calmness and strength reside in their faces, and we still see them standing erect and certain of their place in the world, or seated foursquare with the confidence of unquestioned wealth, the riches of a contented mind.

\Flourish 

Even that was not all. Thanks to Daniel and such early photographers, these men and women were the first of their kind to be able to transmit a record of their physical appearance to posterity. A landowner might contemplate with mingled pride and discomfort the portraits of ten generations of his ancestors, but the face and gestures of a ploughman could not survive beyond the memory of his grandchildren. Now that was changed. Even a labourer could hope to be remembered, and Richard Jefferies' lament that "the faces fade as the flowers, and there is no consolation" would lose a little of its poignancy. So it is not surprising that the "old people" approached the taking of a "likeness" with something of a sense of occasion, a matter to be undertaken with forethought and solemnity. "Well, Annie," one old man was heard to say, " us'll have the likeness, an' if I do die fus', thee shall have en, an' if thee d' die fus', I shall have en."

It is sad to record that much of Daniel's work was lost or, worse, jettisoned as so often in the Edwardian village by a posterity that did not value what it stood for. In late middle age he moved down the valley from Drucombe and, with his mason's skill, built himself a "Model House" in Roadwater. He trained his son Rudolph in photography and one or two of the son's photographs still exist. Whether the son predeceased him I do not know, but Daniel died in 1918, full of years and, one would have hoped, honour, but hundreds upon hundreds of his precious glass plates were thrown out, used for cloches or tipped into a ditch. A few were saved by a minister home on leave, but they cannot represent more than a small fragment of the work of the best part of a lifetime.

Even so, enough remains for us to value what was done by Daniel Nethercott and others of his art to raise the dignity and self respect of a class who for the first time could call more than their souls their own in such quiet places as the valley of the Brendon stream.