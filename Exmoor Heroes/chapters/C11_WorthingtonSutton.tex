\Chapter{Worthington Sutton}{Lime-burner and Salvationist}

The wags might chuckle at Worthy Sutton's 'moker car', but unlike the new fangled automobiles it would never let him down. As long as one of his three donkeys, Traveller, Jacker and Jimbo, could pull the little contraption, Worthy could travel from his cottage in Bilbrook to the kilns at Warren and home again six days a week, and his four footed fellow travellers never suffered ill- treatment or neglect.

The dates of his life are unknown to me, though I imagine he lived from about 1840 to 1920. Be that as it may, he made his mark on the neighbourhood and impinged for good on many lives. Once you had seen Worthy, you would never forget him, and familiarity bred only respect. "In stature," wrote an admirer, "he was just above the average height and of spare build, with a weather beaten face of pronounced features. From the hard life he had lived, exposure to the weather and the fierce heat of the kilns, his skin had become parchment like and his eyes were filmed, one being nearly blind. His whole career had been one grim fight with necessity and the elements, and so at first sight he impressed one as being a hard man, with a snap of the jaw that showed great determination. But closer acquaintance revealed his true nature.

"Far from the shrivelled and unyielding type his outward form suggested, he was really a responsive soul, with a great vein of tenderness and much pure gold of moral quality. I have seldom known a character whose countenance more belied him. His hands too were horny and burnt by the lime he handled, until to all appearances little of the human touch was left in them; but we who knew his story could form a juster estimate of the man.

"His week day clothes were also far from attractive. An old green jacket that once long ago had been black, a pair of stiff fustian trousers and an apron made of sackcloth, such was his workaday attire   and what did it matter when he stood at his dusty work by the kilns? But on Sundays and special days the transformation was remarkable, though the better attire only served to accentuate the parchment like features and the hard and grimy hands. But week day or Sunday, the surprise of the man with that hard exterior was his gentle and almost feminine voice which conversed with you in a quiet and forceful manner that set you quite at ease and inspired confidence and esteem; and more remarkable still, whatever the subject might first have been, the conversation was soon made to serve the purpose of religion   for Worthy was an intensely religious man", and in the light of his experience he put into practice the maxim, 'What we have felt and seen, With confidence we tell.'

Not for Worthy, however, a cassocked and strait-laced religion. His village. a mile distant from the parish church, `had never had one of its own, but in the 1850s his widowed mother had opened her cottage for preaching, probably to the Wesleyans, even though harbouring Dissenters meant the risk of losing employment and the certainty of social estrangement; and services continued there for half a century. 

But the Wesleyan services, earnest but respectable and conventional, left Worthy dissatisfied. "His heart," it was said," was bigger than his sect, and.as he grew up he desired a greater freedom of spirit than some of his co religionists were prepared to concede;and so when first the Salvation Army found their way down into our quiet countryside with their red jerseys and their warm hearted appeals, the movement captured Worthy and held him."	 

Other influences, meanwhile, had also helped to mould him, and a friend surmised that while the harsh nature of Worthy's employment had formed his outward appearance, the splendour and freedom of his surroundings had worked no less decisively on his inner life. The place of his employment lies off the main coast road from Watchet to Minehead and can only be reached by a track; but this is how, in Worthy's time, it impressed the friend: 

"It was a romantic spot, the great limestone cliffs towering up some 300 feet above the shore and exposed to the fury of the storms that would sweep up channel from the south west. A century or more ago it was well inland, but so serious has been the erosion of the coast line that two whole fields have disappeared, so the ancients used to tell me, leaving the great kilns on the very edge of the cliffs... Often in my boyhood I would wander out to that breezy promontory and watch the lime burner at his work.... And how the larks would sing above those rugged cliffs and the white seagulls wheel in and out the cavities below, while across the broad reach of Severn the coast line of Glamorgan for fifty miles shone bright in the sunlight, and far beyond the Cambrian mountains lifted their dark grey masses into the sky! 

"It was an inspiring scene for a thoughtful man, with a seventy mile stretch of the sea beneath him, a rare vision of the Exmoor and Quantock hills, and all the enchantments of Nature in the wild. There she tried her many moods, from the soft Spring breezes to the mighty raging storm. I have seen the butterflies sporting on those cliffs in the sunshine and I have seen them swept with tempests which smote terror to the hearts of the old sailors living at the foot of the hill and drove ships up channel with their canvas torn to shreds, and scarce a living thing could stand before the ale. It was a great place for a man to earn his daily bread in ... and Worthy was not insensible to the appeal of it." 

One can almost believe that, while he observed the beauty of sea, earth and sky and their inhabitants on the cliffs of Warren and pondered their evidence of creative power, the winds blew away whatever in him was trifling and formal. If the somewhat predictable Wesleyan services had left him dissatisfied, he responded readily to the early Salvationists in Watchet and greatly admired their courage and vigour. Soon he bought a red jersey, and at the risk of his neighbours' ridicule he wore it, an unconventional act among a then most conventional people. Hard work and he had always been the best of friends, not to be separated even on Sundays, and that so called "day of rest" found him as busy as on any other, travelling a large area to visit the aged and sick, to cheer them up and, at need, pray with them. 

No one now will recall the person of Worthy Sutton even as a distant childhood memory, but a few years ago his nephew, the late Mr Ernest Binding, told me one of those little tales which bring men and women of the past vividly to life. 

It must have been in September 1914, and young Ernest had gone over to Bilbrook, as he often did, to visit his old uncle and read the newspaper aloud to him. They were anxious days, for both the British Expeditionary Force in France and the French army were being pushed remorselessly back by the German onslaught, which had now reached a point only forty miles from Paris. There were no doubt local boys in the B.E.F., and something of the anxiety must have crept into in the lad's voice, for suddenly the old man said, "Now don't you worry, Ernest my son. The Germans won't take Paris. I've been talking to Father about it" ( 'That was the way he spoke,' said Mr Binding) "I've been talking to Father about it, and they won't take Paris."   "And sure enough they didn't,' Mr Binding added “A few days later they were held at the battle of the Marne."