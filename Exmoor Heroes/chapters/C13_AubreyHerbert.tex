\Chapter{Aubrey Herbert}{The Uncrowned King}

Of all the perversions of plot that film-makers have foisted on a tale of adventure, none strains belief more than the sight of Richard Hannay in The Thirty-nine Steps shackled to an unwilling lady companion , trudging over the mist-laden moors for a weary day and bedding down for a drenched night without either of them apparently needing to attend to either of the humblest needs of nature. But there, as John Ford said of his Stageoach, “If the Indians shot the horses, you wouldn’t have a movie’” and so John Buchan and the Steps keep their popularity. Fashion decrees a suave hero in his twenties, while the real Hannay, turning forty, was a tough mining engineer with a hard-won fortune and connections with men of power and influence – and as tough as himself.

Buchan’s heroes, Hannay, Lord Lamancha, Sandy Arbuthnot and the tragic and admirable Afrikaner, Pieter Pienaar, to whom must be added the real-life Aubrey Herbert, move in a world now almost vanished. They are directed by a code of honour all the more compelling for never being spoken aloud; and if their ideals seem alien today, the misfortune is not theirs. In times of peace they enjoyed a life of ease chequered with the cares of country estates; in war they governed or fought, but often they were sent on hazardous missions, disappearing for weeks and months at a time.

Such a man was the hero of Buchan’s Greenmantle, which appeared in the dark year of 1916, soon after the disastrous expedition to Turkish Gallipoli. Greenmantle is Sandy, the brilliant linguist, working in disguise as a charismatic figure in eastern Turkey to foil German plans for a vast Muslim uprising – and the timing was perfect. Those who read The Thirty-nine Steps of the year before remembered that Hannay’s adventures had started when he gave refuge to Franklin P. Scudder, ‘who acted as correspondent for a Chicago paper, and spent a year or two in South Eastern Europe ... He was a fine linguist, and had got to know pretty well the society in those parts. He spoke familiarly of many names that I remembered to have seen in the newspapers,’ and ‘told me some queer things that explained a lot that had puzzled me. – things that had happened in the Balkan War, how one state suddenly came out on top, why alliances were made and broken, why certain men disappeared, and where the sinews of war came from’. It all went to confirm Hannay’s idea that ‘Albania was the kind of place that might keep a man from yawning’, and when Scudder, in half an hour, transformed himself from the eager, amiable American into a captain of the Gurkhas who had clearly seen ‘long service in kharab stations,’ wore a monocle, 'carried himself as if he had been drilled’ and shed every trace of an American accent from his speech, the new figure was near enough – saving the gimlet-like blue eyes – that of a hero familiar by name to the public for his recent adventures in Albania and the Balkans.

And so, when in the following year Sandy Arbuthnot appeared as Greenmantle, they surmised that he might not be a wholly imaginary figure but a Scudder naturalised, so to speak, and re-born : T.E. Lawrence, perhaps, but more probably – and with reason – Aubrey Herbert, nominally of Pixton Park in West Somerset but just as likely to be anywhere between Gibraltar and the Persian Gulf.
 
The Honourable Aubrey Nigel Henry Molyneux Herbert had come into a self-confident world in 1880 amid the palatial splendour of Highclere, as the second son of the fourth Earl of Carnarvon, with the world open wide before him. But the silver spoon held a drop of gall, for his eyesight was so weak that he could only read by placing the page close to his right eye. This virtual blindness exposed him to bullying and ill-treatment at Eton, but he wasted no time in idle complaint, and threw himself into life with the vigour of one with perfect health. But before going up to Oxford he was taken to Wiesbaden where Professor Piel , the leading eye-surgeon of Europe, operated to give him distant vision in his left eye. From then his career at Balliol followed a familiar pattern of baiting deans and proctors, mainly by scaling the walls of university buildings and climbing up on to the roofs or taking or risking his neck in impossible leaps; but he was also awarded a first class history degree, and a list of the men who gave him their friendship indicates the kind of man he was to become: Hilaire Belloc, John Buchan, Raymond Asquith and Carton de Wiart. As a boy he had travelled over much of Europe and thus, according to his obituarist, “developed the wanderlust that dominated his whole life.”

But Aubrey Herbert was no mere globe-trotter. He had extraordinary qualities – perhaps inherited qualities if considered in the light of the involvement of his half-brother Lord Carnarvon in Egypt and its ancient civilisation – and these qualities and talents, courage, vision and a remarkable gift for languages, fitted him for an extraordinary career. 

After Oxford he went into the diplomatic service and was appointed at the age of 22 honorary secretary to the British Embassy in Tokyo, but never felt at ease there. Two years later he was transferred to Constantinople (modern Istanbul), and there he learnt Turkish, Arabic and Greek – the first of these so well that when after the 1914 – 1918 War he accompanied a Turkish envoy to Germany, he was taken for a Turk belonging to the mission. He twice visited and toured in Albania, and conceived a great enthusiasm for that turbulent little country and its redoubtable people. He learnt Albanian, Greek, Turkish and Arabic and spoke them all fluently, so that in his travels in the Balkans and, later, Asia Minor, he could talk and question freely without having to depend on an interpreter. 

“Effortless mastery” is the phrase commonly applied to such men as Aubrey Herbert, but Nature, it seems, exacts an inexorable if delayed price from them, as if acquiring mastery of knowledge without labour does not fit in with the scheme of things. And after all, if physical achievement inevitably takes its toll of the body, why should it not of the mind as well? An old Roman proverb asserts that “those whom the gods love die young”. Like much of proverbial wisdom it can be disproved as often as proved, but tragically it held firm for Auberon Herbert. 
 

In 1910, at the age of 30, he married Mary, daughter of the Viscount de Vesci, nominally a member of the Protestant Ascendancy in Ireland but actually a Catholic. To start them off in married life Herbert’s mother gave him a country house and estate of 5,000 acres at Pixton Park, near Dulverton on the fringe of Somerset Exmoor, and a substantial villa at Portofino, and his mother-in-law gave them a fine house in Grosvenor Square. Then, as a landed Somerset gentleman, in November 1911 he entered Parliament as MP for South Somerset, but even a happy marriage could not keep him away from adventure. In the Balkan Wars of 1912–1913 – a trial run for the slaughterhouse, so to speak - he took part in the fighting between the Turks and the Albanians, most probably because he was friendly with the young reformers of the ramshackle Ottoman empire.

The Albanians captured him, but so completely did he win their confidence that, strengthening their bid for independence from Turkey, they offered him the crown of Albania, For family and estate reasons he could only refuse and then the affairs of Albania were swept into the maelstrom of the European war. .

Long before this he was back in England and held a commission in the yeomanry, but when war broke out he volunteered for service overseas and nine days later he was in France, not with the yeomanry but with the Irish Guards. 

This was not a simple cross-posting. To his chagrin, the Army had found his eyesight too poor for the death-or-glory department and he had no mind for a safe staff billet; and so he bought a service uniform, waited for a draft of the Guards to embark, slipped casually in among them and crossed with them to France. It says much fo his charm and self-confidence – and not a little for his social influence – that the Guards accepted their unexpected recruit with hardly a word of protest, and got round the problem of his presence by appointing him interpreter and therefore not taken on the active strength. But “not active” was only a form of words, for even in these early days of the war the mass attacks and industrialized slaughter set the pattern for the next four years. The British army waged a fighting retreat day after day until they reached the defensive line of the River Marne, and their losses were such as no one could ever have contemplated. Th Queen’s Royal Regiment, for instance, landed in France on 22 August, 800 strong; after two months only 38 were left. 

Herbert, thrown into his maelstrom, survived and more. He interpreted and fought and rode with battlefield messages as if his eyesight were normal. He was wounded at Mons, expected to be bayoneted but was treated with rough kindness by a German sergeant and taken prisoner. He was soon exchanged and sent home to recuperate, ... 

This time the Army was not to be circumvented, even in this patriotic cause. But the high command,for once, recognised what a weapon they had in their hands, removed him from the bloody stalemate of the Western Front and sent him to where he could be most useful and beneficial to Britain: to the Middle East, and in the next two years he ranged over Gallipoli, Egypt, Palestine and Mesopotamia ruffling the feathers of self-important garrison officers but winning hearts and minds to favour the Allied cause. .

He had to be allowed to work in his own way, and his attitude to the people of that part of the world was far from typical of his nation. To this he owed the success of his missions, though, like T. E. Lawrence, with whom he collaborated, he and they were betrayed by the politicians who had sent them out. His friend Desmond MacCarthy said of him, “He had little personal ambition . . . He loved to serve and be “in it”, but to be “in it” with – the contradiction suggests him – a devoted detachment. It was impossible that such a man should not have been up to his neck in the war, and that he should not have instantly put his life and energies at the service of that cause and seen it through. But. . . . he reserved independence of thought and feeling, the right to admire and even love an enemy, to appreciate, when he saw it, his point of view, and, on the other hand, to criticize anything he deplored in the attitude of his own countrymen. He was the antithesis, therefore, of the “party man”, even in patriotism and war.” And more than once, when he met men he had known in Turkey or the Balkans before the War, his generous open mindedness reached out across the gulf of war and found an understanding response. 

“No one could ever doubt his good faith ; it was a talisman he carried with him into all companies and into different parts of the world. . . . . Only where his candour encountered a self-protective duplicity did it fail. He took it for granted, until evidence to the contrary was overwhelming, that everyone, whoever he was, with whom he had to deal was what is often called “a gentleman”. This supposition worked better in old semi-civilized communities, in Turkey and Albania at that time, for instance, than at Westminster; and better in rural than in commercial England.”

One may mark as one example, probably the best, of his humanity in war and the response of the Turkish enemy, both soldiers and civilians, to one who treated them as equals. Herbert was appointed liaison officer with the ANZAC troops at Gallipoli, where the enemy commander was Mustapha Kemal, later known as the founder of modern Turkey. Herbert had met him before the war, and they agreed on an eight-hour truce on Whit Sunday 23 May for the burial of the dead of both armies, but it was Herbert’s task to oversee the truce and guaranee the safety of the burial teams by his presence. He wrote: “There was some trouble because we were always crossing each other’s lines. I talked to the Turks, one of who pointed to the graves. “That’s politics,” he said. Then he pointed to the dead bodies, and said: “That’s diplomacy. God pity all of us poor soldiers!” “Much of this business was ghastly to the point of nightmare. I found a hardened old Albanian chaoush (sergeant) and got him to do anything I wanted. Then a lot of other Albanians came up, and I said : “Tunya tyeta.” I had met some of them in Janian. They began clapping me on the back and cheering while half a dozen funeral services were going on all around, conducted by the chaplains. I had to stop them. I asked them if they did not want a Imam for a service over their own dead, but the old Albanian pagan roared with laughter and said that their souls were all right. They could look after themselves. Not many sins of fanaticism….

‘I retired their troops and ours, walking along the line. At 4.7 I retired the white-flag men, making them shake hands with our men. Then I came to the upper end. About a dozen Turks came out. I chaffed them, and said they would shoot me next day. They said, in a horrified chorus, ”God forbid!” The Albanians laughed and cheered, and said, “We will never shoot you.” Then the Australians began coming up, and said, “Good-bye, old chap, good luck!” And the Turks said, “Smiling may you go and smiling come again.” Then I told them all to get into their trenches, and unthinkingly went up to the Turkish trench and got a deep salaam from it. I told them that neither side would fire for twenty-five minutes . . .A couple of rifles went off a few minutes later, but Potts and I went hurriedly to and fro seeing it was all right. At last we dropped into our trenches, glad that the strain was over . . . . . I got some raw whisky for the infection in my throat, and iodine for where the barbed wire had torn my feet. There was a hush over the Peninsula.” 

One cannot in these few pages give an account of all his travels and missions, but one can at least say that whenever affairs in the Near and Middle East demanded a skilled negotiator needing no interpreter, the British Government chose Aubrey Herbert – even if, knowing his honour and incorruptibility, they either did not tell him what had been decided behind the scenes or challenged the report he made on his return to Parliament. 

The meeting with the old chaoush at Gallipoli had kept clear in his mind the needs and ambitions of the people of Albania, briefly free and independent after the Balkan war but since then occupied by foreign armies. In November 1915, after sick leave in England, he was sent to Paris and Rome on a secret mission related to Albania, though six months earlier, to persuade Italy to join them in the war, France and Great Britain had signed the secret Treaty of London, deciding, among much else, after the war. Central Albania could exist as a rump state, but that Northern and Central Albania could be divided between Serbia, Montenegro and Greece, and that Italy would not object. He seems to have been kept completely in the dark about this, and quickly tiring of the futility of his role here, and relieved by the War Office, he took up an offer from to serve as Captain of Intelligence in Mesopotamia (now Iraq) and the Gulf.

Mesopotamia was roundly cursed as “Messpot” by the British soldiers suffering with their Indian Army comrades the appalling consequences of an ill-provisioned and incompetently conducted campaign. In brief, an Indian Expeditionary Force of British and Indian troops including the 1/4 Somersets and 1/4 and 1/6 Devons and commanded by General Townshend, had advanced from Basra up the Tigris to try to capture Baghdad, but following defeat by a Turkish army they had been under siege in the town of Kut-el-Amara since December 1915 and two relief expeditions had failed. The War Office, seeing no rescue possible, offered Herbert’s services to General Townshend to negotiate with the Turks, assisted by Lawrence and a colonel of the Indian Intelligence. 

Lawrence wrote with mingled admiration and bitterness, and Herbert would have agreed – and the words have lost none of their bite: “I went up the Tigris with one hundred Devon Territorials, young, clean, delightful fellows, full of the power of happiness and of making women and children glad. By them one saw vividly how great it was to be their kin, and English. And we were casting them by their thousands into the fire to the worst of deaths, not to win the war but that the corn and rice and oil of Mesopotamia might be ours.”

To salvage whatever might be brought out from the disaster, the rescue team were instructed to oversee the exchange of prisoners and wounded and eventually offer the Turkish commander up to £2 million to give up the siege. The Turks, though, were not to be bought. Too few boats could be had to evacuate the wounded, and the defenders of Kut were taken prisoner. Of the nearly 12,000 who left Kut on 6 May 1916, 4,250 died from brutal treatment or debilitating illness on the march to Anatolia or of hard labour in the prison camps awaiting them there. Herbert, was outraged by the brutality of these rear echelon Turkish troops, so different from those he had known in the front line at Gallipoli, but he was just as indignant at the Allied mismanagement of the campaign, and he sent a telegram reporting what he had seen and heard and condemning the incompetence, to Austen Chamberlain, Secretary of State for India. The Government of India ordered a court martial, but the War Office, with determination and wisdom, refused to give him up. He came back to England for a short leave and recuperation, but idleness was not for him and he returned to Parliament for a few months to make his voice heard in persistent demands , eventually successful, for a Royal Commission on Mesopotamia.

By October the authorities had found a plausible way to be rid of this troublesome but prominent officer critic for a while and he was appointed liaison officer with the Italian army in Albania, which was in he Italian front line against against Austria. .. The post should have been ideal, and he for the post, but what he did not know was that Albania was already betrayed. In the secret Treaty of London in April 1915 the major Allied Powers, to bring neutral Italy into the war on their side, had promised her certain areas of the Austrian Empire as a reward afterwards; Serbia and Montenegro would also gain;, but as for the nascent state or Principality of Albania, the carvers were at work , and Article 7 stipulated that “if “ - the very word breathed treachery - :”if the central portion of Albania is reserved for the establishment of a small autonomous neutralised state, Italy shall not oppose the division of Northern and Southern Albania between Montenegro, Serbia and Greece”. When this came to light, it was too much for Herbert’s honour. He resigned his post and was back in England in December, after less than three months away.

One can only assume that his parliamentary status allowed him such freedom of action and movement. Presumably he prevailed on the War Office to allow him to return, or else he would surely have been court-martialled, and any “ordinary” officer allowing himself to be ruled by his conscience would have jeopardised his career and  his commission. But again, Herbert was a far from ordinary soldier, and the War Office, far from being the club of dug-out dunderheads that critics supposed, contained many senior officers who had learnt the hard lessons of an unfamiliar war and valued rightly the service that this Irregular of genius could provide. 

At this time he also heard of the death of his cousin Bron, the son of the pacifically minded Auberon Waugh, and from then on he consistently supported the idea of a negotiated peace, though how Germany could be persuaded to give up her conquests and not use an armistice to re-arm. was a problem demanding more than “the best will in the world”, a quality difficult to find on either side in 1917.

For the time being Aubrey Herbert’s Balkan wanderlust had to be stifled as he worked with the Director of Intelligence, General MacDonogh, on plans for a separate peace with Turkey. On 16 July he conducted a series of negotiations with Turks in Geneva, but this promising venture petered out in the stress of Turkish internal politics. turmoil of Turkish internal rivalries. Mustapha Kemal, whom he had met in Gallipoli two years before, had become a military hero but with political ambition and a commitment to reform which would soon be realised; but for the moment power rested with the Sultan’s chief minister, Enver Pasha, who, on the evidence of he Western Front, could gain nothing by breaking from Germany. Six months later, when General Smuts was sent to Bern to meet Turkish representatives, his chief negotiator, Philip Kerr, warned him, ‘Enver is a pure militaristic Germanophile, having no ideas on the future save that Germany will win the war and recover the Turkish empire, of which he will then be Dictator or Sultan. Talaat Pasha, (the Grand Vizier and prominent leader among the Young Turks), is also Germanophile but is now in an uncertain frame of mind and willing to go along with the winning side” (Lloyd George: War Memoirs, II, p.1504). In the long run defeat in war decided for them both. 

Just before the above negotiations , in November 1917, Herbert was again sent to Italy, where he was in charge of the British Adriatic Mission, coordinating special intelligence in Rome. And now Albania came again into life, Over in America expatriate Albaniana , leagued as ‘Vatra’, proposed raising an Albanian regiment that Aubrey Herbert would command. This promised well for the little state, but the matter was somewhat delicate in that as Italy's interest in a post-war Albania seemed to be becoming more proprietorial than benevolently neutral, ‘Vatra’ became more and more anti-Italian. Eventually, on 17 July 1918, the proposal was ratified in Boston, and the Italian Consulate accepted, on condition that the Albanian regiment served as a unit in the Italian army. By that time, of course, the European was was dragging to its bloody close. Herbert finished the war as commander of the British mission to the Italian army in Albania with the rank of lieutenant-colonel. He felt obliged to turn down the temping offer,and the plan foundered in the mire of postwar confusion and broken promises.

The armistice of November 1918 with Germany brought no genuine peace or reconciliation, and the Paris Conference Conference of the next year might be seen as a continuation of warfare by other means. Certainly it brought bitter disappointment and frustration to the British agents in he Near and Middle East and to the Arab tribes and families whom they had persuaded to revolt against the Turks with the prospect of Arab rule for Arab lands, only to find that those lands were denied independence and became “sones of influence”, British in Iraq, Transjordan, Palestine and Egypt, and French in Syria and Lebanon. 

By May 1919, however, the Directorate of Intelligence had been turned over to Scotland Yard Special Branch because of the perceived threat of Communist activity in the industrial areas, and thus Herbert found himself conducting interviews with foreigners in the somewhat unreal guise of a Scotland Yard inspector.

In February 1921 he was sent to Berlin to interview Talaat Pasha, the former Grand Vizier of the Ottoman Empire, who had escaped from Constantinople in a German submarine and had remained in Germany ever since. An court-martial trial of 120 Turkish politicians in Constantinople had condemned hum to death in absentia. He had held untimate responsibility for the deportation and massacre of the Armenians in 1915, but the British were more worried by his present activities He might be in exile, but changes were taking place in Turkey which might let him return home. The country, under Kemal’s leadership, was recovering self-confidence, the government refused to accept the terms of a peace treaty which would deprive Turkey if its foothold in Europe and hand over Smyrna to Greece, and the army was ‘spoiling for a fight’. These were encouraging signs for Talat Pasha, and he was known to be seeking support from Moslem countries to raise serious opposition to the Allies .On top of that, he “dared threaten that he was going to incite the Pan-Turanist and Pan-Islamist movements against England, unless she signed a peace treaty favourable to Turkey.x Herbert interviewed Talat in Wiesbaden for three days, thoroughly enjoying the conversations in Turkish and the sparrring matches with the eminent enemy, for the openness and affability that were his very nature made all thought of a cold interrogation unthinkable. They were on the best of terms, and this led to a curious and significant encounter. They wee travelling by rail and conversing amicably until the train stopped at a station and a third passenger, a German, got in. After a while, recognising Tallat, he told him – in German, of course – that he had served with General Sanders’s mission in Turkey four years before. Tallat replied politely in German and a conversation ensued, edging aound to all the ills that Germany had since suffered. Herbert gave no sign that he understood the German and took no part in this converation other than a few remarks to Tallat, always in Turkish. The German, quite convinced that both his companions were Germany’s former allies and that he need not fear betrayal to the Occupying Power, loosened his tongue and let his indignation speak unbridled : “Germany has been beaten by England and France . The English fought well and I do not complain of them. But the French – they won by treachery. They can never, never be forgiven. Yes, we are beaten now, but we will be strong again, and we will return to France and be revenged!” Then the train stopped and he got out with a few words of farewell.. “You see,” said Tallat. “ You speak Turkish like a Turk, you convinced him you are one of us. He would never have spoken openly else.”. As for Tallat’s fate, however, nothing came out of Herbert’s interviews to counteract the decision already taken behind the scenes – and surely unknown to the honourable interviewer – that Talatt should be deleted. And only a week later, when he returned from a short visit to Stockholm, he was conveniently murdered by an Armenian in revenge for the atrocities he had witnessed six years before. 

Together with these activities Herbert had returned to Parliament, though not with the clear run that his record of service, both in Parliament and in the Army, should have given him. Indeed, it was his freedom from political sectarianism that caused the trouble. By upbringing and inclination he was a Conservative and sat under that label, but he had many friends among the Liberal MPs and the Prime Minister, Herbert Asquith, and his family, and he voted freely as his conscience directed him, regardless of the party whips. Back in 1916 he had outraged the civilian warriors in his constituency by his support for the idea of a negotiated peace with Germany, and it had taken all the courage, steadfastness and determination of his Mary to fight them off. Fortunately she was equal to it, but for the new election it was only with difficulty that he managed to obtain his party’s endorsement of his candidature. When the new Parliament assembled he surveyed them with sinking heart : so few of his old friends, whether Tory or Liberal, had survived the war and the election, and their places had been taken by the infamous “hard-faced men who had done well out of the war.” 

In his twelve years as M P he spoke or asked Parliamentary questions more than 450 times, a quite remarkable record considering his long absences from the House and country on national business., but not likely to endear him to whips who, far from counting on him “voting at his party’s call”, knew that he was just as likely either to abstain or vote with the Opposition . For a number of other reasons he never made the name of a great Parliamentarian : Members as independent in mind as he were few, far too few to form an alternative power base; and the matters which most concerned him, the fate of the defeated nations and the broken promises of the victors, meant little to the others. He took no pleasure in Parliamentary debating, either. One-to-one persuasion by the charm of his manner, by his eagerness and conviction, that was his forte ; he had no relish for scoring petty debating points or humbling rivals. To be devious or ruthless was beyond him, beyond his understanding and beyond his desire. 

After a year or two he seemed able to be reconciled to the tragic losses among his family and friends and to accept the changes in the new world as not wholly bad. The lavish hospitality he offered to scores of friends – and some enemies – at Pixton Park caused financial problems, but they were overcome by the ingenuity of his mother and Mary, and his popularity in Dulverton and indeed the whole county and even among party members offended by his independent thought and action - was secure. Besides, his social eminence, his generosity, his perfect manners, his transparent honesty, were such that “malice domestic” hid its head. and only one identified enemy could bring him down. But that enemy, having bided his time for twenty years, struck early in 1923.

Herbert was making a speech on one of his favourite topics when suddenly a veil of darkness came down over his eyes and he had to be led from the platform. He became totally blind. Someone very unwisely told him that having all his teeth extracted would help restore his sight, but the dental operation resulted in blood poisoning and he died in London on 26 September 1923.

On all sides his sudden severance from a busy world, this ruthless cutting-short of an active and, in the truest sense, romantic life, brought a feeling of irreparable loss : to his immediate family, naturally, and to the men and women whose daily work and living depended on him; to the scores of friends and acquaintances who, viting Pixton, enjoyed a wealth of hospitality which he provided without stint – even to his own disadvantage; to the “faithful few” in Parliament who shared his deep concern for the good name of Britain in the emerging world and, as they saw it, for Britain’s failure to honour promises made to the men in the Near East who had helped her in her hour of need.

And in all honesty, what was there left for him to do? Blindness barred him from the traditional sports of the landowner – not that he had cared for them over much before, and he had now played his part in a greater game altogether ; nor could he have retired into anonymity as Lawrence had done (and nor would he have considered such retirement as anything but a defeat). He had already written of his travels and military service in the Balkans and the Levant? Could he give the rest of his time to distilling his knowledge of those lands and presenting it to the public, and particularly to statesmen who would value and act upon his expertise? But who in government, apart from a few clear-sighted realists, cared about them? - In the disjointed world of the early 1920s every prospect for him, apart from the life of the family and the estate, must have seemed a declension, a sad falling-off, from the life of adventure, initiative and responsibility which had filled the past twenty years and given him such satisfaction and pride. 

And yet, even after his death, something of his gallant personality lived on, to be renewed in the next generation. His son Auberon seemed in his early manhood to be of quite a different mettle from his father – not inferior in any way, but different. Whereas Aubrey Herbert had been the soul of sociability, expansive, venturesome, supremely confident, able to converse on equal terms with all classes and in half a dozen languages, and always ready to make a stir in the world. His son was just as affable with those he met, but he did not seek out acquaintances and, lived quietly at home for the most part and gave his mind and energy to the running of the estate and serving his county as a district councillor. 

When war came in 1939 he volunteered for the army, but like his father, was turned down on medical grounds - but for the prosaic reason of flat feet. Disappointed but not deterred, he cast around for another service and enlisted in the Free Polish Army, a choice which for its initiative and and acceptance of a linguistic challenge would surely have delighted his father. 

\section*{Epilogue}
 
Aubrey and Mary Herbert had four children. Laura married the novelist Evelyn Waugh, who featured the Italian villa in his war trilogy Sword of Honour. Herbert was a slim man of more than average height and contemporaries described him as having perfect manners. 

It is widely believed that Herbert is the inspiration for the character Sandy Arbuthnot, a hero in several John Buchan novels. The series starts with The Thirty-Nine Steps, but Arbuthnot's first appearance is in Greenmantle, hence the title of his grand-daughter's biography of him, The Man Who Was Greenmantle. Herbert's Italian family villa is the model for that in the Sword of Honour trilogy by his son-in-law Evelyn Waugh.
The cameo character of the 'Honourable Herbert' in Louis de Bernieres's novel Birds Without Wings is clearly based on Herbert. He appears as a British liaison officer with the ANZAC troops serving in the Galipoli campaign. A polyglot officer able to communicate with both sides, he arranged the burial of the dead of both sides, achieving great popularity and trust with both sides.