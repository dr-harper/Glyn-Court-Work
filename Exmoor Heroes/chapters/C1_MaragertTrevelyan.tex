\Chapter{Margaret Trevelyan}{Gentlewoman}

For four hundred years Nettlecombe Court, near the eastern edge of Exmoor, was the home of the Trevelyan family, represented in the 1640s by George and Margaret and their children. George, born in 1613, had married Margaret, daughter of Sir Robert Strode, of Parnham, Dorset in 1637, and although she had come from a distance, she already had a link with the neighbourhood in that her mother was a Wyndham from Orchard Wyndham, only a mile from Nettlecombe, and Thomas Luttrell, of Dunster Castle, was her uncle.

Life for a landowning family with young children in the late 1630s should have been very pleasant, as it was for the Trevelyans, with a growing family and a flourishing estate in a sheltered and well-watered valley; but then the Civil War broke upon them. George Trevelyan had already been charged with the task of raising a regiment of foot for the king and now he was commissioned as captain of a troop of horse in Sir Charles Berkeley's regiment. 

It would take a bonfire of historical novelettes and costume films to erase from the popular mind the stereotypes of the gallant, impetuous Royalist and the sour, envious Roundhead, for in the maze of misconceptions the genius of Cromwell and Milton, the intrepidity of Blake, the culture and moderation of Fairfax and Hutchinson, count for nothing. But a clearer eye can discern, above the smoke and turmoil of those years, a common restraint from the worst excesses of war and a chivalry in which the palm often went not to those trained in military exercises from boyhood upward but to those who were "but warriors for the working day." At all events, Margaret, a devoted Royalist and even more devoted wife, had the tenacity which distinguished her Roundhead kindred. 

Like so many Royalists, the Trevelyans gave unsparingly to the king's cause, looking for no early return   realistically when dealing with the Stuarts. But the little ready cash of a country estate did not last long. More had to be raised, and the duty fell to Margaret. Ancestors of the Trevelyans had acquired lands in Glamorgan centuries before, and some still remained in the family, and in June 1643 she obtained a pass from her uncle, Thomas Luttrell of Dunster Castle, to allow her to travel to Wales to dispose of the remaining property "for the service of the King and Parliament". (Thomas Luttrell was strong for Parliament, but humanely refused to let his convictions embitter family feeling; and although "King and Parliament" was the Roundhead watchword against the Cavaliers' "for the King", both could consent to the first half). So Margaret took ship, most probably from Watchet or Minehead, sailed over to Wales, transacted the business and returned with enough money to defray the expenses of the troop raising and keep the Royalist cause going in those part for a year or two more. 

By then, however, the king's war was manifestly lost, and even his most zealous supporters were acknowledging defeat. George, now a colonel, took Thomas Luttrell's advice and reluctantly submitted to Parliament, but he was, in Parliamentary language, a "principal malignant" and they levied a fine on him   a standard one   of one twenty fifth of the value of his estate, or £1000. Thomas Luttrell did what he could and preserved the estate from plundering until his kinsman had submitted and the Parliamentarian representatives had guaranteed its safety. But in the later stages a detachment of Colonel Popham's horse, perhaps to settle a private grudge, plundered the house and also made off with "twelve plow oxen, two fat oxen, one hundred sheep and two horses". (They missed the family plate, as Margaret, anticipating such a raid, had had it concealed under the floor of the nursery, where it was found a century later). 

The Committee, to do them justice, supported Mr Trevelyan in his complaint and gave him all the help they could to recover the goods, but without success.

Meanwhile, a pardon still had to be obtained for the "delinquency", and the estate had to be brought out of sequestration. Lengthy operations both, and family counsels pointed to Margaret as the likelier of the two to carry them through. One can imagine with how heavy a heart she accepted, for the eldest of her children was only twelve years old and her youngest a matter of months, and they needed her. But duty was imperious, and the difficulties were great. Country families lived mainly on the produce of their home farms, money rents were low, capital was scarce, few facilities existed for raising loans and Nettlecombe was already exhausted by the earlier gifts to the king.

As a first step the corn stored in the barton had to be threshed and sold, and then, early in 1646, Margaret set out on her journey to London.

The estate horses had been carried off by the raiders, and so they had to yoke six of the remaining farm draught oxen to the family coach to take her to London. It would be a painful ten or twelve day journey in normal winter times, jolting endlessly along rutted tracks or floundering in seas of mud, a journey lightened only by the hospitality of great houses along the road, and now some of those - Basing House, for one - had suffered the ravages of war, and inns could be unwelcoming and innkeepers rapacious. But she survived the hazards, arrived safely and set about presenting her petition to the Committee for Compositions sitting at Goldsmiths' Hall.

Then she encountered the law's delays   and even a Cromwell, try as he might in the next decade, could not reform the legal system. Penned in London, while down in the West the hills were taking on the fresh green of spring, she tendered document after documents to the Commissioners, and on through a weary summer, until in August they yielded to her perseverance and the House of Commons resolved "to accept the sum of fifteen hundred and sixty pounds of George Trevelyan as a fine for his delinquency", grant him a pardon and discharge the sequestration of his estate. Two months later the Bill went to the Lords and Margaret could now attend to the unfinished business: to try to obtain redress for the failure of the authorities to recover the plundered goods. The law moved ponderously but early in December she was allowed to enter an affidavit.

Sir Walter Trevelyan, 200 years later, wrote, "It must be admitted that great forbearance was shown by those who managed the matter on behalf of the Parliament, among whom were the well known regicides Ludlow and Martin. But a more general observation is called for. With the exception of the plunder of the house, which may be suspected to have had its origin in a private grudge, these transactions, abnormal and irregular as they were, were tempered by a spirit of equity, order and exact attention to business, characteristic of a law loving people."

Margaret's work in London for her family and home was complete and she set off for home with eager anticipation of a joyful reunion for Christmas among those from whom she had been separated so long.

But a different, sadder Christmas was reserved for the Trevelyans. Margaret travelled one stage, but at Hounslow she was overtaken by a messenger whom every man and woman feared. For seven days she fought against the small pox, but in vain, and a memorial tablet erected by her sorrowing husband in the parish church of Hounslow bears witness to his grief and her sacrificial love. Some years ago, returning home from London, I turned aside to Hounslow and visited the church. It had been demolished in the 1840s but they had set up the tablet again in the new building. A hundred years later this also was destroyed in the blitz, but again they rescued the tablet and set it up in the new church, and it reads:

‘Here lyeth Mrs Margaret Trevelyan, the wife of George Trevelyan of Nettlecombe in the County of Somerset, deceased December 24, 1647, leaving issue eight sons and three daughters, viz.George, John, Robert, Henry, Alexander, Francis, Amyas, Anthony, Margaret, Susan and Katherin. For her Vertuous Life and Godly Death hir Mortallity shallbe made Immortally Glorious."

And surely, in a story of great misfortunes bravely borne, of duty faithfully fulfilled to the end of life and strength, there is matter not of sorrow only but of inspiration, and in the words from ‘Samson Agonistes’ by one of the greatest poet of her age:\\

\begin{quote}
 	"Nothing is here for tears, nothing to wail,\\
 	Dispraise, or blame, nothing but well and fair,\\
 	And what may quiet us in a death so noble."\\
\end{quote}

