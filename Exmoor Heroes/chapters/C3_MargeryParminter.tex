\chapter{Margery Parminter}

Behind most of the legends of the Duking Days one can find solid fact. They date back to the year of the battle itself, and though they may have acquired ornaments over the centuries, the basic facts admit of no doubt.

One story, however, that of Margery Parminter of Withypool, in Exmoor, seems never to have been written down till much later, but that disproves nothing, for in that district the telling of old tales round the fire of an evening - I speak from personal experience - kept them alive from generation to generation.

Not many Exmoor men were reported as having been away with the Duke, but fourteen rebels were hanged in Dulverton, Dunster and Minehead, and so it seems likely that a few Exmoor hearts had been in the right place - if an unwise one - and that John Parminter and a handful of men from Withypool, Winsford and Dulverton shouldered their scythes and tramped off to serve the man whom they saw as the last if not the best of hope the Good Old Cause.

The story goes that Parminter escaped from the battlefield and tried to make his way home, but in Winsford he was captured by a troop of horse and condemned to death, the sentence to be carried out in Dulverton in order to strike terror into as many hearts as possible.
His wife Margery had just been delivered of a child. (And what, one may ask, was her husband doing, gallivanting around the country when so painfully needed at home? - but that is a modern question which would probably not have occurred to either of them in 1685).

At any rate, the news reached her while John was still in Winsfordl, and with her new-born child in her arms she climbed up to Winsford hill and sat by the roadside to wait for the escort.

Soon the troop appeared, with john among them, bound, and Margery begged the captain for her husband’s life. “Ask me again in Dulverton,” answered the captain with refined cruelty, “and I will spare his life” - for he could see only too clearly that Margery, carrying a child, could not keep up with his troop and reach Dulverton before her husband was hanged; and so they rode on.

But the officer had not reckoned with a woman’s determination. Margery followed them, half-running, half-stumbling, putting forth all her weary strength to keep up, but soon being left behind.

Then the youngest trooper, riding at the rear, took pity on her and let her hold on to his stirrup, but even with this help she felt herself weakening. At last the trooper drew rein on some pretext and told her to mount behind him.

They trotted on in this way as far as Court Down, two miles from Dulverton, but here the horse went lame and Margery had to dismount and stumble with her babe the rest of the way on foot, watching the troop fade in the distance.

When at last she arrived in Dulverton she saw a public execution already in progress, with three rebels swinging from the gibbet and her husband about to hang beside them. But Margery summoned up her last reserves of strength, pushed through the crowd and cried, “Captain, I am here - in Dulverton - to claim my husband’s life.”

 “Egad,” said the officer admiring despite himself, “if these be your West Country wives, I’d liefer fight the men. Turn him loose, boys.”

 And they did.

\Chapter{Tom Faggus}{“Gentleman of the Road”}

All the world loves the idea of the honest rogue, unless they find themselves at the pointed end of his activities, when they revise their opinion rather smartly. In the ordinary way there is a world of difference between an honest rogue and a scoundrel. The latter makes his living by cheating, often within the law; but when the honest rogue breaks the law, he does it with panache. He observes certain restraints, and he only takes from those who, in his view, can afford to give.

It is a sad commentary on our society that while scoundrels still flourish, honest rogues seem to have gone out of business, but the West of England has known them in plenty, and never was a better one than Tom Faggus.

By all that is known of him, Tom deserved national fame much more than mohocks of the highway as Dick Turpin and Jack Sheppard, but by a quirk of history he has dwindled to the dimensions of a character in a romance. There are, of course, those nigglers who challenge his very existence on the ground of lack of written evidence, but life teaches that every legend rests on a basis of fact. A stone in an Exford cottage inscribed “Thomas Fugars 1674” was linked with him by tradition, but a more promising lead comes from Johanna Faggus, who in the 1840s was found as a weeping, deserted little girl in the hillside village of Atherington, between Torrington and South Molton. She was taken home by a Mr Graddon of Eastacott, who learned that Johanna was descended from a highwayman whose family had had property in Bath. Since then the name Faggus seems to have vanished into limbo, but in the convincingly Devon form of Vaggers it still survives and flourishes around Bideford and South Molton.

With the half certainties one fact is sure, that Tom Faggus was known to the people of Exmoor long before Blackmore brought him into Lorna Doone. Indeed, the written sources for his life were the same as for the Doones, in a manuscript book prepared in 1839 and containing three stories, De Wichehalse, The Doones and 'Tom Faggus and his Strawberry Horse. The stories were told to the Rev Matthew Mundy, vicar of Lynton, two of whose friends helped him collect notes from Ursula Fry (born in 1776) and Aggie Norman (born in 1777). Mundy wrote the stories up, and the senior girls of the National School made a number of copies under the direction of the headmistress, but the manuscripts never found their way into print.

Even so, the sources can be traced further back still to Ursula Babb, who was born in 1738, within the lifetime of those who had been children when Tom was riding the tracks of Exmoor. The Babbs were servants with the Wichehalse family, and Ursula's grandfather John played a small if regrettable part in history when, it was said, he shot and captured the rebel Major Wade fleeing from Sedgemoor.

\Flourish
 
Tom came from the village of North Molton, on the southern fringe of Exmoor. In those days it was, as Jan Ridd straight-facedly remarks, "a rough rude place at the end of Exmoor, so that many people marvelled if such a man was bred there." 

In his youth Tom practised the trade of a blacksmith in the village, and his skill made him a man of property with land worth £100 and gained him a reputation only too envied. But, in Blackmore's words, “being left an orphan (with all those cares upon him) he began to work right early, and made such a fame at the shoeing of horses, that the farriers of Barum (Barnstaple) were like to lose their custom. And indeed, he won a golden Jacobus for the best shod nag in the north of Devon, and some say that he never was forgiven. But whether it were that or not, he fell into bitter trouble within a month of his victory, when his trade was growing upon him, and his sweetheart ready to marry him. For he loved a maid of South Molton, Betsy Paramore, and her father had given consent; and Tom Faggus, wishing to look his best, and be clean of course, had a tailor at work upstairs for him, who had come all the way from Exeter. And Betsy's things were ready too, when suddenly, like a thunderbolt, a lawyer's writ fell upon him."

In short, Tom fell victim to a lawsuit instituted by Sir Robert Bampfylde, who "could pay for much swearing". He lost his goods, his farm, his smithy, he lost Betsy, who was married off the month following. Of all he owned he kept only a suit of clothes, a brace of pistols and his favourite mare, but they were enough. Worsted for the moment by the machinery of the law, Tom turned robber   or in more courteous terms, became a gentleman of the road, with truer claim to the title than most.

His skill and adroitness served him as well in his new career as in the old. Levying toll on all those whom he reckoned able to pay, he worked sometimes with a man called Penn, but generally alone, and very soon he had repaired his losses and more. To quote Jan Ridd again,"His name was soon as good as gold anywhere this side of Bristol. He studied his business by night and by day, with three horses all in hard work, until he had made a fine reputation: and then it was competent to him to rest, and he had plenty left for charity."

To tell the truth, though, Tom was having his revenge for the injustice he had suffered. He was selective in his choice of victims and kept up the ancient Robin Hood tradition of taking from the rich to give to the poor. "And all good people liked Mr Faggus   when he had not robbed them   and many a poor sick man or woman blessed him for other people's money; and all the hostlers, stable boys and tapsters entirely worshipped him. And so the landlords did; and he always paid them handsomely, so that all of them were kind to him and contended for his visits."

Besides all that, Tom never shed blood, and never insulted a woman.

One of his earliest exploits was very properly to waylay Sir Robert Bampfylde, who was accompanied by only one serving man. Tom never blustered. His eye and a firmly held pistol were enough. Sir Robert, after the briefest hesitation, handed over his purse and jewellery. Tom took them and immediately gave them back with a bow and the comment,"Sir Robert, 'twould be against all usage for me to rob a robber."

All the traditions of his exploits show his resourcefulness, his humour and ready wit, and moreover they all have the ring of truth, even when the locations assigned to them differ.

 He ranged far afield to relieve the rich of the burden of their wealth, and as his fame spread, so did that of his mare Winnie. She was a strawberry roan, "wonderfully beautiful, with a supple stride and soft slope of shoulder, and glossy coat, and prominent eyes full of fire," and gifted with an almost human intelligence. (And incidentally, in the 1870s an Exmoor farmer wishing to sell a mare to Squire Luttrell of Dunster described her as "Faggis coloured, your honour").

More than once Winnie's sagacity saved Tom from capture and death. He was discovered in Barnstaple and attempted to get away, but on the Long Bridge over the Taw he found himself trapped between his pursuers and a new party. But at a whispered word in her ear Winnie cleared the parapet, leapt into the river, swam to the bank and so brought Tom clean away. (The same story was also located, more plausibly, at Exebridge, where the mare leapt into the meadow several feet below and galloped away across the fields).

On another occasion Tom was trapped while dining at an ale house in Withypool, but his shrill whistle reached the ears of Winnie in the stable. She kicked down the stable door, bit and kicked her way through the crowd surrounding the house and bore her master off in triumph.

A similar tale is told of Porlock, where Tom was again surrounded in a house by a hastily armed mob shouting "Faggus is taken! Faggus is taken!" But a Faggus taken was not a Faggus held, and audacity carried the day. The door was flung open, and Tom dashed out on the strawberry mare and away through the crowd before a shot could be fired.

Over on the west of the moor he waylaid a party of farmers who had agreed to ride home together from Barnstaple market and thus avoid an encounter with him. They were ambling along near Bratton Down when Tom appeared out of nowhere, snapped "Stand and deliver!", made them drop their purses by the roadside and ordered them to ride on while he made off with his takings.

A few miles from there Tom lived for a time at Yeoland, near Swimbridge. One Sunday morning when most of the folk were at church he returned home; but one old man caught sight of him, hobbled up to the door of the church and piped, "Tom Vaggus be come hoame," whereupon the congregation left parson to preach to himself and rushed out to arm themselves and surround the house. Tom saw them coming, too close for him to make a dash for it, but they were no match for his quick wits. He put his hat on his fire fork, the long support for his musket, and pushed it at arm's length up out of the chimney. A shout went up: "There he be, tryin' to get out o' the chimley!" They loosed off a volley at the hat, which dropped, and they ran for the front door. Tom slipped out the back, mounted his mare and, before any of them could reload, was well away.

He showed as much ingenuity when hard pressed in Alcombe, near Minehead. At the top of Marsh Lane, now the site of the village hall, was a smithy kept for several generations by the family of Chappell. One day Tom rode up in a lather. He had outdistanced his pursuers, but time was pressing, and by dint of gold or pistols he prevailed on the smith to re shoe Winnie "back end vore" and thus send the posse on a false trail. (Other outlaws, notably Hereward the Wake, are supposed to have adopted the same stratagem, but that makes the tale more likely than less).

At Exford   though some say Doniford, near Watchet   his adversaries displayed a degree of imbecility delightful to behold. The hue and cry had been raised, and as Tom was expected to pass through, an armed posse with magistrates assembled to secure him. After a while they caught sight of a stranger of grave demeanour riding slowly down the hill toward them. 

He stopped, raised his hat most courteously and asked them what they were waiting there for. 

 "Tom Faggus," they said, "he'll be here soon." \\
 "What sort of arms have you got to shoot with?" asked the stranger. They showed him. 

"I don't suppose they've been fired for a long time, have they?" he said critically. Then looking at two or three he said again, " No, I didn’t think they had. You won't be able to fire those things in a month of Sundays," whereupon, to prove him wrong, they all fired off their muskets. 

Tom whipped out his pistols, held them to two of the magistrates' heads, collected all the valuables and made off, leaving the faint pursuers far behind.

Another time, in his guise of gentleman, he persuaded the posse, which included six J Ps, that he held a commission "to make cease a notorious rogue, Thomas Faggus" and that, as before, they ought to fire off their muskets to make sure the priming was not damp when needed (a nice Cromwellian touch there). They accordingly obliged, and Tom and his pistols relieved them of their ready money and scattered it to the poor folk standing around. Then he bade them all "Good morning" and Winnie carried him out of range before any musket could be reloaded.

\Flourish 

As to Tom's end, there are two stories, and since neither can be proved, we may take whichever appeals as poetic or other justice.

By tradition he fell victim to his kindheartedness and sense of hospitality and was captured by a low trick in an ale house at Exebridge. An officer of the law entered disguised as a beggar and squatted down in a corner of the room, but Tom, out of the kindness of his heart, invited him to "draa vore", gave him alms and ordered him some food. Now that the highwayman was off his guard the officer repaid his kindness by tipping over his chair from behind, slipping a noose round his feet and hoisting him up to the ceiling; and this time Tom whistled his mare in vain. The lawmen outside had shot Winnie in her stable. So Tom was tried at the next Taunton Assizes and hanged.

It has never been proved, though, and Blackmore is just as likely to have been right when he made Tom, who had been ruined by the lawyers, use them in his turn to "purchase from old Sir Roger Bassett a nice bit of land to the south of the moor, and in the parish of Molland. When the lawyers knew thoroughly who he was, and how he had made his money, they behaved uncommonly well to him, and showed great sympathy with his pursuits. He put them up to a thing or two; and they poked him in the ribs, and laughed, and said that he was quite a boy, but of the right sort, none the less. And so they made old Squire Bassett pay the bill for both sides; and all he got for 300 acres was £1,050, though Tom had paid £1,500." So Tom, now Squire Faggus, with King William's pardon in his pocket, and Winnie's assistance, built up a stud of Exmoor ponies for the London market, acquired a second fortune, became a good husband to the lovely Annie Ridd, "lived a godly (though not always sober) life, and brought up his children to honesty, as the first of all qualifications."

And if this ending is not true, it ought to be, for "since Tom Faggus died, there has not been such a man to be found, anywhere round these parts."