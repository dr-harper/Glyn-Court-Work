\chapter{Epilogue}

Over many years of writing to recall the vanished men and women of my neighbourhood, I have been, as it were, torn between two duties: the first: to praise famous men and our fathers that begot us – and the duty is easy – and the search for those “whose bodies are buried in sleep and have no memorial” , no written or sculpted word, nor even a weather-worn stone raised over a mound in a churchyard. And it may be that  infrequent (and often inaccurate) entries in a parish register simply are their only written memorial -and these few words ought to be re-recorded and treasured from generation to generation. These ancestors did not move the earth – but they would not themselves be moved, and they hung on, and in long course of time, from an ironmonger’s in Bristol, brought forth one of  England’s great captains and commanders, and for the men who served with him, the most admired, trusted and humane: 

“Uncle Bill”  from Bristol                                                                                                                                                                           
                                                 Lt-General (later Field Marshal) Sir William Slim, G.O.C.XIV Army, Fourteenth Army, Assam and Burma, 1943 – 1945                                        
  
When General Slim came to speak to us, the Queen’s Royal Regiment 1st Battalion in Burma, before our return to action after a period of rest in the  hill station of Shillong, I straightway, at the very first sight , felt greatness embodied in him and making itself unquestionably known and conveyed to us and drawing us in as never anyone before -  or, for that matter, since. But the greatness and power were not simply the effect of his rocklike stance and firm  jaw and commanding eyes: they came from the unshakeable certainty of victory that he held and conveyed to us. But it was the way he spoke with us, not to us, that marked him off from all the rest.

I repeat, the very sight of him inspired trust and beyond that, confidence, and I remember saying at the time, “He’s not going to throw us away.”

Later, we learned his history. He had not come from a military family (his father was an ironmonger in Bristol), and unlike the commanders drawn from the aristocratic and landowning classes,   and perhaps all the more readily for that,  understood the nature and needs of the private or rifleman who, after whatever the artillery might throw in to help, had to go in at the sharp end, achieve the impossible and then hold fast.   Unlike some of the others, our general  could not be accused of fighting  to maintain  the supremacy  of his class or family when the fighting was over. And if , miming an invalided veteran, we sometimes sang , ‘I’m one of Lord Louis’ broken toys’, no one, not even the most resentful  conscript, would have dreamt of linking such derision  to General William Slim.   

He did not come from a family with military traditions, and perhaps  understood all the better the nature and the needs of the private soldier  or rifleman who, after the artillery had thrown in a few shells to help him, still had to go in at the “sharp end”  and achieve the impossible and then hold fast. 

And our General had lived the army life at our level, not in engineers (though that could be perilous enough) or Service  Corps or Ordnance, but in the p.b.i., one of the 800  rifles and bayonets of the Warwickshires, the 800 whom all the rest existed to supply and serve.   Thirty years earlier, in his first command, he had led from the front at Gallipoli and had gone on leading in “peacetime” defence of the North West Frontier of India against the continual incursions of Afghan masters of ambushes and sudden raids - with the added  terror of the Afghan women’s attention to our wounded. And in every rank he attained he led. (And  though no one “let on”, even now our General  was still risking   his life  - wrongly, we would have thought -  in aerial reconnaissance to see for himself.
  I say again,  the very first moment I saw him, I  knew  that he would not throw any of us away; and for the rest of us in the Fourteenth Army,  there was little if any  of that dreadful sense of fatality that weighed upon many soldiers  in the 1914 – 1918 War. Expected casualties in a battle or campaign – not necessarily fatal - were ten per cent, and I remember talking this over with my pal David Cook  and optimistically reckoning that we therefore had nine chances out of ten of coming through ,  whether whole or not.  Some of us – one in ten, but we hoped no worse than that, would not make it home, but we also knew without question, in our heart  and minds and aching feet, that we could trust General  Slim when he took us  where we had to go, and play straight with us and look out for us along the way.
Alternative  paragraphs to work on:
   He had a plan for victory, he had the resources, he had soldiers who had beaten the Japanese man to man time and time again,  but he (we) had to advance from static positions and the jungle into the open plain of Central Burma , trap the Japanese army by a brilliant manoeuvre of deception  and smash  it in a  hundred irrecoverable fragments – of desperate and  starved but still defiant men. Before the decisive advance  he called us together, from riflemen and cooks to brigadiers and major-generals and explained, simplicity itself, “This is what we are going to do” : and it all went as he had promised, it all came true. But afterwards “we” became “What you have done” – “you” not “we”, certainly not I, General Slim.”.   Was any other great captain so mindful of the unspoken need of gratitude and dignity of his fellow-soldiers?
  George MacDonald Fraser has said of him, “The thought of him was home and safety”  -  because he knew what must be done, and took us into his confidence; ew recognized this, that was the way of it.        
                                                                 CODA
 All General Slim’s  warfare, basically, was conducted not to massacre the Japanese by thousands – though that was the way that so many of them chose – but to break their will to fight on; and to accomplish this not by  frontal attacks on bunkers – often more costly to the attackers than the defenders – but by learning and practising  the encircling movement  where intelligence, in both senses, and anticipation paid off.  Inevitably, great battles erupted such as at Mandalay and at Pyawbwe, where the Japanese were surrounded  and,  refusing to surrender, could only die in frenzied devotion  to their Emperor -  a devotion which elsewhere compelled repeated obedience to catastrophic orders issued in ignorance of the local situation and blinded them to military sense. 
  The British soldier recognised the military efficiency and discipline that had won the Japanese an empire in a matter of months and had dragged hundreds of thousand of unprepared British soldiers into slavery. But on the jungle-clad hillsides of Assam the British drew a line and said “No further”. And for a long month in May and June 1944  the 1500 men of the Queen’s Royal Regiment and the Royal West Kents staved off 15,000  Japanese. At the height of the battle of Kohima   the front lines off the two armies, scene of hand-to-hand and close quarter grenade fighting, ran along the “tramways” of the District Commissioner’s tennis court , and the 250 British wounded, who could not be evacuated, lay out in the open with no  more protection than the canvas walls of a dressing station. After 40 days the British Second Division, flown from India, broke the siege  and the survivors marched out with fixed bayonets – but only 27 of them.. The Fourteenth Army had taken the measure of the once-victorious Japanese, and never again was superiority in doubt. Many years afterwards a Japanese general said ruefully, “We threw everything we had at them in Kohima – but we could not break the British soldier”.

 
 







	







                                        









   


 




















