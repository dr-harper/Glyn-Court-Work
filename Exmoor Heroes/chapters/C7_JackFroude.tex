\Chapter{Jack Froude}{Hunting Parson}

At the head of one of his stories of life with the British Army, André Maurois quotes the historian Justin McCarthy: "The ideal of the Established Church has been to secure one resident gentleman in every parish   and there have been worse ideals." 
 Besides, the ideal was a reasonable and attainable, and the parsons who fulfilled it, living peaceably in their habitations and exerting an influence for good, were happily too many to be named. There were others, however, who illustrated the ideal only by their unremitting failure to live up to it, and who, human nature being as it is, have strutted in the limelight and entertained us ever since. So having paid a brief tribute to parsonical worth, may I now invite your indulgence for the far from Reverend Jack Froude.

\Flourish 

Knowstone, in North Devon, is a pleasant little spot nowadays, within easy reach of the Bampton Barnstaple road yet quiet and unhurried; but in Froude's time it was regarded as remote with inhabitants half civilized. Parson Froude ministered, or ruled, or terrorised, there from 1803 to 1852, only one year short of a half century, and the parish suited him well, for the remoteness from bishops and their visitations gave him freedom to act pretty well as he wished.

His character is puzzling. He came from a distinguished family, with one brother a famous historian and another a divine, but he himself, not to gloss over the matter, was a ruffian and a bully, though a courageous rider and physically as hard as chilled steel. Perhaps he represented a throw back to a coarser, more brutal age. But then his character had another side, for he was shrewd and keen witted and possessed considerable intellectual ability   for which last he found little scope, and maybe this also explains something of his character.

While still a young man he gained the reputation of a leading sportsman, even if sportsmanship did not figure prominently among his virtues. He was an excellent judge of horse and hound, and as he had private means he kept his own pack and hunted whenever, wherever and whatever he could find. Few men dared to challenge him on this ground or any other. Some had tried, but quickly learned that it did not pay. Inexplicable accidents happened.

Froude was his own law. He ruled the roost and had a gang of almost feudal retainers who would carry out his wishes unquestioningly. He had no need to issue direct commands, he had only to say thoughtfully, "Well now, wouldn't it be a terrible misfortune for Farmer Huxtable if his hayrick should catch fire and burn down. I shouldn't like for that to happen to the poor old chap," and next morning the rick would be a smoking ruin; or he might say, "What a shame 't would be if Thomas Galley was to lose his old sow," and soon after Galley would find his stock depleted.

As mentioned above, Froude was an excellent judge of horseflesh and a horse coper no more burdened with scruples than the rest of the trade. But to be fair, it seems that the game of hoodwinking a customer meant more to him than the money he gained. The speech of Devon was his natural mode, and he enlivened his rascality with so much humour that his victims could not long nurse a grudge. They ruefully confessed that anyone who dealt with Froude had better keep his wits sharp, eyes open, purse buttoned and fingers tightly crossed.

According to H.J.Marshall, a former vicar of Porlock, Froude once sent his boy of all work, George Slocombe, to Bampton Fair with an old, broken down horse. "Jarge," he said to him, "you be gwain' take th' ole 'oss to Bamp'n Fair an' zell en. He bain't zound, ner nort else to spaik of, zo don't 'ee say he be, mind. If they do ax 'ee about en, jis' say, " If thik 'oss idn' zound I'll varveit a sovereign.: Then 'tis all fair an' 'bove board. Ax twenty pun fer en, not a penny less."

George was not over bright in most of daily life, but to compensate he was doubly shrewd in money matters. He set off early next morning. The horse was old and stiff, and George rode him at a walk all the way to Dunster to take the stiffness out of his joints. Then, on the outskirts, he woke Dobbin up and trotted him into the middle of the fair.

By this time the horse had taken anew lease of life. He had been a good fellow in his prime, and now that the walk had limbered him up and stirred his blood, the bustle and excitement of the crowd made him fell young again and he trotted to and fro looking and feeling like a four year old.

Even the shrewdest and most cautious dealers thought there was promise in him, and at length one of them said, "What's the price of him, boy?"
 "Twenty pun, sir," said George. "Maister says he wadn' to go fer less."
 "Is he sound, though?"
 "If thik ther 'oss bain't zound," George returned solemnly, "I'll varveit a sovereign."
 A satisfied buyer handed over £20, and George walked back to Knowstone, carrrying saddle and bridle, and recounted the events of the day to "passun".

There the matter rested until a few days later, when the buyer appeared on Froude's doorstep and demanded angrily to see him. He was conducted to the room where Froude was sitting, and demanded, "Have you a boy called George Slocombe living in this parish?"
 "Yes," said Froude. "What d'ye want of him?"
 "The young rogue sold me a lame horse down Bampton Fair and claimed he was sound. I want my money back."
 "Oho!" said Froude, "this needs looking into. I"ll send for the boy and hear what he's got to say for himself. Will 'ee have a drop o' gin?"

George was sent for, the uninvited guest was lubricated, and a mellow atmosphere took the place of anger.
 Enter George, stolid and unperturbed.
 "George," said his master, " this gen'leman says you sold en a 'oss in Bamp'n Fair t' other day. Is it true?"
 "Iss, zir," said George. 
 "Did you guarantee he was sound?"
 "No, zir, I did not."
 "Tell me, then, what exactly did 'ee say to the gen'leman?
 "I said, "If thik ther 'oss bain't zound I'll varveit a suvvereign."
 "So that's how 'twas, then," said Froude. "Well then, George, hand the gen'leman awver his sovereign."

\Flourish

On another occasion Froude advertised a pony for sale, and a prospective buyer came a considerable distance to inspect the horse. Froude insisted that they should dine first. "We'll tot up," said he, "with a drop of something hot first, and then my huntsman shall show 'ee the horse." (It hardly needs saying that his huntsman, Jack Babbage, was a willing accomplice, but more of him later).

During dinner Froude regaled his guest with mug after mug of his own home brewed, and after lengthy treatment he was at last fit, as Froude saw it, to judge the horse, and they went out to a field where stood some hurdles "feathered" with furze bushes. Burge urged the horse over the hurdles in fine style, shouting whenever he neared the jump, and the visitor, content with what he had seen and in no state to try for himself, made out a cheque and drove away, pleased with his purchase.

A week later came a letter. It went straight into the fire unopened. A second and third letter came, with the same result. Finally the buyer himself appeared, furious. His horse was blind.
 "Not my business, sir," said Froude unabashed. "You wanted a hunter that could jump, and that's what I sold you. There's no denying it. You didn't ask if he could see. When you ride him, take a knife, and when you come to a fence, jump off and cut a furze bush. Lay it down before the fence and canter up to it, and shout as you heard Babbage do, and so soon as he feels the prickles about his legs he'll jump right enough!"
 "Take him back, sir," shouted the buyer.\\
 "Certainly, sir."\\
 "And give me my money back."\\
 "Certainly not, a deal's a deal. I've done my part, cashed my cheque and paid the butcher, and there's an end on 't." \\

\Flourish


Unlike Ralph Allen of Bath, who did good by stealth, and blushed to find it fame, any good done by Jack Froude was generally accidental, and blushing was a practical impossibility for one of his port wine complexion. As for stratagems and the thrill of the chase, neighbours, servants, fellow parsons, foxes and bishops: all were fair game.

The Bishop of Exeter in Froude's later years was Henry Phillpotts, a formidable disciplinarian whose sarcasm awoke trepidation in ninety nine out of a hundred of his clergy. The hundredth was of course Froude, but even he viewed a visitation by Philpotts with disquiet, and when the bishop came to South Molton and summoned him to attend, he sent his excuses. The bishop was not to be fobbed off and decided to call on him instead.

Froude had prior knowledge of this and posted look outs along the lane to warn him of the bishop's approach, but he had not counted on Philpott's promptness or energy. He was talking to the huntsman, and the hounds were prowling round the rectory lawn, when the look out boy ran in with the news that the bishop's coach was coming up the lane. Babbage hastily removed the hounds and Froude even more hastily removed himself and dived into bed, hunting coat, boots, breeches and all.

Moments later the doorbell rang, and Jane, the old housekeeper, went to answer and informed the bishop that her master was ill in bed. "In that case," said the bishop, "I shall visit him in bed."

Jane showed him in, sat him down and went to Froude for instructions.
 "Tell his lordship," said Froude, "I don't rightly know what I"ve got, but 'tis zummat catchy for sartin:  scarlet fever, I reckon."\\
 Jane relayed the news, but Philpott brushed the warning aside, went up to Froude's room and sat down by his bedside.\\
 "What'll your lordship take?" Froude greeted him, huddling down under the bedclothes so that only his head showed. "Cruel cold 'tis to day. A drop o' hot brandy an' water to keep off th' infection?"
 "Nothing, thank you."\\
 He paused. Then   "I am sorry to say that strange stories about you reach my ears, Mr Froude."\\
 "Whisky with a slice of lemon, then?" said Froude, unconcerned.\\
 "I beg you, no," answered the bishop with some heat. " I want the truth of these stories about you."\\
 "Why! I've heard some strange stories about your lordship!" retorted Froude. "But there! Us gentlemen don't give heed to all thik tittle tattle.   You'll excuse me, my lord, I be turr'ble bad. Pleased to have seen 'ee. Good bye"   and with that he tucked his head under the blankets and vanished from sight, and the bishop retired discomfited.

\Flourish 

He did not give up easily, though, and some time later he wrote announcing another visit and fixing a day and time. Froude could obviously not play the same trick again, but nor would it pay to offend the bishop a second time. He consulted with the churchwardens, and together they concocted a scheme.

The lane from the Barnstaple road crosses the stream called Crooked Oak, a small tributary of the River Mole, but in those days it formed a ford, shallow enough in dry weather but often impassable after heavy and prolonged rain, and a curious observer might have noticed that after the interview with the churchwardens some unusual activity had deepened the bed of the ford.

It was pure coincidence, of course, that as the bishop's coach came trundling along the lane the churchwardens were working with teams of horses in the fields alongside. The carriage reached the stream, drove on in, reached the middle and there stuck fast, with a fuming bishop marooned inside.

Along came a solicitude of churchwardens, intent on rescuing him with the minimum of disturbance to the rector and the maximum delay and fuss. After long discussion as to the best way to "draa his lardship out o' thik ther stream", they went in search of their teams. These they hitched on both sides of the carriage and the fun began, with the coach horses pulling in one direction, one team at right angles to it and the other team opposite to it   that is, south, west and east, with more than a touch of nor' west and nor' east   and the bishop shaken and tossed about like a cork in a whirlpool.

At long last they brought the coach safely to land   but on the side it had come from, and then they assured the bishop that there was no other road to Knowstone. Phillpotts may not have believed them, but this time he knew when he was beaten and retired to recoup his forces and fight again. 

He could never bring himself to countenance Froude's hunting, but the rector was incorrigible. The bishop, meeting Froude one day when he was exercising a greyhound   or "long dog" in Devon speech   inquired, "Pray, Mr Froude, what sort of dog may you call that?"

"That?" said Froude. "Oh, a long dog, me lord; an' if you was to shake your apern to en, off he 'd go like a dart."
 			* 
Macaulay, a contemporary of Froude, once wrote, "A taste for severe practical jokes may be pardoned in a boy, but when habitually and immediately indulged by a man of mature age and strong understanding, it is almost invariably the sign of a bad heart." Be that as it may, Froude certainly relished the discomfiture of others.

He did not object to his maidservants having "followers", but insisted that they come in openly and that he be told. Some of the young countrymen, however, found his banter hard to take, and they slipped into the kitchen by the back door. But Froude was not to be outwitted, and one evening, when Dick Gathercole was being entertained by his sweetheart and the other maids, they heard Froude's footsteps in the corridor.

As quick as thought they bundled Dick into the huge furnace in which the family brewed the ale. In strode the master and called, "Mary, light the furnace, or the puppies'll die. Look sharp. 'Tis bitter cold."

Mary could not disobey. She fetched faggots and stoked up the fire as Froude looked on with inward amusement. Dick remained hidden until he could stand the heat no longer and clambered out.

"Ah!" said Froude, "I thought you were there"   whereon Dick knocked him down.\\
 That at least was Dick's story, though according to Froude he tickled Gathercole up with his riding crop so that he sped from the kitchen at a speed of which any of his master's "long dogs" could well have been proud."\\
Jack Babbage, his huntsman, henchman, groom and retainer, had always obeyed his orders with a fidelity worthy of a better cause and let his conscience ride with a easy rein. But one day he fell ill and seemed like to die. Concerned at last over his soul's possible destination he sent for the rector. Froude was taken aback but fortified himself with brandy and water and strode off to visit the sick man.
 "Do 'ee think I'll get to Hev'n, sir, or t' other place?" Babbage enquired anxiously.\\
 "I couldn't rightly say, Jack," returned his master. "Let's see, though. Have 'ee broke any o' the Commandments?"\\
 "All o' 'em, zir, I reckons, 'part vrom murder." 
 "Oh." Froude pondered. "Wull, Jack, seems 'tis a bad look out fer 'ee. Tull 'ee what, 's know, us had best see if there's ort wicked thee hassn' never done. Look yere, now, have 'ee ever zhut a vox?"
 Babbage's face brightened. "Naw, zir, wicked I may ha' bin, but not zo wicked I 'd zhut a vox."
 "Wull then, Jack lad, I reckon there's hope fer 'ee 'itt."