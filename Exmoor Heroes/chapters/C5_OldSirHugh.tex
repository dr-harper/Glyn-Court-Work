\Chapter{Old Sir Hugh}{Night Rider}

In the hamlet of Five Bells, close by the ‘kirk upon the hill’ of the Ancient Mariner, there lived in the 1820s and thereafter a lonely man known to his neighbours as Old Sir Hugh (almost certainly of the Popham family, as witness the deeds of his cottage, but nobody mentioned this.). He need not have lived this solitary life, for other Pophams lived sociably enough in the neighbourhood. William, for example, as wheelwright in Roadwater. Nevertheless “Sir High”  had cut himself off from them – or so it was widely believed. Certainly he was a hard man to approach. The blame was not entirely his: a natural reserve had been taken amiss by people less sensitive than he. They shunned him, and he in his turn withdrew even further, so that to their superstitious minds, dark deeds and dark thoughts hovered around him, spells and charms lay hidden in the least of his few words, and a dark power was his unseen companion. He grew morose and resentful of his neighbours’ suspicions, but his isolation bred a sharpness of tongue which kept open disrespect at bay. (The ‘Sir’, may I add, was a popular honorific without heraldic sanction). 
 
His very appearance inspired awe. Gaunt of face, sunken-eyed, with black, lank hair hanging to his shoulders, he rode a skinny grey mare, with a huge black hound for companion, and took his secret ways from farm to farm in the light of the conspiring moon. If he were not the devil in person, he must be in his close counsels, so much was the common belief. But the gossips knew nothing of the deep places of his spirit, and even those he stealthily aided and served could only guess.

For Sir Hugh had a mission, and when he rode out of a night, it was not to a meeting with the powers of evil but on an errand of mercy. These were hard time, the years after Waterloo. After poor harvests, the scarce corn was locked in barns to keep up the price, and famine stalked the land. One resident born in the year of the battle remembered that in his family the Sunday dinner was often no more than a herring between six with a few potatoes. Certainly there was bread for the rich, but the poor starved and “rioted” round the baker’s van, and their old folk and children died. The iniquity of it pierced Sir Hugh’s soul. Since there was corn in the barns, why should not some of it go to the families who had laboured and sweated in the fields to plant it and harvest it into those same barns? - Yet to take from the rich to give to the poor was to risk hanging or at least transportation, and there would be informers and denouncers even among those whom he sought to help. 

His agile mind and a grim humour pointed the way. Did they suspect him of a pact with the devil? Very well, their pitiful suspicions should serve his purpose, and when he rode out his form should chill them with terror, make their hide their faces and bar their doors against his approach.

Painstakingly he made heavy skeleton keys which would turn the locks of all the barns round about and strung them on a leather thong. Then from the vault of the nearby church he took a skull, and in it, behind a flink of glass, he fixed a candle, and mounted it on his old beaver hat. Thus accoutred the strange knight-errant rode out, grim and erect, with a black dog at his heels, and when the ghostly crew passed along the winding lanes of Nettlecombe or over the windswept uplands of Cleeve Hill, with the keys clinking and jangling like the chains of hell, terror seized those who dwelt in lonely cottages. They locked and bolted their doors and put up prayers with more than usual fervour, and all who happened to be out and abroad and glimpsed the spectral form in the distance trembled and fled.

All, that is, except one man.

A few of the people, here and there, had shrewd doubts of the genuineness of the apparition but they kept them to themselves. The night-rider was no enemy of theirs. ‘Best leave ‘en be,’ they said. But this one man was more venturesome than the rest; less credulous, maybe, certainly less wise. He would waylay the ghostly horseman, he would reveal a creature of flesh and blood, he would challenge him by name. Accordingly, at the turn of a chill, still, starlit night in midwinter, he waited by a five-bar gate near Tumbland, a mile from Sir Hugh’s cottage, listening for the clip-clop of a horse making the long ascent of the hill leading from the Williton road to Fair Cross and thence to the barns of Huish and Wood Advent. Suddenly the sound of hooves was upon him, and the padding and panting of the great hound, the chinking of the keys and the eerie glow of the skull. For all his boldness a shudder ran through him, but he stood his ground and called out, “Good night, Sir Hugh”.

The figure raised its head. Starlight flooded the hollow cheeks, and the watcher saw in the deep-sunk eyes a wholly human anger, and a human voice, heavy with menace, returned, “Who the devil are you? Try that trick again and by heaven, you’ll regret it!’

Scarcely checking his pace, the rider moved on up the hill; but his challenger, all courage drained out of him, fled headlong, anywhere to be out of reach of the curses and terrible anger of Sir Hugh; and never, till the old man’s death, did the watcher speak openly of that midnight encounter. And so, through that cruel winter and for years beyond, Sir Hugh went on with his errands of mercy; and if the squires, the parsons and the farmers cursed him both loud and deep when they found their corn-bins low of a morning, blessings were rained on his head by the poor who, for a while, could eat bread and be satisfied.

As Sir Hugh had grown, so he remained, solitary, silent, withdrawn. Did any of those he had helped come to offer him their halting thanks? We do not know. We may guess that he would, as before, have turned them away with a brusque word. But he received his belated reward; for in the minds of the people of his ‘country’, till prosperity dulled the edge of gratitude, there lingered for many years the poignant memory of Old Sir Hugh.

\Flourish 

Postscript\\
And perhaps in a more visible if irrational way Sir Hugh has not ended his travels even now. His cottage at Five Bells still stands, externally little changed, but sometimes footsteps are heard in an upper room, though they cause no fear or distress. The deeds still show the name of the family from which he was cut off, and a gold coin from the time of Sir Hugh has been found in the garden, where workmen have unearthed the unsuspected foundations of an outbuilding which may have been his stable . But in the 1990s nothing of this, nor anything of the story of Old Sir Hugh, was known to the little girl who came running into the cottage from the garden to ask, “Grandma, who is that strange man with old-fashioned clothes and a funny hat?”
 