\Chapter{Jan and Nathan Palmer}{Hucksters and Originals}

‘Colourful’ is not the word one might have given them when seeing them in the flesh. Their clothes were worn and demoded, and even if they travelled over a wide area they led mentally restricted lives, They never created a stir, and never buckled a single swash between them. It was what they unwittingly or at least unquestioningly did that earned them a place in legend.

The parish registers of 19th century Luxborough, in the Brendon Hills of West Somerset, are an Araucanian forest of family trees, and the Palmers grew as close and impenetrably as any. Most of them followed the usual village trades, blacksmith, mason, dressmaker, servant, others emigrated; but two left their mark on the popular legend of this neighbourhood.

John Palmer, locally Jan, and his son Nathan were fellmongers, traveling a wide area of the hill country collecting skins, furs, rags and scraps of various kinds. More importantly, and justifying their claim to recognition by posterity, they had a vein of independence which, no matter how crudely expressed, showed that in an age when a labouring man hardly dared, as they said, “call his soul his own” they were willing to stand up, in their own way, to "the petty tyrant of the fields"; and for that, at very least, they deserve to be remembered. 

Jan Palmer was born in 1752 or 1753 at Uplowman in Devon, and some time in the 1780s had come up with his donkeys to Luxborough, in the Brendon Hills of West Somerset, and "squatted" outside the churchyard, where in an angle of the wall he built a house in the form of the letter D, frequently bringing home stones from long distances for its walls. His mansion was known as Palmer's Castle. The squire of the day, Sir John Lethbridge, did not see Jan as an architect quite in the class of Wren and tried more than once to persuade him and, later, his son to move to another cottage so that the Castle could be demol-ished; but Jan looked down his nose and stayed put. He had squatter's rights.

Laudably obstinate though he was in this matter, however, another incident recounted by William Thornton, vicar of Exmoor, shows Jan in a very poor light In his trade he travelled the district with a distressful string of donkeys, and he was a hard master to his beasts: 

“There was a certain archdeacon residing in this part of the world, a man of 70 years of age or thereabouts, an old public school boy, an old-fashioned, orthodox clergyman, fond of theology and not averse to port wine, a man vigorous for his years and a general favourite. One day, while riding in a lane in the Milverton direction, he encountered a string of donkeys laden with ruddle to sell to the hill country farmers for marking their sheep. The donkeys were thin and overloaded, and behind them walked a wretched, dirty, emaciated old man with a heavy knobbed stick in his hand.

“The last donkey had a large open sore on one of its quarters, and just as the archdeacon rode by, down came the knob of the stick right on the running sore.

“You cruel old rascal," cried the archdeacon, "if you beat that poor brute like that, I’ll beat you."
“It'll take a better man nor you to do that, I reckon," said the other, and whack came the stick again upon the sore. The archdeacon said no word, but rode to the first gate, dismounted, tethered his horse, walked back, and went for the donkeyman with a will.
“No one was there to chronicle the bout, but it got wind, and report said that at seventy, beef, port and science had prevailed over potatoes and poverty." The archdeacon's popularity increased; the donkeyman's did not : a matter which gave him no concern whatever.

One day the carpenter at Dunster said laughing to the doctor, "I've had an odd order to day, sir. A living man, an old fellow from Luxborough, has been to my place and ordered a coffin. I told him to lie down on the floor that I might take his measure, and then I chalked round him."

"A little more room for the shoulders, sir," he said, "if you please. I might grow a bit stouter before I die."
 Shortly afterwards the old man came with a donkey and took the coffin away, and when the doctor next called he discovered that some of the stones had been shifted from the wall, and the earth, which was higher inside than out, had been removed and the aperture lined with slates. There the coffin was thrust in, and was used as a drawer to contain bacon, old nails, twine, mouse traps and "other sweetmeats".

Not everyone, though, was disposed against Jan, and he would sometimes get a night's lodging for himself and donkey in the stable of the Green Dragon, Bilbrook, where lived William Symons, apothecary, tradesman, stage-coach entrepreneur, Methodist class leader and unofficial adviser to the neighbourhood. Jan, for his part, was not one to “presume”, and although invited to spend the night in the house he would curl up in the stable in the company of Nicholas Nye, content with a bundle of straw for a bed; but knowing the Methodist attachment of his host he would serenade the household with hymns and psalms before retiring. In later years he had an unsuspecedly religious side to his character, and in his will he directed that his house should always be available for religious services, and for a time the Bible Christians held meetings in his house, though he was not of their persuasion.

It was William Symons who prepared Jan's will. It contained at least one curious provision, that his most prized possession, a brass kettle - it was actually copper – should be shared among his four sons, the vatically named Samuel, Daniel, Joel and Nathan, and he directed that it should be used by each of them in turn! (Perhaps it was to forestall doubts as to his mental competence that he sang from memory the Hundredth Psalm.

Coincidence it may be, but more than a century later a strange sequel emerged. An old cottage, little more than a shed, near Palmer’s Castle had been used as a primitive club house but had fallen into neglect and disrepair, and in the mid-1980s a heap of rubbish was cleared out. Among this rubbish was an old, blackened and grimy kettle, but when a lady living nearby had worked on it and restored its pristine gleam the lineaments of the late eighteenth century kettle were clear to see.

Some of the actions credited to Jan may, I feel, have been the work of his son Nathan, but no matter; and if heredity has any meaning, no man deserves all the credit for what he does. Both Jan and Nathan had a mind to defend the use of rights of way and the innumerable tracks which even then were being barred at the whim or for the convenience of the landowner; and this is more or less the exchange that ensured when Jan, taking a short cut through the squire’s private grounds, had marched more than half way when the squire caught him:

“Palmer, you’re trespassing. Go back to the gate.”\\
 “Aw, I ha’n’t come more ‘n a vew yard, zir.”\\
 “Nonsense, man, you’re more than half way through now.”\\
 “Wull, zir, if I be more ‘n haaf-ways dru, best I sh’d go all the way, ‘cause if I do go back now, I’ll ha’ went vuther ‘n if I’d a-went vore. You let I go out t’ other aind, then I ‘ont ha’ went zo vur as if I’d a-went back, zir.”
 

On the other hand, and in the public interest, not merely his own, Jan is supposed to have carried out in his old age one imaginative feat which, simply on account of the date, I think should be credited to Nathan, but no matter. The circumstance was the building of a new road in 1829 from Luxborough to near the village of Roadwater, running along the valley floor and replacing an old hillside track which served two cottages but was fit only for horse and foot. The squire wanted the new road, but the village wanted to keep the track open for the sake of the cottages, and Jan took this to heart. He had heard that when the case came to a tribunal, the track could only be saved if wheeled traffic had passed along it within living memory, and unfortunately even the Methuselahs of Luxborough had no such recollection. But for Jan, to believe was to act. He took a pair of wheels off a light cart, strapped them on his back and walked the three-mile length of the track to the parish boundary; and when the new road was built, the old one was at least left unblocked. 	

Jan died away from home and his body was brought back on his donkey and buried in the house. When Sir John offered the old man's son a cottage in exchange for the Castle, it was accepted, and the son took up the body without troubling for permission and removed it to the churchyard, but kept the coffin lid, with its inscription, as a memento and an ornament to the walls of his new home.

Nathan Palmer was born in 1792 and survived until 1887, the last man in the neighborhood to wear the traditional countryman’s smock. Old age came early to the labouring poor, and Nathan bore the marks of hard living. Dr Francis G. Hayes of Dunster, four miles over the hill from Nathan’s home, left a speaking portrait of the old man and his steed: “He was a wiry, hard featured old man, with a grey beard and a withered up, parchment-like face, with very marked senile rings around the coloured part of his eyes that gave him a peculiar expression. He was a familiar object on the roads, and one that would instantly attract attention. He used to ride an old screw that he called his ‘hackney’, which was a mere bag of bones, encased in a tight-drawn skin.”

The pony was, in the local phrase, pumple-footed – it had virtually a club foot. ‘The hoof of one of the forefeet was enlarged and deformed, and could not be shod, and naturally the poor wretched old creature went almost entirely on three legs. The old man was as bony as his steed, and they made a good match. His riding costume was completed by hay bands wound from ankle to knee around his legs in place of gaiters. His weight was very little, and the pony used to hobble round the hill-country with him all day, Nathan collecting the skins of rabbits and hares, and sheep that had become casualties. I have seen him well loaded toward evening.’

An unprepossessing figure, then, but it would be best not to judge by appearances. Nathan was no fool, and his sense of independence found vent in sly humour. Like his father, he trespassed serenely, and one sunny afternoon as he progressed along the squire’s drive the peace and calm of the scene so worked on him that he tethered his ‘hackney’ to graze on the rich grass and sat down on a bank by the roadside. Before long he fell asleep and rolled down the bank, but woke just as the squire came up:

 “Trespassing, eh, Palmer? I shall have to prosecute you, man, you have done this too often.”\\
 “Oh, you ‘ouldn’ do thik, would ‘ee, zir? Jus’ think o’ the shame o’ t.”\\
 “You should have thought of that before you came trespassing on my land, Palmer.”\\
 “Oh, tidnn’ fer old Nathan to be ‘shaamed, zir, he ha’n’t done nort wrong. But if you was to summons me, whatever would ‘em think down Wullit’n (the magistrates’ court) if I was to tell ‘em squire do leave his drive get in such a state that a pore ole feller on his ‘oss valled down hile he were only tryin’ to pass en. I ‘ouldn’ want fer to putt ‘ee to shame, not fer the worl’, zir.”\\
 
Nathan used the same technique when the squire caught him taking home a rabbit:

 “Poaching this time, is it, Palmer? Well, it’s the maistrates’ court for you.”\\
 “Oh, come on, zir, you ‘ouldn’ grudge I a li’l rabbert, would ‘ee?”\\
 “It’s not one rabbit you’ve had, Palmer, it’s a dozen, as you well know.”\\
 “Well, zir, maybe I did vind thease li’l chap layin’ in top field an’ pick en up, like, but you ‘ouldn’ have me up in court, now, would ‘ee? Whatever would ‘em zay down Wullit’n if they knawed squire up yere were too minjy fer to let ole Nathan have a li’l rabbert to his Zunday dinner? Giddon with ee, zir, you ‘ld never live it down.”
 
No doubt about it, Nathan was a grainy character, and perhaps the grain here and there was a little "thrawn" in the growing. At all events, two of the stories about him show a degree of perverseness which can hardly be put down to independence.

Dr Abraham, in practice with Dr Hayes in Dunster, was also well versed in the manners and customs of the country, and he told a tale which put the old rascal in a sinister light.

One morning Jan called at the surgery in Dunster asking for some medicine for his wife, who was suffering severe internal pains. The doctor gave him a bottle containing a considerable number of doses of a sedative, probably laudanum, and instructed him to repeat the dose at the intervals stated on the bottle as long as the pain persisted, and report again if his wife were no better.

Late in the afternoon a message came from the "big house" in Luxborough and the visit took the doctor past Nathan's cottage. Having heard nothing, he assumed that Sarah was on the mend but naturally called to enquire. He rode up to the door and knocked on it with his whip, and the old man opened it.

 "Well," said the doctor, "how's your wife? Better, I hope." \
 "Naw, doctor," replied Nathan calmly, "her's daid, her've bin gone these dree four hour."\\
 "Dead?" exclaimed the doctor. "Did you give her the medicine as I told you?"\\
 "Oh iss," said Nathan. "When I got home I gi'ed her the bottle an' her never come round arter." \\
 "Do you mean to say your wife is dead, and you have not called the neighbours?"\\
 "Well," said Nathan, "they keeps theirselves to theirselves and I keep meself to meself, so I just put two penies on the old woman's eyes and laid her out meself, tidy."\\

The doctor, apprehensive, hurried into the house and found the old woman still alive, though exhausted. "Run, you villain," he cried to the old man, "run for your life and get me a drop of spirits. You have laid your wife out and she is still alive. Run quick, or I'll have you hanged."

Nathan, spurred into unwilling activity, shambled off and procured the spirits; and when the doctor had administered them to Sarah she spluttered and came to (in fact she outlived her villain of a husband and at last died in Williton workhouse).

The doctor taxed Nathan with his stupidity or worse. "You rogue, you know you had some fine young woman in your eye when you laid out your poor old wife like that."

"Wull, sir, I considered her were daid, surely; but when your honour rode up I was jus' thinkin' o' Susan Floyd (or other name). I was thinkin' the ole woman's clo'es would fit her 'zackly, like, an' wouldn' require no alterations."

But as for any admission of guilt, Nathan was too fly a bird to be caught. Folk had their doubts, but he must be given the benefit of them.

On another occasion he was involved in a more comical misunderstanding with the medical profession, and this time he was not to blame. Dr Hayes had been to Taunton, leaving another partner, Dr Clark, in charge, and on his return he heard that a messenger, Nathan’s grandson, had called in to ask him to come over to Luxborough at once, as “old Nathan Palmer had tumbled down dead in a fit.”

The messenger had been quite certain that the old man was dead , so Dr Clark, a busy man, had seen no point in making a special journey to see a dead man, but he had said he would tell Dr Hayes, who might then be able to issue a certificate.

Hayes said that he indeed knew Nathan well but had no reason to anticipate a sudden death in spite of his age. Clark then agreed to walk down to the police station and tell the superintendent, who in turn would inform the coroner; and a day or two later the coroner announced he thought an inquest quite unnecessary for such an old man in circumstance devoid of suspicion.

That seemed to close the matter, but a week after the “melancholy event”, the grandson appeared in Dr Hayes’s surgery in the morning.

 “How is it,” he said, “that you ha’n’t bin out to see Granfer? ‘Tis a week now since I comed in to fetch Dr Clark. Do ee plaise come to-day.
 “What in heaven’s name,” queried the doctor with some annoyance, “is the good of coming over to see a man who ‘s been dead a week!”
 “Er idn’ daid! Er’s mortal bad, but er idn’ daid. Do ee plaise to ride awver.”
 “Why on earth did you bring in such a fool’s message?”
 “Mother told me to be sure to get the doctor, as Granfer was tarble bad, and I was to tell en to make sure he come”; and on his way over the hill he had concocted a message which he felt certain would bring the doctor flying to the rescue. And when Hayes rode over late that evening he could find “little wrong with the old jackass”, who got completely well and lived to ride the countryside for years afterward.
 As old age came upon him, however, he gave up his distant journeys and spent more time at home with his (presumably) beloved wife Sarah; and the tale of one of his encounters with authority leaves a much more honeyed taste on the tongue.

Deafness had come upon him, but some of his neighbours wondered whether it might be less an affliction than an invention of his ingenuity. Once – several times, in fact, but this one was special – the squire caught Nathan trespassing and soundly berated him for the offence.

Nathan, not in the least put out, and feigning deafness, cupped his hand to his ear as if he could not hear a word the squire was saying. When the latter had done, Nathan beamed and said, “Thank ee, zir, ‘tis real kind of ee.”
 “Confound your impudence, Palmer, you’re trespassing yet again, and I won’t have it!”

 “That’s jus what I bin tellin’ ‘em, zir, an’ thank ‘ee very much I knowed you ‘ouldn’ forget old Nathan,and I han’t had a brace o’ pheasants in a long while. Shall I come down for ‘em or will ‘ee zend ‘em up ‘ouze?”
 And what could Squire do but haul down the flag and admit defeat?
 
 
\Flourish

Even after all this time it is a puzzle to analyse Nathan Palmer. Was he a none too agreeable fellow with the saving grace of humour? Or an ignorant donkey-driver with occasional flashes of shrewd wit? Who knows? The authorities saw him as little better than a scrounger but they ignored his tenacity and dedication to a free, open air life of travel; and others, notably William Symons, glimpsed in him and his father an independence which lifted them out of the toil-worn rut and won their acceptance.

And after all, to be remembered by posterity for one’s verbal passages-at-arms and rubs and clashes with unjust authority is a reputation much to be desired. 