\chapter{Bampfylde Moore Carew}

Bampfylde Moore Carew, King of the Gypsies, might demur somewhat at enlistment in the cohort of Somerset heroes, for, like Tom Faggus, he was neither born nor bred in the county, and he cared as little for heroism as Falstaff did for honour. Still, Somerset set the scene for some of the most ingenious of his mystifications, and so he must briefly show his scampish self again for diversion’s sake, but not that alone.

Besides, his birthplace was only a comfortable hour’s ride over the border; and if one looks for a riverside Paradise Garden, where in all England will to be found a more enchanting spot, a more delightful Arcadia, than Bickleigh, Devon?

Green lawns by the river, a noble old stone bridge with an inn nearby, the rushing and splashing and dashing of the waters over the weir, this   if you can ignore the cars and lorries and delivery vans   will transport you in the imagination to a simpler, more leisurely age than our own.

"Here," one may think, "is a place that no one can ever have wished to leave. Here every prospect pleases, and even the human element comes pretty well up to scratch. This surely would be Everyman's home from home."

But humanity is a perambulating contradiction, and for every Adam or Eve cast out of a rural Eden by poverty, another has broken away in search of adventure or fortune, led on by a wanderlust that would not be denied. Thus little quiet Bickleigh once produced a son who would not be confined within his parish but took the world as his fairground and roistered happily through the whole kingdom and overseas.

\Flourish 

"The Noted Devonshire Stroller and Dogstealer, the Accomplish'd Vagabond, the Compleat Mumper" is not an encomim that everyone would desire, but they pay a tribute to the skill of Bampfylde Moore Carew in living on his wits and profiting by any witlessness ready to hand.

He first sidled on to the world's stage in the reign of William III, and with a gentle push from destiny might have become either a cleric, like his father, or else a second John Law, winning a fortune from a gullible public and retiring into private life laden with "honours"; and as sure as day he would have been forgotten within ten years. But since Carew brought artistry and inventiveness to his trickery, memory has dealt kindly with him.

The name of Carew served as an introduction anywhere in the West Country, and when the babe was baptised "there never was known," said his biographer, "a more splendid appearance of gentlemen and ladies of the first rank and quality at any baptism in the West of England than at his."

His father sent the boy to be educated at Blundell's, and for a time he worked hard and "contracted an intimate acquaintance with young gentlemen of the first rank"   and perhaps these employments did not match too well.

In an incautious moment he and three other boys went out with the school hounds and the pack caused so much damage that farmers came to the school to complain. To escape punishment the boys ran away and took up with a party of gypsies. They seem to have fallen in with them some time before when they were celebrating with much jollity and feasting, and "such was the air of freedom, mirth and pleasure that characterised them" that now the young fellows "conceived an inclination to enlist into their company."
They made this known and eventually their wish was granted. They went through the requisite ceremony, took an oath of fidelity and were admitted to the company.

Two of the lads soon tired of it   baked hedgehog and roasted squirrel are after all acquired tastes   but young Bampfylde seems to have said to himself, "This is the life for me, i’ faith" and taken to it with zest. Whatever the adventure, high or low, he would be the lad for it, and he went off with his new found friends, leaving his parents in utter ignorance for eighteen months of what had happened to him.

At length he drifted back home again, but found it impossible to settle down, so much did he yearn for the life and company of the gypsies. Once more he took to the road.

He developed a quite remarkable talent for trickery and deception, and his skill in outwitting and outfacing even those who knew him personally was, in more senses than one, imposing. He was always devising new methods and stratagems for extracting money from people, playing on their sympathy or credulity, and taking advantage of them in all sorts of disguises, and so he acquired a reputation as a trickster throughout the West of England and further afield.

It all sounds very reprehensible, but from Carew's point of view the game was the thing. More often than not he revealed his true self to his victims afterwards, and they seem to have borne him very little ill will. And as a gentleman's son he would not have dreamt of insulting them further by offering to restore the spoils!

Even the slow moving justice of early Georgian England could not tolerate him indefinitely, and either the interest of the enforcement officers or his innate restlessness drove him to Newfoundland. He soon found it anything but a land flowing with milk and honey for mumpers, horse copers or anyone else, and he came home again, pretending to be the mate on a vessel, and eloped with an apothecary's daughter from Newcastle. To give him due credit, he married the girl and maintained a rough faithfulness for the rest of their wandering life.

\Flourish

He was only a gypsy by adoption, with no trace of Romany blood, but few of the nation could outdo him in craft and skill. But not for this only he won their respect, and when Clause Patch, a king of the gypsies, died, Carew was chosen in his stead.

Perhaps they also appreciated that his literacy would give them an unsuspected advantage when dealing with the gorgios. Other less tolerant persons, however, showed no appreciation of his rank, and they packed him off as a vagrant to the plantations in Maryland.

To dump an unwanted king of the gypsies in the plantations was one matter; to hold him there, the Marylanders found, was quite another.

He escaped. They recaptured him and forced him to wear an iron collar. He escaped again, and this time fell into the hands of friendly Indians. Whether he taught them the three cup trick or learnt new japes from them is not clear, but they removed his iron collar, and thus freed he came into Pennsylvania, land of Quaker virtue and honesty unparalleled!

But Bampfylde, far from being overawed, found devious inspiration in this. He made himself out to be, of all persons, a Quaker, and as such he rode or footed it back to New York and thence to New London, Connecticut, where he took ship for England.
 He may have had the honest colonists fooled, but now he came up against a tougher lump of humanity, the press gang, as ready to believe the worst of a man as the Quakers would the best, and even though his dress proclaimed the Friend, the gang would have quickly unmasked him.

He had to think fast and act faster. Ever inventive and at his best in a emergency, he pricked his hands and face all over, rubbed in salt and gunpowder, and there, for all the gang world to see, was as pestilent and devastating a case of smallpox as might be met with in a month's sail, one dreadful enough to keep the whole Royal Navy a cable's length away.

In England again, he sought out his wife and daughter, wandered up into Scotland and then down to Carlisle and Derby with the '45, no doubt relieving the Scots of some of their unaccustomed plenty   though the Highland Scots behaved impeccably toward the English, a kindness which the English repaid with signal brutality the following year and after.

Carew seems to have escaped any accusation of involve- ment, and if he had given any thought to the matter he would certainly have chosen a prosperous, stable society with unlimited possibilities of loot rather than a Jacobite cause doomed from the start.

Over the next few years he enjoyed more jaunts and jollities than  Jorrocks ever did, among strange lands and nations whom he diddled proficiently, and the people of the West Country, his boyhood home, supposedly sleepy but shrewd, proved as gullible as any other. Spacious living does not entirely depend on the rolling prairie or the trackless wilderness, and 250 years ago the 500 square miles of the Exmoor country offered as many opportunities for a man of ingenious mind as the 500,000 of the American colonies.

That, at least, seems to have been Bampfylde's view, and for one who had taken the whole of the English speaking North American seaboard as his oyster, the time he gave to the pleasant pastime of extracting pearls from incautious natives of the West Country was quite out of proportion to their numbers or the area. But perhaps the West Country should take it as a compliment, for Carew, in his tricks and drolleries, possessed that inimitable quality: style.

True, he had a living to earn and a family to support   if a little erratically   and when the wolf howled at the door of his caravan, no doubt he took whatever, wherever and from whomsoever he could. But I suspect that this hunger driven, tight-corner Carew was rarely forced to show himself, for he and his companions had long perfected the art of living off the land; and if he returned time and again to the West Country, he was probably drawn not only by the familiar scenes of childhood, but also by the challenge of familiarity.

Pulling wool over the eyes of an unsuspecting stranger would shoe Carew’s horse and grease the axles of his waggon, so to speak, but it was poor stuff and unworthy of a King; but to persuade a man of many years' acquaintance that he had never seen you before and hoodwink him again in a different guise a day or two later   that showed the artist!

Unquestionably his family name helped him much more than he had a right to expect. Any lustre that he was adding to it must have seemed irremediably tarnished in the eyes of the respectable members of the clan, but others forgave him much for the sake of the name.

Some time after his return from Newfoundland his travels took him to Watchet, and he decided to pay a visit to Sir William Wyndham, at Orchard, two miles away, but not as the scion of a friendly family: that would have been unpardonably simple and straightforward. He must act the shipwrecked mariner.

He donned a jacket and pair of breeches and made his way over to Orchard Wyndham. Lady Luck, the patron of successful "wide boys", had put in an hour's warming up on Carew's behalf, for instead of being obliged to work out some devious route from the servants' quarters to the Presence, he fell in with Sir William in the park, walking in company with Lord Bolingbroke and several other gentlemen and clergy and captains of vessels   a fairly awe inspiring conglomeration for most petitioners, but for Carew a spur to ingenuity.

With a convincing show of timorous respect he approached Sir William and engaged him with a yarn that he was from Silverton, near Exeter, and the son of one of Sir William's tenants named Moore. (Sir William owned a substantial part of this parish, which adjoined Bickleigh, Carew's birthplace). Carew went on to say that he had been to Newfoundland and that on his passage home the vessel had been run down by a French ship in fog and only he and two others had been saved. An Irish vessel had taken him to Ireland, and thence he had crossed to Watchet.

So far the story was typical of any vagrant who had taken the trouble to enquire into Sir William's connections, and he was not to be lured into believing an easy tale.

He tested the "shipwrecked mariner", asking him a great many questions about the inhabitants of Silverton and the principal gentlemen of the neighbourhood, and with Carew's local knowledge these gave him no trouble at all.
 At last, coming nearer home, Sir William asked him if he knew Bickleigh and its parson.
 "Very well, your honour," replied Carew, solemn as twenty judges (he did not add, "'Tis my own father.")
 "And what family has he?" asked Sir William. "Has he not a son named Bampfylde? And what has become of him?"
 "Your honour means the mumper and dog stealer," replied the virtuous Carew. "I don't know what has become of him but"   censoriously   "it is a wonder he is not hanged by this time."
 "I hope not," replied Sir William good naturedly. "I should be very glad, for his family's sake, to see him at my house."

A common trickster, one without Carew's artistry and sense of timing, would have taken up Sir William there and then, made him justifiably offended and, for the sake of a petty triumph, spoilt the story.

Bampfylde knew better. After a few more questions Sir William relieved his imagined distress with a guinea, Lord Bolingbroke followed suit, and the rest of the company contributed according to their rank, being all the more inclined to do so as the captains found he could give a very exact account of all the settlements, harbours and most noted inhabitants of Newfoundland.

Sir William then ordered him to go up to the house and tell the butler to provide for his entertainment. So away he went, conveyed the message and sat down with considerable relish.

At that moment Lady Luck, feeling she had done quite enough on his behalf for that day, signed off   but not before ushering in another character, a foot postman from Silverton, of all places, with letters for Sir William.

Like the guests in another drama, Carew waited not upon the order of his going, and made haste to put several miles behind him before anyone should raise the hue and cry. In fact, no one did, and only the chanciest of encounters betrayed him; and even then the affair turned out better than he could have reasonably hoped.

A little while after his hasty departure he met a Dr Poole, who was on his way to Orchard Wyndham but knew Carew and stopped to speak to him   naturally without eliciting any compromising information. Later that day at Sir William's he happened to mention this encounter, and his description made it clear that it was none other than the Silverton mariner the gentlemen had generously helped.

To their credit, the episode raised a storm not of annoyance but of laughter! But Sir William was not finished with Carew yet.

About two months later, of course knowing nothing of the doctor's revelation and perhaps a little over confident, he returned to Orchard Wyndham, this time in the dress and character of a grazier who had fallen on hard times.

He met the baronet and his lady as they were taking the air in a chaise, in a meadow where hay was being mown. His approach was all honest simplicity, and he began a moving account of the misfortunes he had met with in life. But Sir William cut short his eloquence by calling on the haymakers to seize him and hold him fast, and in an instant the farce seemed likely to turn to tragedy.

Sir William, however, gave him a choice: he must either confess his true name and profession or be committed to prison.

Bampfylde chose the first and confessed that he was Moore Carew, "sovereign of the whole community of mendicants", and thereupon Sir William, with a good deal of humour and good nature, treated him with the respect due to royalty, entertained him generously at his house and made him a very handsome present on his departure, inviting him to call again whenever he passed that way.

Perhaps the King of the Beggars went away in a slightly more sober and reflective mood than when he came. 

Perhaps.

\Flourish 

He practised his tricks at other places in the West. Once he and a companion were dared by the landlord of an inn near Porlock to spend a night in a haunted house. Bampfylde could not resist a wager, and this one offered him lodgings free of cost, together with a good meal.

They went to the haunted house with a farmer's son who had taken it into his head to tag along, and Carew gave good value for his part of the wager. By various contrivances he nearly frightened the farmer's son out of his wits but in the end convinced him that he had exorcised the ghost! And on this account, they walked away next day richer by their lodging, a good meal and 20 shillings in ghost layers' honorarium!

Another time in Porlock he sat for his portrait to William Phelps, a local artist of fair celebrity. From there he went on to Minehead and in his true self called on various gentlemen, including the doctor and the parson, who treated him very kindly.

That was no insurance. Some days later at Timberscombe he met a gypsy woman he knew who had a young child with her. He borrowed the child and one of the woman's gowns and a petticoat, put them on and went back to Minehead with the child in his arms, among the gentlemen he had recently visited, and pretended to be an unfortunate woman whose house had been burnt down. Coughing violently during his appeal and making the child cry for good measure, he got money and victuals from his benefactors, and they seem to have had no suspicion whatever.

We may wonder, why go to all this elaborate pretence? But Carew would quite shamelessly have answered, "The play, the play's the thing."

Thoroughness was his hallmark, never more than in another chapter of the "shipwrecked mariner" saga when, happening to be near Portland, in Dorset, he heard one evening of a ship driven on to some shoals and in imminent danger of breaking up.

Early in the morning, before broad daylight, Carew stripped off his clothes, flung them into a pit and swam out to the vessel unseen by anyone. He found only one member of the crew alive, clinging on to the side of the vessel and hanging between life and death.

In grave danger himself, Carew kept his head, expertly combining humanity and profit, offering the sailor his help to get ashore and, in the same breath, enquiring the name of the vessel and master, and what cargo was on board, whence she came and whither bound.

The sailor gasped out these details, but even as Carew was urging him to let go his hold and entrust himself to his care a large wave crashed over the ship and swept the sailor away. Carew could do no more. With great difficulty he struggled to land and, with one arm injured, was thrown violently on to the beach.

By this time a crowd had gathered, and when they saw Carew naked, spent with fatigue, injured and apparently the sole survivor, he raised "a tender feeling of pity in them all"   especially in a certain lady's housekeeper, who pulled off her own cloak to give to him, bound up his wounds with her own handkerchief, took him home and seated him before a good fire with two large glasses of brandy with loaf sugar in it.

She then went off to tell the lady of the house, who, no less affected than her housekeeper, ordered a bed to be well warmed for the "shipwrecked sailor" to be put into and taken the best care of. Carew lay there quiet for three or four hours   genuinely, one may guess   then waking, he seemed much disturbed in mind, groaning, tossing from one side of the bed to the other but nowhere finding ease.

The good people then brought him a suit of clothes and he got up. They told him that the bodies of his shipmates had been washed ashore, and he seemed greatly affected and tears started from his eyes. A visiting J P, a Mr Farewell (an apt name, in the circumstances) gave him a guinea and a pass for Bristol. The local people collected some £10 (equal to several hundred to day), Farewell lent him his own horse to ride as far as Dorchester, and the parson sent his man to show him the way.

The well wishers were not to know that Dorchester was not in Carew's way of thinking at all. Only five days earlier he had appeared there in the character of a ruined miller and raised a contribution from the mayor and corporation; but with a guide at his elbow there was nothing to do but ride in and trust to brazen impudence to bring him away in one piece.

It did.
 As soon as they arrived, the guide presented the pass on behalf of Carew to the mayor, who ordered the town bell to be rung and assembled some of the members of the corporation.

Whether they were short sighted or short memoried one cannot say, but with Carew in different dress and provided with a pass from Justice Farewell certifying that he was a shipwrecked mariner, not one of them recognised the broken down miller of the week before... They treated him with great kindness and relieved him very generously, and he went on his way cheered with a great many good wishes for his safe arrival in Bristol.

But having tried his luck thus far he knew better than to tempt it Bristol wards, where a messenger might catch up with him. He steered his course toward Devon, raising contributions on the way from various squires and officers.
 At last he settled down   at least, he seems to have, though no one is quite sure. A relative is said to have offered to provide for him if he would give up his wandering life.

He refused, the call of the road was too strong. But eventually he won some prizes in a lottery, apparently enough to bring him a small but adequate income. Maybe he thought over his long career of dishonesty. Maybe he thought over his long career of dishonesty and formed a lower opinion of himself than before. Maybe the thought that "the play's the thing" no longer guarded him against an attack of conscience.

Or maybe, in spite of it all, he did not settle down but went on with his wandering life, a bedraggled king of the gypsies and beggars, until his skill deserted him and he sank into obscurity among the vagrants and homeless.

He is said to have died in 1770 at the age of 76 or 77, but who knows? One suspects that he would really have wished to end ‘sensationally’, as did Rex Harrison’s Rake in a wartime film  , ‘The Rake’s Progress’ (1945).

Conway, a rake  in the 1930s, could have had access to  documents excusing him from military service, but instead he enlists in the army, and in the campaign for Normandy  steers his armoured car straight at a German strong point, not counting the cost.  The film ends with his body lying in a dressing station, and a senior officer remarks, “We owe it to men like him that we have made it through Normandy.” 

   Sergeant (peace-time friend):  Yes, sir. He died as he liked to live, driving a car he hadn’t paid for.”\\
   General:  I consider that remark in very poor taste, sergeant.” (Exit)\\
   Sergeant (to body) :  Well, you ‘ld have appreciated it, wouldn’t you, old son?\\


    
