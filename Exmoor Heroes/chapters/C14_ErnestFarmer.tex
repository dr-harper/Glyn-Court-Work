\Chapter{Ernest Farmer}{95 as far as “young GI”}

Up in the far corner of Somerset, Ralph Allen ‘did good by stealth, and blushed to find it fame’, but there was nothing stealthy about Ernest Farmer’s philanthropy. Day by day he was highly visible at his work, which involved making King George’s highway, for a stretch of three miles across Brendon Hill, a little less perilous for man, beast or vehicle. He did not gain much honour or security from it, in blatant contrast to his counterpart in France, the cantonnoier, who had also had a length of road to maintain but was provided with a hut, hammer, stone gauge, hard-wearing uniform and peaked cap and a small right of toll. He was recognized and valued by the community, so much so that when Napoleon praised a sergeant for gallantry and asked what his reward should be, the man replied, “Make me cantonnier for life in my own village” It was done. 

Ernest, working for the council highways department, received no such regal recognition, nor indeed any but of an occasional inspector and of course his neighbours. Even they were few, for his parish, Withiel Florey, was lightly populated . He had no known relatives in the neighbourhood and no cottage of his own, but lived in a hut on the south side of the Brendon Hill ridge road a little east of the turning for Gupworthy. 

A mystery surrounded his birth – or was it that the more charitable tended to forget what was whispered by the scandalmongers? Be that as it may, in 1939 he was working to maintain his stretch of road, lay the hedges and keep the ditches clean and well drained, and war did not make the impact on his daily life that it did on city dwellers. Cars and vans were fewer, and mainly connected with farming , forestry and trade – though there were also individuals, too many, who managed to circumvent the fuel rationing system. The war only erupted in 1940, when German bombers flew over from Brittany and Normandy to attack the cities of South Wales. Then the searchlights at Leighland and Doniford would shaft up into the blackness, and the sky to the north glowed threateningly night after night with the fiery destruction in Cardiff and Swansea.

But at last, in the summer of 1944, Brendon Hill was invaded, and the occupiers were American soldiers, and not in single stragglers but in battalions, 15,000 of them, all of them waiting for the signal that would send them to the invasion of Europe. They were held in two separate camps, white GIs on Langham Hill, coloured on Treborough Common. They were here for several weeks, and in their off-duty hours the whites descended on Roadwater, made their presence felt, spent freely, organised parties and socials, gave their rations to help with the scanty civilian diet, and were generally appreciated. They can hardly have seen much of Ernest or given him more than a friendly wave from a jeep, for their evening trips took them away from his stretch of road. 

But the evening hours in the village, where the GI s gave so muc of themselves, were far outnumbered by those spent in the harsh preparation for D-Day, with marches of fifteen to twenty miles in battle order, along the ridge road to Cutcombe and beyond, and then back again, running at the double with a 60-pound pack from Wheddon Cross to Langham Cross and on past Ernest’s working place, and none of them now had the time or the energy to give him a wave.

Marches of this length were the lot of the infantryman in every nation, but his “curses not loud but deep” were balanced by his pride and satisfaction in beating whatever the army could fling at him. Fitness and a strong heart would carry him through. 

But the heart of one young American had a weakness that none of the army doctors detected, and on one of these marches, near Ernest’s dwelling, he collapsed by the roadside and did not rise again. They gave him a temporary burial in the camp until a place could be found elsewhere. 

Ernest took very much to heart the tragedy of this lad dying so far from home with nothing to show for the years of hard training and physical exertion , and with all the promise of his life unfulfilled. A few days later he appeared at the gate of the camp and asked the sentry to take him to the commanding officer. The sentry was inclined to turn him away, but Ernest explained his errand and he was escorted to company headquarters. The captain listened at first incredulously, then with sympathy and appreciation, and they took Ernest to the grave of the solder and another who had died of pneumonia, and he laid on the graves a bunch of flowers he had picked that morning. 

Soon the invasion began, and the 15,000 soldiers and their equipment and tents were all moved out by truck, with near-miraculous efficiency, in a single night; but Ernest went on tending the graves until a work party came to remove the bodies for reburial elsewhere. Some time later, when these events had moved away into the background, something happened to show Ernest that the Americans had not dismissed him as just a simple English countryman. The captain, writing the customary letter of condolence to the young soldier’s parents, had told them of Ernest’s kindly thought and action, and they, as moved by this as the captain had been, wrote him a letter of sincere thanks, with a gold watch in memory, followed in due course by a substantial sum of money. 

To set a seal to this story, it emerged that the boy had died in the very parish from which his ancestors had emigrated to the United States generations before: and as his compatriot T S Eliot had written of another place in Somerset only three years before, “In my beginning is my end : In my end is my beginning”, no less true was this for the young American soldier.

\Chapter{"It Snowed, it Hailed: They Never Failed}{}

‘Neither rain nor sun, nor heat nor gloom of night, stays these couriers from their appointed rounds”.
 Those words of the old Greek, written in honest admiration of the couriers of an enemy, have always appealed to me. They can have meant no less to Benjamins Franklin when he adopted them as the motto of the nascent U.S. postal service, and time has proved his trust.

The Exmoor postmen of my boyhood days would have shifted uneasily at such outspoken praise, but they earned it - leaving Apaches, outlaws and dust storms out of the equation - almost as fully as any heroes of Wells Fargo or the Pony Express; and the comparison is not fanciful, for I remember them coming back to the office from their rounds in winter with faces and hands blue with cold, and stories of struggling through driving rain and howling storms, wading thigh-deep through snow-drifts or even walking on top of the hedges to deliver the snowbound Brendon Hill farms. 

The Somerset post office in which I grew up took the responsibility, in fact, for an unusually wide area - perhaps uniquely so in Southern England. It must have covered at least 150 square miles. Two postmen on bicycles did the round of the village and immediate neighbourhood - two deliveries, of course, morning and afternoon - and two others went very far afield. One in a Morris Cowley van served the post offices of the Roadwater Valley and the farms of Brendon Hill, while the other delivered to the offices of the east side of Exmoor - Dunster, Timberscombe, Wheddon Cross, Exford, Winsford and Withypool - and some of the most cherished memories of my boyhood are bound up with such post office string and sealing wax. We did not handle thousands of letters, but all the mail to and from the moorland villages up to twenty miles away passed through our hands for distribution in the Exmoor villages and farms by local letter-carriers such as Mrs Mary Elizabeth Hooper of Withypool, who served 33 years from 1916 to 1949, never failing even during the appalling winter months of early 1947. 	

Looking back on it, I realise that our sub-office under Taunton employed quite a team : my mother at the head, with two office girls to carry out some of the postal business and serve the shop, and no fewer than five postmen - if not all at the same time.

At the head of the outdoor phalanx marched Silas Locke. With many years’ service and then, in his early sixties, approaching retirement, he had carried over into the 1930s something of the dignity of his profession in more spacious days. He had been born at Frogwell, Upton, on the Brendon Hills in the early 1870s, and he brought down to the valley something of the breezy tang of the heights, with humour never very far from the surface.

In his heyday he had driven the post office pony and trap, which I can just remember, but about 1930 this was superseded by a motor van and he ended his career on a very aptly named “push-bike” which took in the 1200-foot climb to Brendon Hill. But whatever the weather, we always knew he had returned when we heard his cheerful whistle of the Lincolnshire Poacher. Maybe he had other tunes stored in his capacious memory, but I never heard any, and certainly none could have better expressed his sturdy, cheerful independence.
 
When off duty he wore a cloth hat, tweed suit and leather leggings polished mirror-bright, and walked the village street with sprightly step on one of his errands - to cut a neighbour’s hair for a “tanner”, say, to lend a hand in the harvest field, or to enjoy a chat with kindred souls on the bridge over the old mineral line and hold forth on the failings of the feckless younger generation:(though I must regretfully edit) “They bain’t wo’th a hounce o’ hoss-dung.
 
 He delighted in cricket, and in his retirement he often went by ‘bus or train to Taunton to watch the county team, and on Saturdays he umpired in local matches with good-humoured firmness:
 
 “’Owzat?” roared an exultant bowler at Roadwater. \\
 “Out l.b.w.,” pronounced Silas.\\
 “I weren’t l.b.w., and I bain’t out,” protested the batsman.\\
 “Now don’t ee arguy wi’ me, young man. When I say you’m out, you’m out; an’ if you don’t believe me, read nex’ week’s Free Press .”\\

\Flourish

After Silas Locke, two other postman, Jim Ridgway and Sam Taylor, shared the local deliveries. Jim was of the same generation as Sam and his round was restricted to the village, but although he limped badly, as his right leg was as stiff as a broomstick, he covered the ground as fast as a man half his age. To accommodate him in his customary place in chapel the stewards had sawn out the lower part of the pew in front, so that he could sit with his leg stretched out in comparative comfort, and that little alteration remains to this day.

Jim gave his off-duty hours to his garden of which he was rightly proud, but not only his own, for with a little financial encouragement from the owner of Washford mill, Arthur Badcock, he took over a few yards of steep road-bank at the foot of Tom Smith’s (or Station) Hill and made it into an attractive rock garden bright with summer flowers, and so it remained for many years. 	

Sam, in his younger days, had played half-back for the village, and a group photograph shows him, shortly before1914, as an alert, already soldierly figure with a waxed moustache worthy of an R S M. By the 1930s he had lost some of that martial aspect, but the war had left memories not easily erased. One day, I remember, the conversation turned to one of the perennial topics, the two sisters who lived in an old cottage nearby, in such squalor and so fetid an atmosphere that it drove you back from the open door. Annie was harmless and dominated by her sister, but Rhoda was a vixen. After one of her outbursts I heard Sam growl , “I’ve killed better men than her.”

His rounds by bicycle took him to the outlying hamlets for two to three hours in the morning and a shorter time in the afternoon, and so it was not too demanding by post office standards and left him time for other work in his allotment. I am sure he was as mindful as his masters of the rule forbidding drinking while on duty, and if he showed an inordinate liking for violet-flavoured cachous, perhaps a sweet tooth explained it. 

\Flourish

These reverend seniors still covered considerable distances, but the longer rounds by postal van were in the hands of younger men, both ex-Army and just as dependable as their elders. 

Jim Leigh and Tommy Atkins were very dissimilar in character, Jim tall and easy-going, Tommy diminutive and very much of a stickler for the rules - but with good reason, knowing that with the strict discipline of the service a postman’s career was balanced on a knife-edge - and so it was with Jim that I was sent as a boy on the unofficial and magical journeys over Exmoor. I have never forgotten them. I think I never shall, for more than any other experience they revealed to me the wonder and mystery of the world. 

Not that I experienced for myself the harshest conditions in which the moorland postmen worked, but I would see them in mid-winter, when the snow lay drifted on the hills, coming back to the office with hands red and raw with cold, and hear them speak of wading thigh-deep through the snow or walking along the tops of the hedges to deliver the mail to the farms on Brendon Hill, or floundering in drifts as high as the five-bar gates. A boy of nine or ten would have been a nuisance, but on other days, just occasionally, I was allowed to travel in one of the vans, and those were great mornings.

Bundled into the front seat of Morris van, I would set off with the postman through the still sleeping village. At half-past six the winter darkness lay heavy on the land, but as we rumbled through the streets of Dunster, gleams of light showed above the castle in the east; so we wound our way, the van protesiing and boiling over, up to the high moorland and the Rest and Be Thankful, a thousand feet above sea level. Then Dunkery Beacon might catch the early light and stand out proud and clear, and as we sped along the straighter roads over the moor the beech-leaves still clinging to the hedgerows turned from grey to sombre brown, and from brown to a rich russet. After nearly an hour we would drop down into Exford, simpler then than now, and in Mrs Wensley’s post office tea would be waiting and I would hear, sometimes, talk of strange happenings on the moor, mysteries of birth and death, or strayings, of disappearances that were disturbing and inexplicable. Then we would go on in the daylight to other places, Chibbet’s Post, Withypool, Winsford, which to a village boy might have been at the ends of the earth but for the mail van which in this unforgettable way opened up the world. 

On other days we travelled the Brendon Hill route, shorter and at that time less exciting, but in the long run it has meant even more. Wherever you travel in West Somerset you must go up hills, and on this round we climbed to 1300 feet, most of that height in a couple of miles; and ever since I travelled with Jim, the hlls have drawn me inescapably.
 Our first three miles were ordinary enough, for I knew the valley; but then the adventure began, and the new names, new to me but old with history - Pittiwells Lake, Langridge Mill, Holywell, Greenland, Culverwell, Luxborough, Chargot, Swansey - struck a chord in me which has never ceased to vibrate. The road led along a hillside through a sombre pine forest. Jim would stop to deliver to the mill, and I would hear the soughing of the wind in the pine tops and the murmur of the invisible stream down below, and Jim would come back and say, “That’s the way up to Greenland”, and my imagination would pictures that abode of ice and snow up on the hillside.

Then, as he prepared to coax the van up the one-in-five hill to the summit, he might say, “This is Veller’s Way - some say ‘tis Felon’s Way, when they hanged them in the old days,” and I wondered vaguely who the felons were and what their crime had been. “Sheep-stealing,” said Jim to my unspoken question.

After that we came to the strange world of the Brendon uplands, the wide beech-lined roads, the lonely farmsteads down their long, long lanes. the reserved, deliberate farmers and their wives, the measured but genuine welcome, the log fires and the home-made bread : all so far removed in every way from our life down in the valley. It was only eight miles away but a far country - and yet one in which, inexplicably, I was made at home. 

In time the Exmoor round was taken away from our office and Jim and Tommy were moved, but to take charge of the Brendon Hill deliveries Will Bellamy came along - and to know Will was the beginning of a liberal education.	

With his active mind he might have gone far in life, and far also from this quiet corner of Somerset; but Will had a widowed and invalid mother to care for, and opportunity passed him by. They lived in a thatched cottage dating from Elizabethan days and formerly a farmhouse, with internal pargeting and plasterwork which, for those who could see though the layers of whitewash, spoke of ampler fortunes. With a kitchen garden to supply some of their needs, he made do for them both on the modest postman’s salary. I doubt whether he had much left over for the purchase of books, but his chief interests were politics and foreign affairs, and in those days the News Chronicle served as almost the only outspoken daily guide to such matters for the thinking person (man, woman and child). Almost alone, while the rest of the press bowed to Hitler and Mussolini, snatched at every conciliatory crumb from their tirades, and hoped, with little evidence, for “peace in our time”, the News Chronicle warned the British people of the threat of the unholy Axis of Germany, Italy and Japan, .

How well I remember Will’s keen, intelligent, generous face, quickening with indignation as he denounced the latest humiliation suffered by the democracies; and his manner of speech had this peculiarity, that indignation and resentment of the folly of our political masters did not bring an outburst of anger: instead, he spoke urgently and rapidly but almost in an undertone, as if the disgrace to the name of our country should not be spoken abroad. Indeed, not only in Central Europe but very near home he felt cause for concern, as he recounted his visit one morning to a nearby manor farm occupied by a naturalised German. Will, conversing with the son of the house about “old Hitler” and his broken promises, was taken aback when the son peremptorily and teutonically interrupted him with “You must not say such things about our Führer!” And when the father, a year or so later, was given commanded of a platoon of the Home Guard, there were others than Will who wondered just how much resistance that particular unit would have been allowed to offer to an invader.

Will’s interests went beyond Europe, however, and every week he tuned in to the “Letter from America” broadcast by Raymond Gram Swing. I am sure he rejoiced in Roosevelt’s drive and reforms, and he also listened with an ear for language. “It’s interesting, the different ways that we and the Americans have of saying the same thing, “ he remarked. “We say “biscuit”, they say “cookie”; we say “lorry”, they say “truck”; they say “viewpoint”, we say “point of view”.” He would have enjoyed the American Usage of his North Devon contemporary Herbert Horwill, but it probably never came his way.

For all his awareness of the darkness at noon gathering over Europe and the free world, he kept a firm faith in the future of mankind. Once on my way home from school I bought a book by H. G. Wells from the station bookstall - I forget the title, but it was published in the self-styled “Thinker’s Library”, hot - or ice-cold, rather - from the rationalist press. I lent it to Will, and a few days later he returned it in great agitation: “That history by Wells, he’s so pessimistic, and he’s no right to be.”

Soon after, like most of the other young men in the village - for none of us were scientists - I left home for the Forces. Six years went by before Will and I met again, and then it was for the last time, when he brought me a wedding present of a water jug and set of glasses, which I have treasured ever since for the memory of a man whom, above many, it was good to have known.

There were of course many more “gentlemen in blue serge” known to me only slightly, if at all; but they all gained and retained an enduring place in the memory of their fellows, whether for some feature of character or because of what they contributed to the life of their community outside their working hours:
 
Jack Marlow at Roadwater, for example, the ex-sergeant who took his martial bearing and strict discipline and Old Bill moustache into his civilian career, and even in his last illness sat upright in his bed and grasped his walking stick with masterful hand; and in another mould and another sphere, Albert Coles of Luxborough, like his father, the legendary Sammy, an indefatigable walker and preacher. Six days of the week his round took him fifteen miles on foot, but most Sundays, instead of lounging at his ease, he would be off over the hill country, climbing the 1200 feet of altitude and walking perhaps the ten miles to Exford or twelve to Dulverton, conducting two services and preaching two sermons, and then ten miles home again for a late supper and short night’s sleep before another toilsome week.

Different again were four postmen-musicians, part-time but dedicated, William Venn of Williton, violinist and orchestral leader in every concert in the days when village folk made their own entertainment and melody ruled; Fred Bond of Timberscombe, violinist and leader in the many concerts organised by yet another postman, Jim Slade of Roadwater, and his successor Walter Lile, tailor by trade, far from robust after being gassed in the First World War, but still good for ten foot-miles a day, a competent violist, devoted follower of Gilbert and Sullivan and chapel organist, without fee or reward, for nearly half a century; 	

On the seaward slopes of Exmoor Frank Glanville tramped his daily round. He entered post office service at Cullompton in 1910, but the War put a stop to that and he could not pick up the thread again till 1920. Then he resumed his career at Porlock and travelled the moor for fifty years.”My duty round,” he said, “was officially sixteen miles : it started up Porlock Hill and continued in a series of ups and downs, never many miles from the sea. Some years later a colleague brought along a pedometer, and we found that the real distance was a few yards over twenty miles. But I was used to an outdoor life, and so much walking made me very robust, so that I thought nothing of walking as much as thirty miles a day, carrying on my delivery route in spite of snow, ice, rain and wind. Many is the loaf of bread I have taken out to dwellers on the moor cut off from supplies by the weather.”

A lowlander by birth, he learnt the ways of the moor, knew which sheep belonged to which farmer by its markings and was able to tell the owners if any were in trouble or had been killed. The wild animals, too, engaged his sympathy: Once he helped a hind whose calf was stranded on the other side of a high fence. Gently Frank picked up the calf and placed it down beyond the fence, then walked backwards and watched. The hind smelt the calf, licked it, and looking at her helper, gave a short, sharp bark. as it to thank him. 	

The Second World War gave him rough and trying times with the military training on Exmoor. Hardly a day passed without shells hurtling overhead, and once, when he misunderstood the instructions of an American artillery major, he came under fire from shells bursting about fifty yards away; and he also recorded, with some indignation, a massive stag-shoot, when jeeps drove the deer across the moor to rougher ground where men with guns were hidden.

Still, the memory of all he had seen on the moor - the abandoned cub he had rescued and fed, the badger digging out a wild bees’ nest, the man in black above Culbone who vanished before his eyes - might have been lost, but that he wrote it down while it was clear in his mind, and his notes, brief but telling, appeared under the name of “Afghan” in the local press and the Exmoor Review. Best of all, he had a true ear for the local “ way o’ spaikin’ ” and preserved such gems of wisdom as this heard after threshing, “Zider do make ‘ee go vor, an’ beer ‘ll make ‘ee go back. I bin drinkin’ both o’ em, zo I reckon I’ll bide.”
	
That should be the right note on which to bring this chapter of wayfarers, and this book, to a close; and looking back at those far-travelled postmen I am struck by the fact that all were men of character. Perhaps it was something in their work that made them so, for while the discipline learnt in the army bred in them self-reliance, self-discipline and self-respect, the freedom of their open-air life was controlled by the responsibility they bore for the mail entrusted to them and for the reputation of a great public service. At all events, they were men held in respect and to know them and learn from them was a privilege I have always valued.

Why, after that, I have left Jack Lyddon to the last I cannot explain, unless it is because, more than anyone, he exemplified an old type of countryman, trustworthy, independent, salty of speech, and down to earth but not to be trampled on. 

When he retired after thirty-five years of sixteen miles a day - the equivalemt of walking nearly seven times round the globe - they paid tribute to his devotion to duty and his strict adherence to regulations. That was fair enough, but let it not be thought that this strictness was reflected in Jack’s everyday dealings with his clients; indeed, I hope I am not wronging him in suggesting that, whether consciously or not, the punctilious fulfilment of his duties was a kind of insurance policy to safeguard his job when he dared to stand up to some insolent “petty tyrant of the fields”. He spoke his mind and feared no man - and no woman, either, which was a tougher proposition, for one or two who made their lairs in the big houses might have persuaded Attila to slink furtively by.

A public servant Jack may have been, but servile and mealy-mouthed he certainly was not.

This was back in the days of King “Teddy”, and whether the Major ordered his daily paper by post to make certain of getting one (one could be sure of delivery in those days) or simply to annoy and “aggravate” the postman, was not known, but Jack believed the latter, and when, one warm, drouthy day in summer he toiled up the long drive with the Major’s pennyworth of Fleet Street blue, he was in no mood to put up with brusque orders or peremptory commands.
 The Major, though, knew no other way. He came out holding a letter.
 “Here, my man,” he barked. “Stick a stamp on this for me.”
 Jack glared

“I bain’t your man, for one thing, maister,” he growled, “I be me own; an’ for another, I bain’t goin’ to lick the king’s backzide for ee, thee’st can lick en theezelf.” 

Inevitably the Major complained, but Jack, interviewed by the postmaster, denied meaning to offend. “I didn’ say nort out o’ plaace, you. A stamp have got a front zide an’ a back zide, hab’m er? Whatever ‘s thik feller gettin’ upzot about? Can’t zee it mezelf.” 	 	

That seems to have been the end of it, and I guess that by the time the rabbit came along he had acquired something of the status of an ambulant philosopher and jester to whom a certain latitude is allowed. 

In those days freshly shot rabbits could be sent by post, unwrapped, simply with a label tied round their necks, and the goods, as I remember, could be messy.

Generally the dispatcher, not the deliverer, was to blame, but the lady receiving the rabbit seemed not to realise this. “Oh, postman,” she complained to Jack, “that rabbit you brought yesterday was in a terrible state. He must have been a long time in the post and he bled all over the place.”

“Very sorry to hear that, ma’am,” said Jack peaceably, “I’ll zee what us can do nex’ time.”

Rabbit the Second duly arrived, a nifty specimen in more ways than one, and Jack kept his word.

He rang the bell and the lady came to the door.

“Yere ‘s another rabbut fer ‘ee, ma’am,” said Jack, “an’ I ‘ve putt stickin’- plaister awver the nawse an’ eyes o’ en an’ stuck a cork up ‘is aass-‘ole . You ‘on’t have no trouble wi’ he.”
 
\Flourish
	
It astonishes me - though on second thoughts it should not - that the ladies in charge of those scattered Exmoor post offices should have made so strong an impression on my mind that even now, sixty years on, I can see them standing clear in the golden light of morning.

To tell the prosaic truth, though, it was in the haggard grey of early winter mornings that by travelling in the mail vans I first made acquaintance with Exmoor, for that was about the only time when the driver could be reasonably sure of not encountering an inspector and suffering at least a severe reprimand for carrying a passenger, even a small boy. 

The office at Roadwater was kept by Mrs Selina Mear. She had been a widow for more than thirty years, but while long hardship had fortified her native determination it had not hardened her heart or changed her cheerful welcome. 

Before her marriage in the early 1900s she had been lady’s maid to the wife of Dr Glyn Handley Moule, Master of Ridley Hall, Cambridge, and later Bishop of Durham, and to the Misses Rashleigh of Menabilly, Cornwall, and the good breeding she had observed in those houses had left its mark in her manner and speech.
	
At Luxborough, little, busy Mrs Shopland reigned in the front room of her cottage, which you reached by a footbridge across the stream, and although we arrived very early with the van, it was never too early for her to offer a cup of tea.

	The comptrolleress of His Majesty’s mail at Exford, Mrs Wensley, ran her office as efficiently, in the main, as to satisfy that most captious and curmudgeonly of inspectors, the Man from Minehead - but for one detail.
	
“Well, Mrs Wensley,” he said at the end of a long inquisition, “you have a well-run office, but just one thing needs attention.”\\
	“And what may that be, sir?” she enquired courteously.\\
	“Well,” said he, “the only fire-fighting equipment you have is a bucket of sand. What would you do if your telephone switchboard caught fire?”\\
	“Du? Du?” cried the lady, exasperated by the recollection of weary hours at that same exchange putting through futile two-o’clock-in-the-morning calls from late revellers and hunt ball drunks, “I’ll tell ‘ee what I’d do: I’d let the darn thing burn!”

It remains a mystery and a regret for me that I do not clearly remember Mrs Veysey, the postmistress of Winsford, but I am sure a few others will.