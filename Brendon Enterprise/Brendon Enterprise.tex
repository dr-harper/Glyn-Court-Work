\documentclass[11pt]{book}
\usepackage[utf8]{inputenc}
\usepackage[bmargin=0.65in, headsep = 0.5cm, footskip = 0.75cm, tmargin = 0.75in, bindingoffset=0.25in, outer = 0.5in, paperwidth=6.125 in, paperheight= 9.250in]{geometry} 

% Packages
% -------------------------
\usepackage{adforn} % For adding flourish
\usepackage{graphicx} % for figures
\usepackage{enumitem} % Allows lists to have no item space
\usepackage[font={small}]{caption} % Allows captions to be altered
\usepackage{tabularx,array} % For tables. Used to format lists
\pdfmapfile{+OrnementsADF.map}
\usepackage[utf8]{inputenc}
\usepackage[T1]{fontenc}
\usepackage{lmodern}
\usepackage{pgfornament}  % For creating flourishes
\usepackage{emptypage}  % pre­vents page num­bers and head­ings from ap­pear­ing on empty pages.
\usepackage{parskip}


% Custom Chapter Heading
% --------------------------
\usepackage{xcolor}
\usepackage[explicit]{titlesec}%
\newlength\rulew
\titleformat{\chapter}[block]
{\filleft\sffamily\bfseries}
{\fontsize{60}{20}\selectfont\textcolor{gray!80!}{\thechapter}\hskip8mm\titlerule[0.8pt]}
{0em}
{\fontsize{26}{26}\bfseries\llap{\parbox[b]{0.85\linewidth}{\filleft#1\vskip0.6ex}}}%
\titlespacing*{\chapter}{0pt}{2pt}{5ex}

% Custom Heading Style
% -------------------------
\usepackage{fancyhdr}
\pagestyle{fancy}
\renewcommand{\headrulewidth}{0pt} % Remove header line

% Custom Commands
% -------------------------
\newcommand{\Chapter}[2]{\chapter[#1]{#1\\ \vspace{0.0cm} \textcolor{gray!80!}{\small#2}}}  % Add subtitle to chapter

%Add Flourish to separate text
\newcommand{\Flourish}{
\vspace{0mm}
\begin{center}
\pgfornament[width = 4cm]{82}
\vspace{0mm}
\end{center}} 

\newcommand*{\plogo}{\fbox{\textbf{MH}}} % Generic publisher logo
\newcommand{\quotemark}[1]{``{#1}''}

% Styles and Formatting
% ------------------------

% Change font for all document
{\fontfamily{cmss}\selectfont
  \renewcommand{\encodingdefault}{T1}
  \renewcommand{\rmdefault}{cmss}

% Prevent Widows and Orphan Lines
\widowpenalty10000
\clubpenalty10000
\setlength{\parindent}{0ex} % Removes paragraph indent
\setcounter{tocdepth}{0} % Set Number of Levels to TOC
\setlength{\parskip}{0.5em} % Change Paragraph spacing


\title{The Brendon Enterprise}
\author{Glyn Court}
\date{}

\newcommand*{\titleBC}{\begingroup % Create the command for including the title page in the document
\centering % Center all text
\huge{The Brendon Enterprise} % Title

\vfill % Whitespace between the title and the author name
\Large{Glyn Court} % Author name

\vfill % Whitespace between the author name and the publisher logo
\normalsize
\plogo\\[0.5\baselineskip]
2018 \\% Year published
\centering
Edited \& Published by Michael Harper \\

\endgroup}


\begin{document}

\frontmatter

\thispagestyle{empty}  % Removes page numbers
\titleBC 

\tableofcontents

\mainmatter	



\chapter{The Beginnings}

Exmoor, Dartmoor, Quantocks, Chilterns, Cheviots, Cotswolds, most people have at least heard of them; but how many could point on a map of England to the Brendon Hills?

But climb aboard your Cessna or whatever in Cornwall, take off and set your course north-east and altitude 1350 feet, and only once in your flight of 800 miles, wind, weather and air traffic controllers permitting, will you have to climb to clear a hill before touching down in Norway.

The obstacle is not one of the well-known ranges of southern England.  At 1300 feet you would soar over them with hardly one protesting bleat from the denizens down below. Higher than all these  hills, with summits higher indeed than any point in England east of a line from Exmoor to Flamborough Head in Yorkshire, are the Brendon Hills of Somerset.

On a relief map the hills, or Brendon Hill in the singular, as known locally, show as an eastern extension of Exmoor, and part of them has been included in the Exmoor National Park.   They run roughly parallel with the coast of the Bristol Channel at four to five miles' distance, from Wheddon Cross in the west to Elworthy in the east. Their western boundary is clearly defined  by two valleys,  those of the Avill, which flows north by Timberscombe and Dunster and out to the sea,  and the Quarme, which merges with the Exe four miles south of Wheddon Cross.  Between the valleys is a 1000-foot watershed or col over which winds the main road from Minehead into mid-Devon. For travellers coming from the west, Brendon Hill begins at this level, but the road takes them up steeply for a mile to the top, where the plateau suddenly appears, stretching for mile upon mile.

From this point, Fairway Cross, the road follows an ancient track known as the ‘harepath’ or ‘herepath’, the “army road, the king’s highway” of Saxon days and long before that. It runs due east with only such deviations as demanded  by age-old boundaries or troublesome contours - for Brendon Hill, although a plateau, is intersected by the headwaters of numerous rivulets and streams. After six or seven miles of this hilltop career, nearly always between beech hedges, one suddenly comes, on the left,  to a panorama of the Bristol Channel and the Welsh mountains, twenty miles away; and at the eastern end of the ridge, a glorious prospect taking in the central plain of Somerset almost to the Wiltshire and Dorset heights. Then the hillside and the road fall away nearly a thousand feet, even more steeply than they climbed in the west.

In all this distance (12 miles) you will pass close to only four cottages (one of them a former toll house), a Methodist chapel and an inn, for the farms all lie away from this hilltop  road,  down long lanes, sheltering in hollows from the drenching  mists and rain-laden wind.

Seen in section, Brendon Hill would present the shape of a wedge pointing south - or perhaps a "veiny" cheese. The north face is a steep escarpment, falling abruptly from 1200 to 300 feet and often clothed with woodland. The plateau tilts upward for a short distance and then slopes gently southward for five or six miles and shades away into the valleys of the rivers Batherm and Tone. Altogether the hills take in about 80 square miles, and the 2500? people who live there, distributed at  30  per square mile, have purer water and air than anywhere else in southern England, with  room and to spare. 

Long ago  the hills may have partaken of Exmoor's nature, but now they are very different - in contour, in vegetation, in economy, for having been mostly reclaimed for farming in the late eighteenth century, they are  laid down to pasture in fairly small fields protected by beechen hedges.

They have lost Exmoor’s wildness and her expanses of bracken and heather or ling; and yet one can still find secluded dells and marshy valleys with stone outcrops where the sense of solitude and unfathomable age are as powerful as anywhere on the moor itself; and in remote corners, where surface water gathers in pools or zogs, one find a sense of mystery, of the invisible presence of some ancient entity left stranded by the onward march of time. You cannot be directed to these places, you will come upon them by chance or not at all. Here and there, too, are memorials from the distant past, barrows or burrows, standing stones and “castles”- Iron Age forts or places of refuge – and, altogether more unexpectedly, ruined buildings of more recent date and unmistakeably the relics of mining activity. How these came to be, and this industry brought prosperity and new blood into Somerset, and then how it faded and the hills returned to their green tranquillity, forms the subject of this book.	             

The geology of Brendon Hill ‘is basically simple, with three broad bands, running west to east, of limestone, slate and sandstone, but complicated by lodes of iron ore.’ All have served the local population.   Slate was quarried near Wiveliscombe and at Treborough - the latter for eight centuries and at its peak employing more than 30 men, and lime kilns, each one the responsibility of a single lime-burner,  were dotted about  wherever the limestone came near the surface. The soil overlying these rocks is generally loam, though within this there are local variations.  But these two industries probably employed fewer than a hundred men all in all. It was with iron ore that   Brendon Hill in the nineteenth century was belatedly caught up in the Industrial Revolution.                                  

The origins of mining on Exmoor and Brendon Hill lie too far in the past to be clearly made out. Popular lore gave the credit to the Romans, but for the untutored “old folk” of the hill country the Romans and the Roman Catholics were one and the same, 1 and archaeological opinion long discounted this origin. Yet as so often, improbable legend has been found to have a base in undeniable fact, and excavations near Brayford and Simonsbath from 2002 onward have unearthed deposits of hundreds of tons of slag from the second and third centuries A D, the height of Roman power.. The quantity of iron produced was far in excess of local needs and may have been exported all over the Empire.  (Later excavations have pushed the probable date back even earlier, to 1000 B C.)

The length of time that the industry continued at Brayford has not yet been estimated, nor the full extent of the working; but whether long or short, most probably it eventually ran itself to death.  The mounds of slag are unmistakable evidence of smelting, and smelting, in those distant times, demanded charcoal and devoured it at an extravagant rate. To yield these (how –many) hundred tons of slag the furnaces must have quite rapidly consumed all the charcoal that could be produced in the locality, say within a radius of five to ten miles; and to acquire supplies from further afield cannot have been either practicable or economic; (nor could iron ore have been sent for smelting elsewhere along the primitive tracks); and so these Brito-Romans may have destroyed their immediate environment and thus put themselves out of business.

German miners worked in England in the reign of Queen Elizabeth, and the name Eisen Hill (Eisen is the German for iron) has been taken as evidence of their presence here. Well, perhaps.  

Evidence of (19th century) mining in Beggearnhuish was fund in the form of black scoria in Baggearnhuish Farm in 2012.

Since then, iron ore has been mined in scattered locations on Exmoor and in the coastal vales, such as at Burrowhayes and Wychanger in Luccombe, and at irregular and infrequent intervals, but generally on a small scale and without the benefit of public investment.  One mine was Wootton Courtney, another at Colton, in the uplands of Nettlecombe parish. The memory of these enterprises often persisted long after the workings had ceased, and during the Crimean War   a soldier from the Huish Champflower area wrote home that he and his regiment (most probably at Inkerman) had “attacked up a hill as steep as at Colton pits.”

The modern enterprise began in earnest when, in 1839, Sir John Lethbridge, of Chargot, Luxborough, had three trial pits dug in or near ancient workings  at Chargot Wood, Goosemoor and Withiel Hill, to search for worthwhile iron ore. Work in the  Withiel pit, renamed the Lothbrook 2 mine, revealed a promising vein of ore, but the iron content was  less than 50 per cent, too low to justify the high cost of transport for smelting. Sir John installed and advertised “a WATER-POWER ENGINE (blast), with Hammer, Wheel and furnace, Dwelling house for Workmen, etc” and the mine went on yielding good ore, both haematite and spathic,  but in fairly small quantities for the next ten years. (Haematite is a valuable iron ore, deep red or reddish black, occurring in dense masses, sometimes with a with a shiny, nodulated surface; spathic ore, as a spar, crystalline, easily cleavable, and non-lustrous, was inferior to haematite in that in gave off carbon dioxide in smelting and thus deadened the furnace). By 1849 the mine had a shaft as well as the original levels, and a notice in the press recorded that “a beautiful iron wheel” had just been installed, “which lifts the water at the mine, and at the same time brings up the ore.”

The infant business needed just that gentle touch of luck, the little push of a half-open window of opportunity, and it came when, at the Great Exhibition of 1851, Samuel Blackwell, an ironmaster in Dudley, put on show a collection of English ores, two of which came from Brendon Hill, and very likely from the Lothbrook mine. Apparently he visited the mine shortly afterward, found that the haematite was not being worked and decided that it ought to be. Unfortunately he lacked the capital, but he took the matter up with his brother-in-law, Ebenezer Rogers of Abercarne. Rogers acquired both the Lothbrook and Gupworthy mines and rights to mining leases over both the Lethbridge and the neighbouring Trevelyan estates, and began new workings at Raleigh’s Cross, some four (?) miles east.

Rogers went into partnership with a Thomas Brown, who soon bought him out and formed the Brendon Hills Iron Ore Company with one of his partners in Ebbw Vale, Joseph Robinson. In June 1853 the Lethbridge Estate granted them a 60-year lease at royalties of one shilling a ton for haematite and eightpence a ton for spathic ore. (Haematite, the name deriving from the Greek word for blood, is a valuable iron ore, Fe2 O3, often blood-red, with red streaks;  ‘spathic’, which sounds equally Hellenic or Spartan by resembling the botanical term ‘spathe’,   merely derives from ‘spar’, as in ‘felspar’ or ‘fluorspar’, and denotes an inferior ore though with some iron content; it is also known as chalybite). The Trevelyan lease did not follow straight away, but the partners had mining rights for virtually the entire length of the hills. They set up offices in Esplanade House, Watchet, and started serious mining, extending the workings at Gupworthy and Lothbrook, deepening the workings at Raleigh’s Cross and opening new mines at Eisen Hill and Bearland Wood, so that in 1855 they raised more than 4,000 tons.

Success, however, brought its problems. Ore could not be economically smelted without coal, of which there was no source nearer than South Wales, and it would have had to be shipped over to Watchet and thence carted, a ton at a time, six miles inland  and  twelve hundred feet of altitude up the steep, rough and narrow roads to the top of Brendon Hill. That would have been rank foolishness. Common sense rejected the idea and dictated that the ore make the trip in reverse, via Watchet: but here was a problem.

It was not the distance, twelve miles from the furthest mine to the port, though the return journey took a day for a horse and cart; and hiring a cart and driver for two shillings a day virtually trebled the cost of the ore. Besides this, a cart going down with a full load would have to return empty unless it brought back lime for top dressing, and this would incur toll charges greater than the value of the load. Persist in this and the mines could never be made to pay.

In 1854 they were bringing down an estimated 15 tons of ore a day, and the bridge over the stream by the ford in Roadwater, as shown by the tithe map, was only one cart wide and unfit for heavy traffic, and the ford itself and the houses nearby were subject to flash floods. It seems certain that the Ebbw Vale Company,   even before the railway was planned, undertook roofing  over the river in the higher part of Roadwater to make what is now universally known as The Bridge. No other competent authority would have had any interest in the work, and it proved to be a great gain for Roadwater.

There is a story recorded by William Joyce in his Echoes of Exmoor  that when the “Iron Lord” - whether Abraham Darby or Sir John Lethbridge was not made clear -  met with one of these carts, drawn by an old horse called Blossom and driven by a young boy, labouring up one of the hills, he was moved with sympathy and decided then and there that some less toilsome way must be found.3  It is much more likely that common sense told him that a railway would solve the problem, and even though steam locomotives might jib at the gradients of the Brendon Hill roads, Victorian engineers were not to be beaten by apparent impossibilities; and so, when the chief initiators of the scheme had undertaken to fund it, a bill authorising The West Somerset Mineral Railway went through Parliament and received Royal Assent on 16th July 1855.

The line, 13 miles long, would initially run the four miles from Watchet harbour to the higher part of Roadwater, where a railhead with a coal yard and siding would receive the ore brought by horse and cart from Brendon Hill. The gauge would be the standard 4’ 8 1/2” (14350 mm for the continentally-minded) and other essential engineering works would be:

\begin{itemize}
\item a jetty and harbour wall, station and engine shed at Watchet., 
\item station platforms at Washford, Roadwater, Comberow, Brendon Hill  and, eventually, Luxborough Road, 
\item girder bridges over the Brendon stream at Washford, Lower Roadwater and Roadwater station, 
\item stone bridges over public highways at Washford, Pitt Mill, Comberow, Brendon Hill (Incline top), Naked Boy, Withiel Road and Gupworthy, and under the foot of the Incline; 
\item and over the stream south of  Roadwater station, 
\item a footbridge north of Torre,  
\item manned level crossings at Lower Washford, Torre, Lower Roadwater and  Proud Street; , a siding and temporary engine shed at Torre;
\item embankments between Bye Farm and Lower Washford, 
\item deep cuttings at Torre,  (creating a red sandstone quarry from which red sandstone could  be drawn for railway buildings), and below and above Pit Mill;   
\item low embankments and diversion of the stream in Roadwater.
\end{itemize}

More ambitious than all these, and more expensive, was the Incline (of which more will be said).  The other major work, though its history is  nowadays little appreciated, was the combined Ramp and stone bridge at Washford, diverting the coach road at a point where the operation of a  level crossing would have almost inevitably caused accidents and, at the very least, continual altercation between competing users. 

The Ramp, built to take a few dozen coaches and carts, now carries, as the A 39, 12,000 vehicles of all sizes daily and a few courageous citizens who trust their lives to the sidewalks provided by the County Surveyor; and a measure of the solidity of the work carried out 150 years ago is the fact that despite the pounding of the 40-tonners day after day, it still holds firm. (2340 words)  The old road can still be seen in the meadow to the south, running parallel with the Ramp, crossing the stream by a  narrow bridge and apparently ending at the foot of a three-storeyed cottage; but in its working life it then climbed steeply to the height of the present road just before the former Methodist chapel  - a fact vouched for by an old lady born in a cottage (now demolished) below the present road level, who remembered that as a child she looked down on the coaches passing along.

Apart from this structure, the building of the line in the valley did not set the engineers many problems. The steady gradient of 1 in 100 demanded embankments between Watchet and Washford and above Roadwater, and cuttings had to be blasted below and above Pitt Mill. (See chapter on Walking the Line).(BIT MORE ABOUT THE GRADIENT)

The capital needed, £50,000, with power to borrow another £15,000), was contributed by  Thomas Brown (£10,000) and Joseph Robinson (£6,000), directors of the newly formed Brendon Hill Iron Ore Company, Abraham Darby and William Tothill, both of the Ebbw Vale ironworks (£10,000 each) and Frederick Levick  of the Pontypool Iron Works (£1,500), (£10,000).

The first sod was cut on 29 May 1856 at Roughmoor, between Washford and Roadwater, and the gangs began to work in both directions. Progress was swift, and within two months the Company purchased a batch of rails for £461 13s 1d (how precise these Victorians were! Pennies had real value).  The rails were    now laid between Washford and Roadwater, and as soon as that length of the line was judged at all fit for traffic, in November 1856 the Company acquired its first locomotive to assist with building the remainder.

The 0-4-0 tank engine from Nielson \& Co of Glasgow, costing £1310, weighed 13 tons and was built with an eye to dependability for hard work rather than elegance. It came by rail to Taunton and was brought thence by road to Roadwater – a taxing and dangerous task for every man and   beast engaged in it, and one horse was killed. The engine was then housed in a temporary shed at Torre, but far from lending a helping hand in the work it needed first-aid treatment itself. One morning in the following January the fireman was ill and his unofficial stoker, a lad of twelve (?), lit  the fire and created a roaring blaze without checking the level of water in the boiler. The directors ordered a replacement and the company repaired the damaged engine, but the unlooked-for extra £1300 expenditure was perhaps a signal of the eventual cost of building the line, which was more than double the estimate. Despite the mishap, the line was ready for ore and goods traffic as far as Roadwater by April 1857. This was the railhead and the yard of the Watchet Trading Company, with Henry Court as agent, to which ore would be brought down by cart until the upper line was complete: and this would take a good deal longer,

Even though the railway was initially for goods traffic only and offered no obvious benefit to the villages it ran through, the arrival of the first train generated great excitement in Roadwater, for most of the people had never seen a locomotive nor, probably, even a picture of one.  William Court, seventy years later, recorded his memories of that first day: 

‘As a boy of eight years I remember well the first commotion of navvies and masons levelling up the railway and building the wall across the meadow and also the station. Roadwater was a lively place then, every house full of lodgers – navvies and miners – as there were no houses on Brendon Hill.

‘I well remember, too, the day when the first railway locomotive this side of Taunton came puffing and snorting up the valley, until we could see its smoking funnel peeping up toward the station. Soon several of us boys were down the line to inspect this wonderful iron horse.  Not one of us had been to Taunton and that was the nearest point where an engine could be seen. The Taunton to Watchet line was not cut until 1862, so we had a railway and station here seven years before Williton or Stogumber had either, (and nearly 20 years before Minehead). The engine driver was a Scotsman, a very fine tall man, Alexander Mossman, who in after years settled down in Watchet and built the West Somerset Hotel.’ One young fellow, though, was not so trusting. Scared almost out of his wits by the “snorting, smoke-spitting monster, he belted off home, hollering that the Devil was come to Roadwater”.

Now, with two engines, the ore could be shifted quickly enough to avoid a build-up on the hill. In the previous year, the first of the company’s operations, the mines had yielded 4,940 tons, or a little over 16 tons per working day, and the horses and carts with their carter hired from Brendon Hill farmers could probably just about cope with this amount . But in 1856 the yield   rose to 25 tons per day, and in 1857 to 32 tons, and clearly the railway had reached Roadwater not a day too soon, and the engines and their crew were straightway hard at work.

Yet only three months more, and another accident, this time much more serious, cost lives and hindered the work of haulage.

On Saturday 27 August, about 4 p.m. – that is, the end of the working week – an engine left Roadwater with a truck containing about 30 labourers returning to Watchet for their pay. When they reached Washford, the crossing keeper, Henry Giles, flagged the train and warned the driver to stop, as an up train with coal was expected any time from Watchet.  The driver should then have taken his engine and truck into the siding provided for that purpose, but John James, the assistant engineer, decided to risk it as he had some letters to post, and so they forged on at about 20 m.p.h., whistling loudly. Mjnutes later their luck ran out, for on rounding a curve just below Kentsford they suddenly saw the coal train only 200 yards away. The engines smashed together head-on. Giles was killed instantly, James and another man on the engine died later of their injuries, and several of the workmen were injured or scalded by escaping steam.

The coroner’s inquest took three days, and the jury convicted James – post mortem – of manslaughter.

The loss of both engines threatened to hamper the work considerably,  but the directors ordered a replacement from Sharpe Stewart of Manchester, and the next month (September) brought, probably by sea, a new and more powerful 0 – 4 – 0  saddle tank type engine, named Rowcliffe  after the Company’s legal advisors in Stogumber.  This, with one of the damaged engines restored, assured the ore transport for the next ten years until a passenger service was added in 1865. (Neither of the Neilson engines seem to have been allowed the dignity of a name, not even Box or Cox). 

Meanwhile work on the line south of Roadwater continued, calling for engineering works such as the diversion of the stream, near Oatway House, the building of a bridge over the Leighland – Sticklepath road at Pitt, and blasting cuttings below Pitt and above. The line reached Comberow by December 1857, but further progress on the Incline to Brendon Hill was slowed by the need to blast much rock and then  achieve a faultless 1 in 4 gradient. The Company, however, could not afford to wait for perfection, and five months later, on 3 May 1858, the Incline was declared open – perhaps over-optimistically. The traction principle was simplicity itself: a loaded truck at the top was attached by a cable, which went round a revolving drum, to an empty truck at the bottom, and gravity provided the power.

Simplicity, however, did not rule out a complication in practice, for as the truck descended its tractive power would increase with the weight of the cable while the load of the empty truck and cable would decrease equally. To control the speed of descent a braking system would have to be in constant use, and the power for this would come from a stationary steam engine coupled to the drum. But again a problem arose.

The drums of the winding gear, made in South Wales, were a massive, imposing wheel no less than 18 feet in diameter, but no captain could be found willing to risk his vessel transporting them across the unpredictable seas and shifting sands of the Bristol Channel. But the miners could not wait till this problem was resolved. A stationary steam engine used in building the line was pressed into service and seems to have functioned well enough until the regular iron drums were installed, well over a year later. At any rate, they had the Incline, a stupendous work of engineering and a key to success for the mines - but it came at a price far higher than anyone had anticipated. Unlike many other engineering feats, it did not exact a price in human life, but it cost all of £50,000, which 150 years on would amount to at least £5 million.  According to my father, the “old folk” would say, “You could lay golden sovereigns touching all the way down for what it cost to build.” A colourful exaggeration? Not a bit. If a golden sovereign measures three-quarters of an inch across, 50,000 of them would stretch 1,150 yards and the 1,150 yards would cost £54,400.

Sellick 99: “Communication (between Comberow and the top of the Incline) was by means of two primitive slotted semaphore signals with white arms and counterweights at the head of the post, each controlled from the opposite end by wires carried on posts some 8 ft. high beside the line. The Comberow stationmaster would first signal that he was ready by raising the semaphore at the top of the Incline, and the brakesman would acknowledge this by raising the Comberow semaphore. He would then release the brakes on the winding drum and engage the engine by means of a clutch controlled by a foot-pad in his hut, allowing the wagons to move. As a guide to the engineman the position of the trucks was indicated by a pair of lead-weighted chains in the winding house.”

This first year of operating was, however, marked by one incident which might have turned out disastrously but for the quick-wittedness of one boy, the William Court mentioned above. At the age of twelve he was learning the craft of a shoemaker with his father, a master of the trade and the company’s agent. He worked either in a shed in the coal yard or in the first floor of a cottage overlooking the level crossing, for which he was also responsible, pushing or pulling the heavy beam which served as a barrier into position when needed – and as the road carried served only one farm, Wood Advent, it carried so little traffic that the normal position of the beam was to bar the road, not the railway. One day, as William sat tapping away at a hobnailed boot, he heard an unfamiliar clattering sound. Looking up,  was shocked to see, only a couple of hundred yards up the line, an iron ore truck which must have broken free part way up the Incline and was now rushing down toward him out of control. Without a moment’s hesitation, William ran out to the bar, pushed it across and made it fast to the opposite post, in the hope that it would at least slow or divert the truck. Then, instead of running away from the truck he ran towards it, telling himself, to face the danger that way lies safety. ‘Turn tail, and the truck or the load or the broken beam may knock me down’.  He was perfectly right. He sidestepped the truck, which crashed into the beam and was then tumbled into the river, so that no one was hurt.

By this time (1859) the mine at Gupworthy was proving valuable and yielding so much ore that it could not be carted away and it piled up at the mine head for several years. The directors decided that the line must be extended from the top of the Incline across the main road by a stone bridge and then continue west for the 4 ?) miles to Gupworthy and even beyond, to the Kingsbrompton road known as Armoor Lane. A local quarry owner and contractor, John Gunn of Maundown, Wiveliscombe, undertook the work and started about April   1863. Difficulty in rock cutting and shortage of labour made for slow progress, even when the men also worked by moonlight, and the line did not reach Withiel bridge until March 1854. In June the earthworks, if not the rails, reached Gupworthy, and three months later the extension line to the Kingsbrompton road was pronounced “sufficiently completed for mineral traffic, and is being used for the conveyance of iron ore as well as for the carriage of coal, lime and other goods”. (The “box” remained in daily service until the mines closed down – but more of that in due course).

The goods mentioned above, with mining equipment and supplies for the shops and stores which opened on the hill made up the  main freight business for twenty-five years. Nothing sensational happened, no blasting charges exploded, but two events took place to show that the line was serving a wider community that simply the miners.  In 1858 the simple Anglican chapel of ease, with its west gallery for orchestra,  which had served the people of the hamlet and nearby farms since the thirteenth century, was condemned for demolition, and in 1862 it was  replaced by a larger red sandstone church in Victorian Gothic. The stone was quarried at Lodge, near Bilbrook, brought by cart to Roadwater, brought thence by train to Pitt Mill, and then carted up the steep, narrow lane to Leighland.

\chapter{Buildings}

Bridge at top of Incline, and station and House: iron plate attached to wall,date 1865. 
   (From Book of Roadwater): The Ebbw Vale Company had not meant at first to provide  public transport; but in so doing they won more good will than they could have expected, and even after the mines closed down in 1883 the terms of their charter obliged them to go on operating the train for another  15 years.”

Such a valuable accession as the railway to the life of West Somerset could not, in fact, be allowed to remain in this rudimentary state. Social life had its demands as well as industry, and the Company, which now had the running of the line (CHECK), acknowledged this when on 7 August 1860 they ran an excursion train, carrying between 700 and 800 passengers in “clean trucks”  to Comberow for a temperance rally on Brendon Hill. The more venturesome, if less vigorous, were then hauled up the newly-completed Incline in open trucks, while the more cautious made their toilsome way up on foot, in panting anticipation of the 200 gallons of the cup that cheered.

Having sampled the pleasure of easy transport, they took to it, and such excursions became annual events, looked forward to keenly and hugely enjoyed, especially after the official opening of the line to passenger traffic had made them regularly practicable.

Regular trains were already being run for the miners, and sadly one man was killed while trying to board a workmen’s train as it steamed out of Roadwater. But Watchet station had been built in 1858 with passenger accommodation included, and it was now only a matter of time before lesser mortals were served as well.  

On May 2nd 1863 the West Somerset Free Press confidently announced that “the mineral railway ... would soon be open for passenger traffic. Visitors would be able to take a trip into a scenic part of West Somerset, especially the incline to the top of the hill, ‘unparalleled in England and Wales.’” (CHECK WORDING) and in September 1863 the Directors reported that “passeneger stations (had) been built at Washford, Roadwater and Timwood (Comberow), a large and convenient Goods and Carriage Shed constructed at Watchet and the necessary Signals for Passenger Traffic erected.”  These stations had been “built with Ladies Waiting Rooms and Water Closets and platforms 120 feet long”, and ”a new siding (had) been made at the foot of the Incline for the Engine to take in water and the Passenger carriages to pass her by.”  In March 1864 the Directors gave a year’s notice to the Board of Trade that they intended to start a passenger service on May Day 1865.  An inspector, Captain H.W. Taylor – presumably of the Royal Engineers – came and inspected the line and required a considerable amount of work to be done before it could pass muster:  to strengthen the wooden trenails with iron spikes, replace all missing fishplates, widen the platforms, increase the clearance at level crossings, install catch points at the foot of the Incline and strengthen the Washford road bridge.  This work was done in the following winter and a second inspector gave the all clear, though with a caution that because some of the rails were badly worn and part of the ballast was pebbles, trains should be worked at a moderate speed, a recommendation that the framers of the time-tables took to heart, no doubt to the disappointment of the young rips who had hoped to career down to the heady delights of Watchet at a mile a minute or near enough.

This longed-for event took place five years after the first excursion, on Monday September 4, 1865 – and oh! for a Time Machine to transport one back to that golden day when the dream came true Everyone knew that it meant a great deal to the neighbourhood  -- progress, social communication, a modest increase in prosperity – and they turned out to celebrate with gusto.

Make no mistake about it. The saying that “Her Majesty is not amused” is a flagrant misjudgment of Queen Victoria and a slander on her age. The Victorians had a tremendous capacity for enjoyment. They threw themselves into the fun of the thing, whatever the event or celebration. They looked forward to their pleasures and put effort and style into making the most of them.

To the best of my knowledge no photos exist of these celebrations, though it is hard to believe that James Date was not recording them. The popular photographs of the re-opening of the line in 1907 will give an idea, but probably 1865 was more zestful and boisterous

The first Monday in September 1865 was a beautiful day for the formal opening of the Mineral Line to passengers The first indications of the celebration were the reports of cannon, fired, apparently, by Brendon Hill miners who started to charge rather early in the morning and kept it up  throughout the day.

Three to-and-fro trips from Watchet to Comberow were advertised, and the 11 o’clock from Watchet,   hauled by Rowcliffe,  left Watchet “amid the cheers of  a large crowd and the deafening report  of cannon. At Washford an excited crowd thronged the platform, and at Roadwater the train was greeted by a tremendous ovation, one enthusiast striking up “Cheer, Boys, Cheer” on a musical instrument. “The station was decorated with admirable taste” – what else would one expect from that part of the world!  – “and the flower gardens of the neighbourhood must have had their choicest products appropriated  to a very large extent.” Garlands and bouquets were suspended at every conspicuous place  and where flowers could not be used large flags and banners served the purpose” Passengers crowded on board, and as the Rowcliffe” puffed out of the station the nose of cannon and deafening cheers sped her on her way.” (But where did all these cannon come from, by the way? Or were the men setting off blasting charges in iron pipes?)

The station at Comberow was also decorated. From there upward the more venturesome passengers travelled up the Incline in open trucks with boards laid across for seats, to pass at the top under “a double arch of graceful proportions, gaily decked with mosses, evergreens, ferns  and flowers, and bearing upon its front the device  ‘Success to the West Somerset Mineral Railway’ in letters of moss. From its summit three flags floated.  There were also decorated arches over the line trimmed with evergreens and flowers bearing appropriate mottoes.” Simple, harmless, unsophisticated pleasures, and, as so often, the best.

And that was not the end of the celebration. In the afternoon  Watchet shut up shop  en masse and caught the 3.30 train to Comberow along with the Fife and Drum Band, and there they picnicked, revelled and generally enjoyed themselves until a special twilight express  came at eight o’clock to carry them home.

 It is difficult to express the degree of pride generated in the valley by the coming of the Mineral Line. Those who knew the line in its heyday have been gone these forty years, but I have a vivid recollection of one elderly Roadwater lady claiming,  “We had the railway long before Minehead, “ and the presence  of the line made the people of the outlying hamlets such as Hayne, Leighland and Chidgley more aware of the busy, hustling world outside.
 
The passenger service began with three carriages hauled by the Rowcliffe, newly overhauled and painted in honour of the occasion, but after three months a new Sharp Stewart locomotive was added, the Pontypool, a 0 – 6 – 0 like the Rowcliffe, but more elegant and considerably more powerful. Her elegance, though, did not ensure her a serene and respectful reception in West Somerset. She was brought to by rail on the Bristol \& Exeter line to Watchet, and the engineer reported:

\begin{quote}
“The new locomotive arrived on the 28th February and on the 1st March we began to get her off the carriage on to the street by the station. There was a little damage done to her while in charge of the man who had accompanied her down, the house over the driver was slightly disfigured and some of the small work over the boiler  injured; one of the springs got broken as well.  She appears to be a stiff and well-made engine” – and he might have added, “and worthy of better care”.
\end{quote}

Evidently the engine had tipped, if not overturned, and whatever the rights and wrongs, it was ever after firmly believed that the gang in charge of the transfer had lubricated themselves so liberally that an accident was bound to happen, and it did. (The name Pontypool at first seemed incongruous on a Somerset railway, and in fact the Company had bought a pair of these 0 – 6 – 0 locomotives. The other one, Brendon had been delivered first to Ebbw Vale and set to work in the valley there, and Pontypool  was dispatched to foreign parts. Why no one ever changed over the name plates is something of a mystery).

With Pontypool at work, the passenger service was soon increased to four trains a day, but speed was not of the essence. Hustle was a mode of life the country people had no use for, and nor did the railwaymen, as the time table will show:

Leaving	1, 2, 3	                1, 2	 1, 2	1, 2, 3
Watchet	  7.30		  11.15	  3.30	  6.15
Washford	  7.40		  11.23	  3.40	  6.25
Roadwater	  7.52		  11.32	   3.52	  6.37
Combe Row	  8.10                   11.42	   4.10	  6.55

CombeRow	  8. 30		  11.50	   5.20	  7.00
Roadwater	  8.45		  11.59	   5.35	  7.12
Washford	  9.00		  12.07	   5.50	   7.22
Watchet	  9.10		  12.16	   6.00	   7.30
 Average speed (including stops) – 14 m.p.h.                  )
 
In 1872 the trains were increased to five daily at two- to three-hour intervals, the first leaving Watchet at 7.30 a.m. and the last leaving Comberow at 6.50 p.m, so that the train would  get to Watchet in good time for whatever delights the town had to offer, but Shanks’s pont had to be mounted for coming home again. All the trains took half an hour for the one-way trip, except the first in the morning from Watchet and the two in the evening from Comberow, which took 40 minutes, presumably to give time for picking up dropping off parcels and picking up workmen in Roadwater, and the afternoon train, which dawdled its way through three quarters of an hour. 

When the mines closed for a year in June 1879 the Company had a statutory duty to keep a passenger service running, and it continued, but it was reduced to two trains each way, from Watchet at 9.15 a.m. and 3.00 p.m., and from Comberow at 10.45 and 4.00. A short revival of mining restored more or less the old level of passenger service, but the enterprise was flagging out, and when the mines closed definitively in June 1883 two trains a day were all that remained of the service.

Fares were the same for every stage, Watchet – Washford, Washford- Roadwater and so on ( First class – 4d, Second class 3d, Third class 2d, and Comberow was the end of the line for the fee-paying passengers.  They were then faced with the Incline, but these Brendon folk had heads for heights and allowed themselves to be hauled up in the waiting ore truck, across which boards had been laid for their comfort. Reaching the top, the Gupworthy contingent could, in the Company’s good time, travel on, again at their own risk, in the returning ore trucks drawn by the antique but still workworthy “box”; and if the time-table for the valley line discouraged reckless speeding,  the men of the hill line had no need of such restraint. The engine wheezed, the ore trucks clanked, but they stayed together and got to Gupworthy in the end, but more important duties than pampering the  passengers had to be seen to on the way. One mile (?)  west of the Incline top the line reached it highest point, 1300 ft above sea level, probably higher than any other steam railway in England. “The Neilson engines were lubricated while descending this section, and at Burrow farm the fireman would clamber on the buffer beam to pour tallow from a kettle on to the motion. If the rails were greasy on the return journey he had to run ahead and sprinkle sand on the rails to prevent slipping,” for there was no sanding apparatus on the engine.


Other duties cropped up from time to time, unpredictable but right and proper for any reasonable person:  stopping to pick mushrooms in season, chivying a bullock off the line, or helping a farmer round up a straying flock of sheep.

Sometimes, too, sheer cussedness – or perhaps love of a joke – may have played a part. Two engineers from Watchet once went up to do some work at Goosemoor, which is even beyond Gupworthy. Arriving at the top of the Incline, they boarded the train and proceeded on their journey. Before reaching their destination, however, the train was stopped and the driver went across the fields to a farmhouse, and he was away for some time.

  “Where have ‘ee been to?” they asked him. “What have ‘ee been doin’?”
  “Not much,” he said. “I been to the farm to ring some pigs.”
      Happy days, when speed and urgency were rarely called for outside an operating theatre! 
      
\chapter{People}    

At mid-century, as shown by the 1851 Census, the population of the rural parishes that the mineral line would serve – and excluding the town  of Watchet – was some 3,000 (??), namely : Old Cleeve  (1550), Nettlecombe (353), Treborough ( 142  ), Clatworthy (part) (        ), Withiel Florey (104)  and Kingsbrompton (part) (30), and of those 3,000  all  but 20 ?? had been born in Somerset, and most of these in the immediate neighbourhood.  (The few exceptions were members of the professions, the Anglican clergy and the landed families at Chapel Cleeve and Nettlecombe Court). In 1861 the picture was very different. The population had risen to 3000: Old Cleeve 1,529; Nettlecombe (327; Treborough 183; Clatworthy (part)  30?; Withiel Florey  (164) , Kingsbrompton (929). 

In the early days the whole of the area on Brendon Hill covered by the mining activities was almost as free of “development” and humans as the wilds of Exmoor, for this was all pastoral or stock-rearing country, and only in the short hay harvest was there a need for the work force of arable lands. From west to east, the farms of Goosemoor, Gupworthy, Eastcott, Swansea, Withiel, Burrow, Brendon Hill, Beverton and Colton and the few cottages attached, held no more than 200 ?    people.   They had, though, very little room for “lodgers”; and “lodgers” there would have to be, for local men had no experience of mining and few were willing to work under ground for the few shillings more than their starvation wages on the land. For a few years the miners “lodged” in cottages down in the valley and their hosts benefited form the rent and the garden produce they could sell, but the mining company addressed the need, and by October 1863 four cottage had been built at Raleigh’s Cross mine, two  terraces of six cottages each, known as Church Cottages (for the nearby “tin church”) and Beulah Cottages (for the Bible Christian chapel), five dwellings known as Turf Huts (near the Incline). There were also wooden houses built in the goods yard to house the labourers working on the Gupworthy extension, and three larger two-storey houses known as Somerset Terrace at Sminneys; one of these latter was meant for the mines foreman. 

Building materials for the mine buildings and houses were at first stocked by the Watchet Trading Company in 1857 in their own warehouse by the railway yard at Brendon Hill, where they sold ‘coal, steel, salt, lime, North Wales slate’ – rather a backhand blow, for the slate quarry only two miles away – ‘bricks, draining pipes, etc’.

To cater for the daily needs of the mining families, John Vickery, of Luxborough, opened a general shop, and other tradesmen made the rounds regularly, including the butcher John Copp whose faulty weights cost him a fine in 1871.  A large stone and slate building, Brown’s Temperance House, stood near Railway Arch, and seven “superior cottages” were added at the rear. It later became a general shop run by. Even so, the miners’ families were not limited to this one source, and on Saturday evenings, with a week’s wage in their pockets, they would take the train down to Roadwater to buy provisions at the thriving Gooding’s Stores. Queues formed along the street, and in the shop assistants standing on the counter would pass laden baskets on walking sticks to customers waiting at the rear. Not only the tradesmen looked forward to these visits, for many of the miners were Welshmen or Cornishmen, and as a group of them approached the village or marched up from the station their choruses added  delightful harmony to the evening air.  

The 250 people of the mining community occupied some 60 dwellings on the hill, which, averaging a little over four per dwelling, seems roomy enough by the standards of the time, but averages are deceptive, for the better homes had two storeys with two rooms on each floor, but the poorer ones had only one bedroom, and that in the loft.  A court hearing in 1866 revealed that in one cottage, three miners, the wife of one of them and a child, all slept in the same room.     Working conditions underground were hard, as everywhere, and although the mines were free of the disasters of coal mining, having little dust and no fire damp, most years mere marked by one or more deaths due to blasting, a roof caving in, falls down shafts or suffocation by foul air. The Company does not seem to have been unduly negligent toward its workmen – indeed, it employed two surgeons to look after their health – but many years passed before they installed proper facilities for the miners to dry and change their clothes. Working hours were long – 7 a.m. to 4.0 p.m. or 9 p.m. to 6.0 a.m.  Monday    to Saturday – a 54-hour week – but this was far from unusual, and most farm workers would gladly have settled for a nine-hour working day.

\chapter{Accidents}

Those who had a mind for a social life did not let the working hours deter them, and for such an isolated area the opportunities were surprising plentiful. The Company ran the industry on the total abstinence principle, and the bibulously-minded had to amble along to the Raleigh’s Cross Inn.3 The landlord, Mr J. Beamish, and his wife seem to have run the place firmly and efficiently, also serving picnic teas in the summer, but fights were of course frequent, and in one between Somerset and Cornish miners in 1866 the landlord himself was assaulted, together with the policeman from Kingsbrompton who had come to restore order.

For the more serious-minded, the responsible family men, there was a reading room entitled The Miners Literary Institute at the Raleigh’s Cross mine, and the Bible Christians had a lending library at Gupworthy;  and on Sundays there were  the services and company of  three Nonconformist chapels and an Anglican one. Many of the Cornish miners were Bible Christians or Wesleyan Methodists, and the Bible Christians, who already had numerous congregations in the other Brendon Hill villages – Kingsbrompton, Cutcombe, Bury, Upton – opened a preaching room at Beverton farm in 1859. As numbers grew, with congregations of 20 to 30, sometimes more, they built a chapel, Cornish more than Somerset in appearance, at the junction of the Wheddon Cross and Bampton roads and called it Beulah (Happiness or Marriage). It cost £295 and was opened on Friday 31 May 1860, when a bazaar realized £30 and “several hundred people took tea”.  Over the next few years, under the benevolent eye of Morgan Morgans,  it became a centre for temperance rallies, encouraged and supported by Sir Walter Calverley Trevelyan, of the nearby Nettlecombe Court, President of the United Kingdom (Temperance) Alliance. Excursion trains filled with passenger ran from Watchet to Comberow, and other supporters came by rail from Gupworthy and places west. Morgan Morgans’s successor, however, viewed the chapel activities very differently, and at a meeting of the circuit (representatives of local 12 chapels) on Boxing Day 1867 heard a rueful report that “Brendon Hill, formerly one of the most important places in the Circuit, is almost ruined, at least for the present, through the bigotry and intolerance of the Captain of the mines.” The incident is puzzling, though, because the new Captain, Henry Skewis, was a Wesleyan and had on occasion preached in the Bible Christian chapel.   No doubt he would have wanted his own Wesleyan cause to prosper,  but can he have been sectarian to the point of  driving out miners who would not conform ?  One would hope not. Rather than traduce him unfairly, best note the mystery and move on.  But the Bible Christians carried on, with only slightly reduced number and outlived the mining industry itself.

At Goosemoor, at the western end of the mining area, the Bible Christians held meetings in  a farmhouse  or cottage almost from the outset (1855) until 1864, when they converted two cottages at Gupworthy into a chapel with a small lending library, and it remained in use as a chapel for the farming families for over 100 years, till 1978. The Wesleyans had a preaching room in a loft over the stable at Sea View House, the residence of the mines captain, and the Anglicans put up a corrugated iron church in an enclosure near the top of the Incline, and this also served as a school during the week. All that may not sound exciting to the 21st century, but it industriously served local  needs in the high Victorian age. Beulah fell into disrepair when the mines closed, but after twenty years of neglect it was restored in 1911 (?) . Services are held every Sunday evening at half-past six  and crowds fill it to the doors on such occasions as Harvest Festival, Chapel Anniversary, Christmas Carols, and musical evenings.

Fortunately the intolerant mines captain was a exception, and “the three congregations seem to have attended one another’s meetings with a refreshing lack of bigotry. The Rev. J.     Vernon of the mission church provoked heated arguments at a lecture by his High Church views on the Reformation (and the Bible Christians replied with a lecture on ‘Honest Hugh Latimer’) but this was exceptional, and all combined in advocating total abstinence. Each June the Bible Christians celebrated the anniversary of their chapel’s foundation with special services and tea – this still continues - .and the mission church and the Wesleyans held similar Sunday School anniversaries . . .   A Brendon Hill \& Gupworthy Temperance Society  was formed, there was also a lodge of the Order of Good Templars of the Band of Hope (called Mountain Hope), a Brendon Hill Teetotal Fife \& Drum Band and a Brendon Hill Choir.’ (Sellick,p.60)

In recent years abstinence seems to have been less frowned on and derided in ‘polite’ society than it was, but the haze of the inescapable alcohol mystique in this century may make it difficult to appreciate the immense support for abstinence in all classes, except perhaps the unskilled workers in large towns, in Victorian England. Yet the reasons are simple enough. Although strong drink was cheap – as in the legendary notice, ‘DRUNK FOR SIXPENCE, DEAD DRUNK FOR EIGHTPENCE, WITH CLEAN STRAW ONE SHILLING” – it could take a good share of the workman’s weekly wage, whether the farm labourer’s 10s or the miner’s 18s. Besides that, at harvest time,  for instance, the abstainer  could earn better wages than the man slowed by the plentiful  cider provided for the workers in the fields, and that money could be spent on his family and contribute to a happier and more comfortable home. Moreover, the meetings organized by the Good Templars and other national bodies such as the Band of Hope were more than just lectures on the effect of indulgence, they offered a meeting-place to which, unlike the Victorian pub, the rural husband could take his family to a magic-lantern show, enjoy social conversation and hear a lecture or news from the wider world. Beyond this, he could feel proud to belong to a growing, prosperous organization with its codes of conduct and ceremonies, brass or flute and drum bands, vigorous, tuneful singing, public occasions and processions. The Band of Hope provided occasional week-night emtertainment and singing and, at Christmas, ‘treats’ for their children.  Above all, the organization was run by officers from their own ranks, not their social overlords. 

Naturally the members met opposition ranging from mockery to downright hostility and sometimes the threat of violence. A labourer faced a very real risk of dismissal and eviction from his tied cottage if the farmer disapproved, and an outspoken tradesman could and himself boycotted at the farmer’s or landowner’s command. The wonder is that so many counted the cost and stayed in.

Not everyone, though, wanted high-mindedness by the truckload, and while the Brendon Hill Choir no doubt had a respectable repertoire (though no record exists), other revellers let off steam with Crosher Bailey:  (presumably sung to the tune of the students’ classic). 
  Was you ever go to Wales, Where they brew the finest ales? 
  If you want a drink on Sunday, You will have to wait till Monday
  Was you ever saw? Was you ever saw? Was you ever saw such a funny thing before?
 

\subsection{CRÔSHER BAILEY}

Crósher Bailey he wass born \\
On a fer-ry windy morn, \\
But the people all objected: \\
Twass before he wass expected. \\

Chorus: Wass you ef-fer saw, wass you ef-fer saw,  \\
   	Wass you ef-fer saw    such a funny thing before!  \\
Crósher had a sister Bella, \\
 		And she had a great umbrella, \\
And she thought so much about it \\
That she nef-fer went without it. \\

Crósher he wass a clat-cutter \\
And so earned his pread and putter,  \\
But he liked not cutting clats \\
And said, “I’ll go gulling flats.”  (tricking greenhorns) \\

Crósher then he went to college \\
For to get a little knowledge: \\
But therein he did not well, \\
For he nef-fer could for spell! \\
 	
Cróher then at length wass die \\
And the people all wass cry, \\
For it cost full many a pound \\
To lay Crósher under ground. \\

Crósher Bailey he wass bury
And the people all wass merry,
Feigning to sing “Vital Spark”, 
They wass shouting, “What a lark!”

Crósher then wass not respected,
Which I guess you have suspected,
But if not you now will laugh
When you read his Epitaph:	   	      	
Epitaph
Here lies Crósher the Clat-cutter
Who would have he wass Flat-guller;
He gulled but one of all the flats:
Himself --  the thickest of all clats.	
             (Hebridean Islanders and some Welsh who were not familiar with English were reported as habitually pronouncing  b, s and v as p, ss  and f  - very probably true).	  	
  The mining population, to judge by the census, was 340 in 1861; in 1871 it had risen to 740, perhaps its highest level, and in 1881 it had dropped to 530; but seen against the population as a whole it was a minority of only 5%, 10% and 8% respectively, and over half of these were concentrated in the purpose-built settlements of Gupworthy and Brendon Hill. The mining families do not seem to have mingled or socialised to any extent with the local population. This may be regrettable but is hardly surprising, for the Exmoor and Brendon Hill born-and-bred have always preferred the company of those who share their everyday concerns, and when they welcome outsiders, do not make a pantomime of it. Besides that, these newcomers had a different tempo of life, different ways of thought, different habits, and above all, different ways of speech To local ears, the speech of an up-country man was even more difficult to catch than that of the English-speaking Welshman. Only with the Cornish could he feel “at home”.
     One need not look far for a reason, and in the light of the frequent religious revivals that had taken place at various places in Cornwall since 1800, and in this part of Somerset since 1820, it can be found in the miners’ chapels, especially Beulah, for many of the Cornish miners were of the Bible Christian faith which had commanded a numerous following among the farming folk in these hill-country villages since 1820. Moreover, three of these miners were local preachers, which won for them respect and acceptance in a wider area than merely the neighbourhood of the mines.  
  The very number of mines – thirty in an area of little more than ten square miles – may give the impression that ore could be had wherever one cared to dig; but this is deceptive, for in fact the ore, though generally of good quality, ran in narrow and erratic lodes which sometimes petered out before the yield had properly covered the initial expense. It was this unpredictability that made so many mines necessary, and some of them had short histories: Florey Hill (1865 – before 1867), Carew and Roman (1865 – before 1873), Colton (1865 – ca 1876), Elworthy (1872 – 1877).  The ore could find a ready market at almost any time in this age of industrial expansion, but “almost” is the qualification, because in 1869 the steel industry began  to suffer a two-year recession and the price of ore dropped from 10s a ton to only 5s 8d  and then 7s.  Then, in 1871, things “began to look up.” Ore fetched 20s a ton for the next three years, then dropped to 15s and 14s, and although the output from the mines rose steadily, the income scarcely kept pace: 
(Output (to nearest 1000 tons) and Net income (excluding all expenses)

        1871 – 28,000 tons -     £27550
        1872 - 28,000  tons -      £27910;
        1873 – 29,000 tons -      £28980;
        1874 – 38,000 tons -     £28360  (because of fall of 27% in price per ton)
        1875 – 42, 000 tons -    £31340;
        1876 – 41,000 tons -     £28950 
        1877 – 47,000 tons -     £32,820 

 and 1879 – 14,000 tons -    £11,500 -  Ah, what a fall was there! 

The miners were in no way to blame. Like miners elsewhere at the time, they were producing too much, and they eventually suffered for it along with fellow-workers on the land.  Through the eye of history 1876 and 1877 are seen as the years which began a decline of English farming. The disasters of two drenching summers and consequent loss of harvest were compounded by two new and huge imports: grain from the prairie wheatfields of Canada and the United States, and frozen meat from Australia and New Zealand. Alongside this, the iron industry suffered a deep depression, due, it was said, to “excessive production and increased and cheap manufacture of steel,” (Haydn’s Dictionary of Dates).thanks to Siemens and his “regenerative gas furnace” and other improvements. One might think that this would solve Brendon Hill’s problems, but in the same year of 1876, when  a disruptive three-year civil war in Spain came to an end, the Spanish iron ore mines started exporting again and the ore was cheaper and had a higher iron content  than ours. The Ebbw Vale Company reported a loss of £42,000 on operations in 1878, and in January 1879 shut down their main (?) blast furnace. Brendon Hill had no chance, and on 10 May the mines closed with a bare week’s notice. (7580) Most of the Cornishmen left to find work elsewhere, but by 1881 there still remained, at Lower Gupworthy, John Richards, aged 49, from Sithney, and his family of five; at Gupworthy New Cottage, James Hoskins, aged 50, from Breage, and his wife; at Brendon Hill, Joseph Dyer, aged 51, from St Austell,and his family; and of the Welsh contingent, at Goosemoor Square, John Peek, aged 27, of Tipton Merthyr, and family of five; and at Brendon Hill, Thomas Martin, aged 53, from Glamorgan and his wife Mary (aed 50) from Crown, Cornwall; and at Brendon Hill 20 more, with 90 in their families. (Their names will be found in Appendix) 

These, and thee local men who stayed on in hope of better times, suffered much privation before the mines reopened in the following November. 
       For a fair while it seemed that their hanging on had been well worth it and that the good times were here again in full measure and brimming over. “At Gupworthy both the Old and the New Pit were deepened, Goosemoor and Kennesome Hill were greatly
Extended and a tramway laid from the latter pit to Gupworthy station, the trains being pulled by horses”. Three other mines – Lothbrook, Burrow Farm and Colton – which had been closed for some years were reopened, and Colton yielded so much ore than an extra siding at Brendon Hill had to be provided to receive it.               
         Output soared, from the 14,000 tons in 1879 (part year) to 27,600 in 1880, 26,300 in 1881, 31,300 in 1882 – then the crash as recession set in. Spanish ore was now coming in at 11s a ton, and Brendon Hill’s costs of extraction and transport the 10,000 tons of 1883, with an end price of 15s a ton, made it impossible to compete. The Company began closing the mines at the end of 1882.By midsummer of 1883 only Kennesome Hjill, Burrow and Colton were still working, and these closed in September. The miners were paid off, but as the closure was spread over a year it caused less distress than in 1879.
     In the next year the movables of the industry were sold off by auction by James Phillips \& (???): 

    Brendon Hill returned to its old pastoral life, but the industry, though vanished, had left its mark on the land. Ten years after the closure, a visitor, J. Ll. Warden Page, wrote: ‘The gaunt chimneys, the ugly pumping houses, do not improve a landscape already rendered sufficiently dreary by the rows of ruinous cottages ordering the roadside. There is in particular a chapel, inscribed ‘Beulah’, whose blistering walls, boarded windows, and overthrown railings are a sad commentary on its title.  A parish doctor told me he could remember the time when over one hundred families of miners occupied the village; now, with the exception of half a dozen cottages, let at next to nothing, the place is worse than Goldsmith’s deserted village. . . Silence reigns where, a dozen years since, the air resounded with the cheerful, if not always melodious, voice of commerce.’   
   Even so, for several years the Neilson engine continued to chug along the Gupworthy line, for although the ore traffic was gone, several years were spent dismantling the mining gear and sending it down to Watchet for salvage or eventual breaking-up. The cottages near the head of the Incline were demolished and the stone sent down to Watchet, where it was re-used to build a row of small houses in West Street. The iron church was dismantled and later re-erected in the Wansborough Paper Mill, where regular services were held (WHEN?). 

   Now that the original purpose for which the mineral line was built no longer existed, it might have been expected that the directors of the Company would close the line, lift the track, ship the rolling stock over to Wales and leave the local people to shift for themselves, but they did not, for they had a legal obligation to fulfil.  But although the tonnage was only a tenth of what it had been, the line soon began to deteriorate when the annual grant to the contractor responsible,                      , was reduced from £460 per annum to £380, so that in 1886 the engineer reported that the section from Watchet to Comberow, which had been in good shape only a year before, was now “becoming overgrown with weeds and otherwise drifting into a state of neglect”. As time went on, wind and weather compounded the neglect. The two Brendon brooks which meet at Roadwater have, over the ages, created a Prospero’s valley, they purl and ripple and fill the ear with sounds and sweet airs, that give delight, and hurt not; but when the clouds burst open on Brendon Hill, whether in the depth of winter or the height of summer, the crystal flow changes in a moment to a rushing  tide which overflows the banks, spreads out  to flood the cottages waist high, and, in subsiding, leaves a deposit of red mud which disfigures the furniture and can never be wholly washed away. One such onslaught in March 1889 underminded the railway in several places where it ran near the stream, damaged a bridge at Roadwater, caused a landslip at Clitsome, and quite washed away part of the embankment at Bye Farm.                                                                                                                                                                                      
   
    In the valley the passenger service was reduced to two trains a day, leaving Watchet at 9.15 a.m. and 3.0 p.m., and returning from Comberow at 10.45 a.m. and 4.0 p.m., which would give Comberow and Roadwater people a useful couple of hours in Watchet but, with a ten minute walk between the WSMR and GWR stations and a wait for a connection, would only allow  a brief visit to Taunton.  Still, they valued the amenity, and right until the line closed down they piled into the three carriages at a rate of 30 a day or 9,000 to 10,000 a year.
   The railway ran goods trains as before, but the 35.000 tons per annum shrank to 5,000 tons or less, of which roughly half was coal, and half manure, lime and feeding stuffs for the hill-country farmers, to a depot established by James Phillips \& Son at Gupworthy. 
   At long last the Company received legal permission to close the line, and on 7 November 1898 the Pontypool steamed for the last time and awaited whatever might be decreed for her. Two months later, when all the rolling stock was assembled in Watchet, it was moved on to the G W R by a temporary connection at Kentsford and thence to Ebbw Vale and so passed out of this story.
  Here, too, but for a fortunate or misfortunate chance, the story of the Mineral Line might have ended also. Grass invaded the track, the rails rusted, and up on the hill the once busy village near the Incline was deserted and falling into ruin.  An air of desolation brooded over the scene, and since it was no one’s responsibility to restore the landscape to its pastoral state, the dilapidated engine houses and chimneys served as painful reminders of an enterprise that had failed. And in a final blow, emblematic of the new century, on the night of 28-29 December 1900 a violent storm battered Watchet harbour, wrecked the 50 boats sheltering there, and destroyed the west pier from which the iron ore, in better days, had been shipped to Wales and  the twisted rail could be seen dangling forlornly in mid-air. Inland, the storm wrecked signals and blew down the crossing-gates at Torre..
  Strangely enough, however, and against all expectations, in due course the disaster brought about a renewal of the mining and the railway. The Watchet harbour commissioners had not the authority to raise the necessary funds for rebuilding the pier, and so the town was constituted as an urban district council in 1st April 1902. Two years later, in spite of another violent storm in September 1903, the port was rebuilt with a solid west pier replacing the old openwork structure.  The engineer in charge of the rebuilding, H. Blomfield Smith, M. Inst.C.E., learnt with interest of the abandoned mines and railway, and as the steel industry was prospering he looked into the possibility of re-opening them.  The Ebbw Vale Company, though no longer interested in working the mining are, still held the lease with (presumably) 50 years yet to run, and at the end of  February 1907 the directors of the mineral line were asked to approve a draft agreement allowing Mr Blomfield Smith the use of the line. A fortnight later (11 March) the Somerset Mineral Syndicate Ltd was formed to work the mines, leasing the railway as far as Brendon Hill. Mr Blomfield Smith was appointed managing director, and the Syndicate was to have a capital of £20,000 with £12,000 in debentures – a modest total which perhaps spoke of caution and reserve on the part of investors because of the problems of 30 years before.                                                                                                                                     
     The Syndicate meant to work the ore by means of adits – tunnels driven more or less level--- rather than vertrical shafts,  with cableways running to Comberow, thus avoiding the need for expensive pumping and winding gear. The railway would of course be needed again, and the Syndicate  leased it for two years from  20 June 1907 for a total of £3,000 payable in four stages, and had to put the line into repair  
      A former Metropolitan Railway 4 – 4 – 0 engine, bought second hand, was supposed to be delivered in Watchet on Sunday 23rd June, but an argument with the G W R over the use of the water troughs at Creech, east of Taunton, delayed this till the last day of the month. Then the locomotive was hauled “dead” from Taunton to Kentsford, where it was transferred to the mineral line by a temporary connection, and from there allowed to run down the gentle gradient to Watchet with two truckloads of passengers. The railway track needed attention after the years of neglect, but this seems to have been done in no more than a week, for on 1st July a wagon-load of new sleepers was hauled by horses to Roadwater, and two days later the engine and one truck made the journey without mishap. The people of Watchet and the villages were on tiptoe with excitement, and on Thursday 4th July the Syndicate ran an excursion to Comberow, with the engine and four trucks (provided with benches) decorated with bunting
   The Free Press reported that “by 2 o’clock several hundred people had gathered in the station yard at Watchet where the train was waiting with the Town Band and members of the council occupying the first two trucks    The departure was delayed by a hailstorm but when this abated a damp but cheerful excursion set off to the accompaniment of music from the band, cheers from the crowd, and the explosion of detonators. The train was greeted with more cheers and flags at each village and hamlet, eventually reaching Comberow, where welcoming speeches defied the intermittent rain, and it was not till 6 o’clock that the party arrived back in Watchet.”  Photographs taken by Herbert  Hole of Williton  give a good idea of the enthusiasm shown by the people of Roadwater.  
  A fortnight later (17 July) the Incline was ready for use again, but the winding engine still needed a thorough overhaul, and for the time being the system had to be improvised. A hand trolley was taken by road to the top of the Incline and there loaded with old rail and concrete scrap and connected to the cable, which was also attached to an empty truck at Comberow. Two horses then tried to pull it to over the edge, but long grass on the track and rust on the brake drum made progress difficulty, and it took the best part of a day before a second pair of horses succeeded.
  The effects of the long neglect soon made themselves apparent. On 14 August a party of platelayers were coming down by trolley from Comberow to Watchet, but drizzling rain impaired the brake(?) of the trolley as it neared Roadwater. Two of the men jumped off, but one of them fell in front of the trolley and was seriously injured. It crashed into the crossing gates and badly damaged one of them. Fortunately the other men were uninjured. 
   The Syndicate’s first venture was to clear out the Ebbw Company’s old workings at Colton, apparently in order to tap into iron ore quickly and get a return while preliminary work was being done on a tunnel at Timwood. The first load was sent down the Incline on 14 October, but it soon became clear that the best haematite had already been worked out and only spathic ore and very soft brown haematite remained, and extracting these involved considerable expense in timbering.  The planned aerial cableway to Comberow was never constructed, and instead, a 2-ft gauge light railway was built. Underground the trams were pushed by hand and then brought out through the west (???) adit in Galloping Bottom; there the ore was tipped into bins and from there into light railway trucks at a loading bay. These trucks were then drawn by cable to the hilltop 250 ft above, by a double-track incline 600 yards long – less impressive than the master-work, of course, but still challenging its operators, as the load was against the gradient and the slope became steeper as it descended.  From the winding house a single line ran to the main road and thence westward beside the road to Clipper’s Pool, where a timber viaduct carried it over a dip in the road. It carried on past Raleigh’s Cross Inn, taking the Bampton road past the then ruined chapel and thus to the old WSMR yard at Brendon Hill, where the ore could be tipped straight into the standard-gauge wagons below. Two 0 – 4 – 0 tank engines worked this section.
   The Syndicate’s second venture was at Timwood, about       yds from the foot of the Incline and on approximately the same level. It was to be a deep-level water adit with a tunnel which they hoped would cut into the veins of ore of the old Raleigh’s Cross mine. .As a sign of the new times, a wooden changing house was built at the mouth of the adit, and a boiler, engine and air compressor were added to work the mining drills. This adit was linked directly to the mineral line by an extension of the 16 inch mine tramway, raised above the valley floor by an embankment of spoil from the mine; it crossed the line by a moveable crossing to a loading bay. (The trolleys were pushed by hand).
    These two mines employed about 75 men, and for their accommodation the old Vickery’s shop at Brendon Hill was turned into a lodging house, mainly for the single miners, and others were housed in a terrace of brick and wood shanties put up by the Syndicate or in nearby villages. One of the shanties at the west end of the terrace served as a butcher’s shop on Saturdays, while other provisions could be had at David’s stores. The miners were paid on piecework, and on an average of 35 shillings a week in those days of cheap food and stable prices they were quite comfortably off. But they worked and sweated for it.
  Within less than a year of opening, however, the Syndicate ran into difficulties. On 27 June 1908 the Free Press reported:
    ‘The sudden slump in the steel trade has been a serious setback to the Brendon Hill mines, but the work will go on.  At Colton, a seven foot lode is being worked which yields ore containing 61% of good metal, and at Timwood there are evidences that iron bearing ground is in near touch, and there is nothing in the state of the Syndicate or the working to warrant the rumour that the mines are coming to an end.’
   That was putting a brave face on it, because while the ore from Colton may have yielded a good percentage of metal, it was soft and clogged the furnace during smelting, and therefore of little value. The immediate measure taken was to form the ore into briquettes in a local lime kiln, but this did not work.  The Syndicate’s initial funding had been too slender to carry it through hard times, and by March 1909 it had no funds left at all. and pumping ceased at Colton and the lower levels were flooded. Timwood was no better: the tunnel had been driven 1600 ft but had sill not reached ore-bearing rock.
   One desperate venture remained: to build a patent kiln to convert the ore into saleable briquettes (which the lime kiln had not been able to do) and thus earn enough capital to get the mines working again. A Watchet Briquetting Syndicate was formed for this purpose and registered on 5 April 1909, but it took another two months to raise the comparatively small amount of £2,000. With this,  J.Osman \& Co (of Taunton??) were commissioned to erect a kiln on land next to the Washford station leased from the railway company. But in the words of a rural ditty popular in village concerts at the time, “For marnin’ to night, Nort never went right”. Some 80 tons of briquettes and 146 tons of ore were shipped in November, but the steel trade remained sluggish. The briquetters struggled on a little longer, but gave up in the end and on 22 March 1910 its property was sold, and shortly afterward the great kiln was demolished. 
 Hoile
      Most of the workmen lived their toilsome (laborious) quarter-century on Brendon Hill and departed, leaving no clear personal memory among the permanent residents, and no record of their existence here but a name on the census roll; but one notable exception was James Hoile.
  Jim, to give him his everyday name, was born in Kingsbrompton parish in 1855, and became a “stationary engine driver” at Gupworthy, where he and his wife Harriet occupied one of the cottages in The Terrace. He could write only with difficulty but he had an inborn talent for engineering, and in his leisure hours, such as they were, he made a working model of the pumping engine in his charge. He may have had a little assistance from the company forge, where he is said to have cast the brass steam chest in a cocoa-tin, but the model was not merely a show piece, for he harnessed its power to drive his wife’s sewing machine – which, all in all, was a pretty remarkable feat for an untutored countryman in one of the remotest parts of Somerset at that time. He lost his job when the mines closed in 1883 (?), but when they re-opened in 1907 he was taken on again, this time in charge of the great brake engine at the top of the Incline, and here he was photographed at work oiling the gear with tallow from a kettle. 
  In his last year he was much exercised over the fate of his model, for very few people at his level cared for relics of the past with no obvious monetary value. He felt that for this he could only depend on the Roadwater postmaster, my father William G. Court, who accepted his bequest and thus preserved one of the few small artifacts of the Brendon Hill enterprise.
  Modellers have wondered how Jim “got the thing to work without a boiler”. Ingenuity again: he connected the boiler by a leather tube to a kettle boiling on the hob and tied down the lid! – or so runs the legend.    
     
ROLLING STOCK
The locomotives working the line were:
The two  0 – 4 – 0 “box” (square saddle tank) locomotives made by Neilson \& Company of Glasgow, and at first working the line from Watchet to Roadwater until damaged in a collision. One of these, when repaired, was “promoted to glory” on the Brendon Hill extension and worked it until the mines closed. 
    To replace one of these, the Directors ordered the Rowcliffe (0 – 6 – 0 which worked the lower line single-handed for seven years and was overhauled in 1865; 
   To supplement this, the Sharp Stewart locomotive Pontypool (0 - 6 – 0) ), upward of 28 tons,  “long boiler” , more powerful, to haul the passenger trains. Its brakes were inadequate and had to be strengthened.  It was twice returned to Ebbw Vake for overhaul, and during its absences its place was taken by the
	Esperanza (0 – 6 – 0 ) from Sharpe Stewart and the
            Whitfield, but these left no trace in popular memory.
  When the mines re-opened in 1907, the locomotives (nameless, unfortunately) were a former Metropolitan railway 4 – 4 – 0, (47 tons unladen), and a 
   Kerr Stuart narrow gauge engine to work the light railway from Brendon Hill, past Raleighs Cross and Clipper’s Pool to Colton.  








				







NOTES
1 Personal reminiscence.
2 Lothbrook refers to the Lethbridge family’s claimed descent from the famous Viking Ragnar Lobrog (Leather Breeches)
3 The estate belonged to the Hon. Mrs Trollope, of Crowcombe Court. The main road here separates it from  the Nettlecombe estate of Sir Walter Calverley Trevelyan,  who would not allow a public house anywhere on his land.  The stone cross originally stood on the other or Nettlecombe side of the road. 
4 Joyce,W: Echoes of Exmoor

(10,020 wds 6 March 2009)   (11 March 2009 – 10226 wds)  (10,336 wds);
 (13.3.2009 – 10,420 wds)       (10,480 wds)  (     17.3.1009 – 10,600 wds) 
 10,770 wds)   (18.3.2009 – 11,.080 wds 24.3.2009 – 11,220 wds)   16.5.2009 –11,290 wds; 18.5.2009 – 11,460 words)   19.5.2009 – 11,566 wds ; 23.5.2009 – 11,689 wds); 27.5.2009 – 11,855 word;  	28.5.2009 – 12.025wds;  1.6.2009 –12,220 wds;  2.6.2009 – 12,470 wds;  4.6.2009 – 12,735wds;  5.6.2009 – 12,980 wds; 17.6.2009 – 13.080 wds;  22.6.2009 – 13, 290 wds; 91.7.3012 – 13,670 words
 

				 
 
 








\end{document}