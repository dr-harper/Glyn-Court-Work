\Chapter{My First Visit}{CLare Court}

During the first years of our married life we led a rather nomadic existence, living for varying periods in Exeter, Paris, Grenoble, Yorkshire and South Devon, and during this time one tended to regard the home that one had left as a convenient repository for all the possessions, wedding presents and books that one had accumulated over the years. My own family had never been given to hoarding, in fact my father tended to the reverse and was a thrower-away. I could hardly fail to notice therefore the marked contrast in my husband’s home in the village of Washford in West Somerset. He was the only child of rather elderly parents with their traditions and customs very firmly rooted in the Victorian era. His mother had been born in 1886 and father in 1876 - quite a generation gap. In the course of their own industrious and even lives changes had been very slow to occur and possessions were valued, often the result of years of patient labour and saving, and in the sober, industrious, rural society of which they formed a part, it would have been unthinkable to squander uselessly any article that had been either laboriously acquired or reverently handed down from a previous generation.

It was not surprising, therefore, however strange it seemed to me at the time, that my husband's home seemed to be full to overflowing with family possessions. Every room (and there were many, for his father had built the house about twice the size of the average modern home) was crammed with furniture, of which every cupboard, shelf and drawer exuded ornaments, runners, vases, trinkets, hairbrushes and the like. Every floor was generously linoed, carpeted and festooned with runners. In addition to curtains, many of the windows had blinds. Investigation into corners of little-used rooms soon revealed further caches, many consignments of boxes, some cardboard, some wooden, and these too held their store of table linen, crochet mats, unwanted Christmas presents, with quite a few toys which had been lovingly preserved from Glyn's childhood. I suppose that that in itself would not have made the house unique, but in addition to the main structure (which as I have already mentioned was designed and built by my father-in-law) there was a family business comprising a rather unusual blend of village post office and boot and shoe store. There were several outbuildings and sheds, dimly perceived, in which I was vaguely aware that something was being carried on. In the fulness of time I came to realise that the other part of the business comprised a shoe and bicycle repair service, which occupied a large corrugated-iron shed directly to the back of the house and, to my eternal annoyance, directly blocking out the view of the orchard from the living room. However, as I was to come to realise slowly and painfully over a long period of time, village folk with a very demanding post office to run as well as a general business were far too near the breadline throughout the hard years of the thirties to worry too much about such niceties. We have now lived here for eight years and the shed is with us still, together with its partner, an unlovely but commodious tin garage of vast proportions. At one time there was brave talk of a "five year plan" in which we hoped to demolish these ugly sisters but after eight years they are still with us, becoming ever rustier but still defying the attack of westerly gales and the encroachment of rust. It is surprising what the eye can become accustomed to after a period of years - I shall probably be quite put out when they are finally demolished.

There had been a very good market for bicycles throughout the 1920's, and my father-in-law William George, ever willing to expand his very limited trade in the small, not very wealthy community in which he lived, had ventured into the sale of Raleighs and New Hudsons. This had proved popular so that in addition to the great sheets of leather hanging in the workshop, multifarious boxes of tacks and so forth essential to the affixing of new soles and heels, not to mention iron bars not unworthy of a fair-sized horse, there were always supplies of new tyres, wheels and tubes. Presiding over this untidy but homely array of useful objects beamed William, neatly turned out in old fashioned striped shirt, round collar, bow tie and always with a clean white apron. He disliked being dirty. There was usually a young lad or man working away in the shed, generally a country boy (or even two) called by some such name as Bert or Les. By the time that I appeared on the scene, cheap labour was no longer quite so readily available and there was a boy of fifteen named John. Pride of place in the workshop was taken by two enormous wooden benches on which all the work was done, and most important pieces of machinery - the shoe finishing machine, a cast-iron miracle of smooth cogs and wheels, and the splendid Singer sewing machine. This had supplanted the old one made by Grandfather, which in its turn had removed the necessity of doing all the stitching by hand. By the time I first came to Washford in 1949, shoes were no longer being made on the premises, for the factories had long been turning out footwear far more cheaply and easily than they could possibly be made in the village, and besides William's heart had always been in wood as a medium of creative expression and he had never excelled in the art of shoemaking. He could nevertheless make a pair when occasion demanded. We have one still. Rather characteristically they did not turn out quite as they should have done (owing it was said to the poor quality of war time materials) and were never worn.

There was a continual coming and going to this shed, and the door would frequently burst open for John to rush forth and serve petrol to an occasional passing motorist - this latest addition of the pumps providing quite a useful extra source of income but not exactly enhancing the appearance of the front of the house.

I am still trying to given an impression of the house, home and business as it was when I first came to know it and I realise that so far I have said little about the actual house. It is hard to remember much of one's thoughts as a young bride of twenty-two years, as at that age I knew nothing at all about houses and allied subjects and as my experiences had been so very limited, I had very little to go by. All I can remember thinking was that it was large, rather gloomy and terribly old-fashioned, I think it would be true to say that I was completely impervious to the charms of polished oak and copper warming pans. I also thought it very strange that my mother-in-law had to disappear into the post office at frequent intervals in order to carry on strange transactions - all very mysterious, as the phenomenon of the Working Mother had never come into my ken before. Equally strange was not being able to have meals when one fancied them but according to the stern law of Closing Time or Opening Time. Tea could not be eaten until six o-clock - a complete reversal of the laws of Nature, though as the years went by and business declined, we did coax Granny into softer ways -but only at the price of leaving the door into the shop partly open. With the intolerance of the young I thought it was all terribly boring. It did not occur for a moment to Glyn to explain any of this to me as for him the way of life was as natural as the tide coming up and our popping down for a swim was to me. It is surprising what little bouts of friction can be caused by such a trivial matter as the hour of afternoon tea - especially as they forgot to point out that on Sundays it took place at five o-clock in order to allow plenty of time to get to evening chapel. All very confusing.

The village in which the house is set, though situated in one of the most beautiful parts of the West Country, bears no claim to the title of Prettiest or Best Kept. In fact, considering the beauty of the surrounding countryside, it is of remarkably dull and plain appearance with few buildings of any architectural interest. The exception that proves the rule, of course, is the well-maintained ruin of the Abbey of Cleeve, a Cistercian foundation of the twelfth century. Apart from this attractive pile only a few of the older cottages, thatched and rose-adorned, satisfy the romantic imagination. Most of the older houses are solidly built of stone with no-nonsense slate roofs, reminiscent of Cornish villages. Of recent years the inevitable modern development has taken piece, the worst vandals in this respect being the local authority of about forty years ago. Imagine a sheltered hollow in the gentle green hills of Somerset, generously wooded and watered by a sturdy stream. The name of the river, which once supported twelve mills, was mysteriously lost generations ago, and even in old manuscripts there is never any hint of a name. This is one of the small mysteries of our local history and it would be one of the happiest days of our life if some hitherto undiscovered tablet or parchment could be found with the solution to our topographical mystery. On Ordnance Maps, to be sure, the name is marked as Washford River, and so it has been known for the last few generations, but "Why?”  one asks oneself. Why, when all the neighbouring brooks - Pill, Avill, and Swill, Haddeo, Quarme and Oare Water, rejoice in Celtic names, should ours, a respectable full-bodied affair, have none? Admittedly, most of them mean water in some form or another, but oh how much more poetically expressed! The often humblest rivulet barely two foot wide in much more remote areas such as Dartmoor, for example, bears a name; one thinks of Walla Brook and Cherry Brook as lovely examples. Other writers have fancifully called our river the Road Water after the nearest village upstream, but logic and etymology both reject this suggestion. So until folk memory is stirred or a scholar turns something up we shall have to rest content.

The valley, then, is beautiful and fertile, as the monks, who called it Vailis Florida, well knew when they founded their abbey. The position is admirably sheltered, especially from the brisk breezes of the Bristol Channel, and its cosseting warmth is adequately proved by the forwardness of spring flowers in the cottage gardens. As I write, in the first week of May, my own garden is in a particularly lovely phase where flowers of at least two seasons are happily blooming together and will continue to do so for at least a few more weeks. Primroses which started blooming last October, are still putting out more buds. Wallflowers likewise have been flowering almost non-stop throughout the winter and icy spring, while stocks, purple irises and masses of bluebells, pink and white as well as the traditional shade, are making big splashes of colour all over the garden. As if this abundance were not sufficient, broom and forsythia are lighting up yellow torches, while the orchard is a mass of cherry and apple blossom. Nature’s bounty is indeed abundant, but in case the reader is beginning to think that he too will come and share this earthly paradise, I should hasten to add that there is a snag. Perhaps to say that "only man is vile" is rather overstating the case, but the hand of man has alas not always enriched the landscape in the same glorious way as those mediaeval builders who first conceived the Abbey. His particular atrocity in the case of our erstwhile peaceful hollow is, not surprisingly in this motor age, the main road running right through the centre of our village. Not so long ago, less than a hundred years, each community lived out its life in splendid isolation. There was a road, to Williton (I wrote "from” and hastily changed it before being executed by the locals) but the route to Watchet over the hill had always been fraught with hazards, but if you wished to proceed to Luxborough you would have had to abandon your vehicle and walked across fields; for the road petered out beyond Roadwater. For better and for worse we are on the main route to Minehead, a heavy burden to bear since the motor car explosion brought the thousands to Butlins. Sadly, our beautiful valley is to be marred by a hideous viaduct when the projected by-pass is constructed. We long for the day when the holiday traffic, amounting to eleven thousand vehicles on a summer Saturday, will be removed from our front doorstep, yet mourn in anticipation the destruction of the environment. But people must live, and there is little doubt that the present situation virtually destroys the social life of the village, rendering as it does, a short journey to the post office a positive hazard to life and limb.

Another small bonus which should not be discounted is the fact that our community is much more representative of a true unit than many of the surrounding villages which, suffering from a surfeit of prettiness, have attracted so many retired couples to live in them. We now live surrounded by elderly neighbours, many of them very pleasant and intelligent people, but many others not sharing at all in our strange tribal ways, harvest fetes and the like. No doubt when the main road is removed from our midst we shall see a new growth of bungalows on the outskirts

In 1950 when I first came to Washford, the train was still running and was still quite an important part of village life. More than one householder set her clock by the 7:50 to Taunton which for nearly one hundred years bore office workers and school children to their day's employment, returning them promptly at 5:45. In fact my first arrival was at the station, which in those days, not so long ago yet very distant in some ways, boasted a full-time station master and, with its well-tended garden, had won the title of "Best-kept Station" on the line. The G.W.R. was a relative upstart in the history of local transport. The West Somerset Mineral Railway had been established for twenty years when the Great Western appeared on the scene, though they were never serious rivals. The Mineral Line had been constructed in 1855 in order to bring iron ore down from the Brendon Hills to the port at Watchet, about which more will be said in due course. For about two miles from Washford to Watchet the tracks ran parallel and now that both have fallen into disuse, stand as silent, leafy memorials to the coming of the industrial age. Until quite late in the history of the Court family business, much of the stock for the shop had been delivered by rail. Boots and shoes and bicycles from Bristol and Birmingham and Exeter used to arrive in large crates and had to be fetched from the station with a handcart. Proximity to the station was a very important point for a small business whose customers rarely possessed their own transport, and who depended to a very large extent on the choice of stock available at the local store. But already in 1950, dependence on the railway line was much less than it had been and road transport was altering the pattern to a very large extent.

I remember quite well my first arrival at Washford Station, and with typically feminine exactness which brings to mind details quite irrelevant to one's history or the progress of the human race, can even recall what I was wearing on that occasion. It was two days after Christmas, the day we had chosen to announce our engagement to our parents. Although I only had to travel fifty miles from my home in South Devon to the Somerset coast, the journey seemed endless, taking about five hours altogether. I remember taking the first Hart's bus which was running on that Sunday service day, at about ten o-clock and hoping for the best. Transferring to the Devon General at Exmouth I reached Exeter where in the fulness of time a train bore me the thirty miles to Taunton. There was no Minehead train for about two hours, so I ventured into Station Road and for the first time in my life had lunch in a hotel (the only place that was open). I remember thinking that this was very daring. I finally got to Washford at about three o-clock where Glyn met me at the station. My overriding feeling at the time (apart from the thrill of being officially engaged) was the innocent desire to please my future parents-in-law. I had nothing to fear, they welcomed me with open arms, called me "dear” and seemed to accept me without further fuss as one of the family. Easily the hardest problem to overcome was how to eat the meal which they had kindly saved for me. I wished in later years when I came to know her better, that I had asked Mum whether she was nervous about meeting me - alas one does not think of these things until it is too late.