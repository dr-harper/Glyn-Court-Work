\Chapter{Treasure Trove}{Glyn Court}

My wife has written of our putting together the \quotemark{museum} and recounted the associations of some of the articles it contains, and I should like to continue the story, for scarcely a single exhibit is without a history to bring before the mind the characters who flourished and the social life that prevailed in this valley a century ago. Musing upon the lives of those men and women and the privations they endured, and the harvest that has come to their sometimes uncaring descendants, I am constrained to use of them words used nearly two thousand years ago: \quotemark{As poor, yet making many rich}.

But now to the homely treasures.
 
\section{The Chest} 
 
The chest had, from the look of it, been through storm and flood; its metal clasps rusted here and there, its leather cover frayed; but still stout and sturdy, a Long John Silver of a sea-chest. A pattern of brass studs stood proud of the lid in the monogram of its original owner, H. C., Henry Court, my great grandfather.

Great-grandfather Henry, however, was not the old sea-dog some might like to claim for an ancestor. For a man born almost within sight and sound of the sea he had remarkably little salt spray in his veins, and that little, in his descendants, has been diluted almost to vanishing point. Still, he had spirit and even in his youth he showed determination not to be cribbed within the confines of a rural round and appressive poverty. He was born in 1802 and in the hard years after Waterloo, probably in his late teens, he set out for London to seek his fortune. He failed to find it, and this set the pattern for his descendants, but he did - which was more important to him - learn the trade of a cordwainer. After ten years away he came home with his travelling trunk, married and set up in business in the little village of Roadwater. But while in London he had also been in service in a gentleman's house (of the Trevelyan Family) and had there evinced an unusual love of study and reading. His employer made him a parting gift of several volumes of translations of Virgil and Homer, which I still have.

He witnessed the opening of London Bridge, but so long had he been away from home that on his return his mother did not at first recognise him.

His business throve as well as the hard times allowed and he employed several men. His was the only house in the village	which received	a weekly newspaper and the shoemaker's shop became, in true tradition, the village centre for social life and national intelligence. But additional income, when it offered, could not be refused, and when, in the early 1850s, the West Somerset Mineral Railway came to Roadwater, Henry was appointed their agent and the Company built him a house. He had a dog cart - literally - and the dog would take it down to ungerford to bring back the daily or weekly newspaper. Here he lived with his growing family for seven or eight years until his death at the age of fifty-nine.
 
\section{The Sewing Machine}  
 
Our sewing machine, discovered in a corner of	the washhouse,	was also unique not only in its massiveness, but	also by its	origin, for it	genuinely was home-made, and the work of a very remarkable man.

George Grinslade - known to his descendants as \quotemark{Granfer George} - was born in Watchet a few months after Waterloo, the son of a sea-captain, and his childhood was passed in the shadow of the poverty of those terrible years. Ill-treatment by a step-mother, he was apprenticed to a shoemaker at the age of seven, and his early sufferings and privations marked him physically for life, though without impairing his spirits. On the contrary, they reinforced his strong will. In his middle twenties he was taken to Taunton hospital with some debilitating illness, discharged as incurable and told he had at most six months to live. George thereupon bought a herbary, gathered simples, brewed potions and lived another seventy years.

Living in an age of discoveries and engineering triumphs, he was interested, so said one of his grandsons, in \quotemark{anything that had the power of 'go'}. In 1851 he visited the Great Exhibition and was enthralled by what he saw; but what took his fancy as of most practical use was an early sewing machine. He thought, \quotemark{I could just do with one of they for bootmaking}, but the price was far beyond his modest purse, so he studied the design of the machine, observed the action, and, over the next eighteen months, with some slight help from the blacksmith over the drive wheel, made a sewing machine of his own. Indeed, he improved the original; the device for raising the foot had been only an india rubber band, but George devised a simple, efficient and easily adjustable lift by means of a cam and eccentric. And having built the machine, he used it in his work for thirty years; and when, after sixty years’ inaction, I turned the wheel, a few drops of oil sent the shuttle flying to and fro as merrily as ever.

Mechanical ingenuity was only one facet of Granfer George's character, which was compounded of tenacity, industry and the inexhaustible curiosity of a lively mind. He had moreover an active interest in many movements for social progress and, with a few friends, founded a Friendly Society in Watchet. Quite early he became a Methodist, and had to undergo his share of the persecution which fell to that body in the distant days when it still possessed deeply held and firmly expressed convictions. In the course of the revival of 1859, when scores of people in Watchet were converted and joined the church, a messenger burst into  prayer meeting to tell George that some of the roughs had set his corn rick on fire. \quotemark{Leave 'em be, lad}, said George, in effect, \quotemark{I'm busy wi' bigger things than corn ricks, now}.

Not that his convictions hampered his sense of fun, and Granfer George, in his time, together with many another young blade, had heard the chimes at midnight. His workshop adjoined the Star Inn, and for a time, being a sociable lad, with ready wit, he led the conviviality next door, until the brighter star set him on a new course. George and his friends made their own entertainment in robust old-time style.

The wedding feast, for instance. It was only a thatched cottage, but there was good cheer, and the family and friends of the bride and groom had crowded in, trenchermen were trenching, if that is what they do, with gusto, and a Niagara of ale and cider was cascading down brawny Somerset throats. No one had eyes for the workaday world outside the cottage, nor for the grinning lad who cautiously drew the door to and made it fast on the outside. No one noticed the two figures who crept round to the back, where the ground sloped steeply upward so that the cottage eaves came almost down to ground level. No one heard George's laborious climb up over the thatch to the chimney-stack; no one heard the thud of heavy objects thrown up from below. But suddenly in the festivities, clouds of smoke billowed out from the fireplace as George firmly planted his clods of earth on the chimney pot.

The coughing guests ran for the door, found it barred, thrust open the small windows, and choked and gasped for air until a jar of cider \quotemark{douted} the fire.

Such buffoonery, of course, George outgrew, and his humour grew more kindly and considerate and the photographs taken of him in later life show an expression of cheerful serenity that all his sufferings could never quench. He retired from business in his seventies and went to live with his daughter and son-in-law in Roadwater. There he became a favourite with the village children, not least his great-grandchildren, and made with his own hands the few toys they had, and for years the boys cantered through the streets on a piebald steed with wooden wheels. Alas! Only his carved head now remains.

He loved the company of the very young, and would speak to them of the wonders of the world and, as a treat, display the coloured plates in his \quotemark{Goldsmith’s Animated Nature}. Even the young in the animal world drew him, and I must recount a tale which I treasure as a link between two centuries. In 1969 or 1970 we were visited by an old gentleman greatly respected in the district not only for his character, his immaculate appearance and his knowledge of country life but also for his remarkable craftsmanship in wood. Mr Davis of Monksilver was born before 1880, and he told us that in the last year of the nineteenth century, he was quietly thatching a roof on a farm up from Roadwater when he heard a voice. He looked down and saw Mr Grinslade, as he called him, walking-stick in hand, leaning over a five-barred gate and holding converse - I think no other word will do - with a group of heifers in the field; and the audible, human part of the conversation went \quotemark{Well now, me little dears, an' how old be you? About two, I expect. An' do 'ee know how old I be? I be eighty-five. What do 'ee think o’ that, now?}

Of course he had his foibles. His son-in-law welcomed visitors of an evening and Granfer would tolerate them; but when the John How pendulum clock laboriously struck ten, granfer, although stone deaf, would look up from his reading and boldly announce, \quotemark{Ten o'clock! Time everybody was in their own homes!} or \quotemark{Mrs Browning, your husband will be wondering where you'm to.}

His deafness once caused him some embarrassment. By way of preliminary, however, I must explain that in the 1870s, a certain landed family had entertained the Prince of Wales when he visited Exmoor for the stag hunting. The hospitality was lavish - the Prince would have certainly voiced his extreme displeasure if it had not been - and rumour said that the family had spent £30,000 which they could not afford. Thirty years later the squire was paying a visit to Grandfather William to discuss some matter of common interest, the Liberal Party or the abstinence cause, no doubt, and he was introduced to the old man, who registered no particular emotion. After a while the squire's attention was caught by a painting of his mansion which Grandfather's son Lewis had done when home on holiday, and he was examining it when an aged but vigorous voice interposed from the chimney corner, \quotemark{That's Crampton Manor, Squire Turrell's place, Sir. They had King Teddy down there when he was Prince of Wales, and the old lady spent thousands and thousands of pounds and ruined 'em}. Grandfather's embarrassment was extreme, but the squire saved the situation with a chuckle and presently took his leave. Granfer, when he understood the situation, was contrite: \quotemark{Why ever didn' 'ee tell me who 'twas you? I wouln' ha' said it for worlds}.
 
\section{The Lapstone}

On a bench in the iron workshop behind our house lay a heavy black stone, some eight or nine inches across and worn smooth by the tides of the Bristol Channel. Its shape was peculiars an oval face, as smooth as wax except in the centre, where an area the size of a penny piece had been fretted away; and two other faces forming a depressed wedge:

But for its colour it might have been one of the numberless stones forming the walls of our garden, but it had its history, which went back a century or so, to the time of my grandfather, William Court.

William, born on May Day 1847, was the youngest of the eight children, but the second son, of Henry. He was barely twelve when his father died, but he had to shoulder the financial burden of the family, as his elder brother, fourteen years his senior, had run away after a violent quarrel which must be recounted later. William found employment as a fitter with the mining company on Brendon Hill, and soon showed qualities of leadership in the life of the village. He had been converted in a remarkable revival which swept through the district in 1859, and at sixteen he became a local preacher on trial in the Bible Christian Society in Roadwater. A little later, the Company sent him to their works in Ebbw Vale, but Grandmother Court pined for our green hills and wooded valleys, and after eighteen months they came home. William set to work to restore his father's business and buy back the family furniture, and took the lease of a William and Mary dower house, attracted to it by a kitchen spacious enough to accommodate the sixty or seventy members of a village society of which he had become the leader.

The expansion of the iron ore mines on Brendon Hill brought an unwonted prosperity into the neighbourhood, and within four years William's business was firmly enough established for him to marry. His bride, Christian Grinslade, came from the little seaport of Watchet, which Coleridge depicted in the \quotemark{Ancient Mariner}, and she was the daughter of a man in the same trade as his son-in-law but of quite remarkable creative talents.

It was on one of his visits to Watchet that William acquired the stone for his work, searching along the rock-strewn sea-shore for two hours until he found one of exactly the shape he wanted. This is how the lapstone was used - and still may be, though I suspect it may now at last be obsolete, though I used it -reluctantly I confess - as a boy in my father's workshop. The cobbler, with a paper pattern, cuts out a new sole from a bend of leather, pares off the instep and steeps the sole in a bucket of water for half an hour. Then he takes the stone, rests it on his lap with the flat surface uppermost, lays the sole on it face down, and hammers it until it has taken on the curve of the old shoe. This lapstone remained in fairly constant use up to the 1950s and acquired a patina compounded of beeswax, shoemaker's dye and, I suspect, linseed oil which gave it the richness of leather itself. It served Grandfather faithfully throughout the few brief years of mining prosperity and the long decades of rural decline; and the business went steadily along, though the depopulation, the drift to the towns and overseas, and the poverty of the farm labourers, prevented any development comparable with the efforts he expended and the hours he worked. By the 1890s the large town manufacturers were producing shoes at prices which the local craftsmen could not possibly compete with, but William turned this to his advantage, bought a cottage and reading room over the way from his home, turned it into a shop, laid in a stock of these ready-made shoes and continued both trades with increased profitability. Even so, profit for a week probably seldom exceeded two or three pounds, but this was sufficient for him to devote the rest of his time to the deep interests of his life.

These interests, I would almost call them passions, were so intertwined and interdependent that I cannot fitly give pride of place to any one, though I am: sure Grandfather would have assigned it to the last I mean to describe. As we discovered the lapstone represented his livelihood, so we found the symbols of the great purposes of his life; the gavel, the portrait and the plan.

The gavel was a simple, unimposing one of rosewood with an ebony handle; you felt that it had figured at meetings for form's sake and that no vigorous use of it had been needed because the tenour of the meetings had more often than not been harmonious. However that may be, finding the gavel brought us to the society which Grandfather had been deeply concerned with.

His childhood, even before the death of his father, had been scarred by tragedy: one sister had died as a child, another had borne an illegitimate child, and his elder brother, in a drunken rage, had struck his father down, left home for South Wales and, like Wordsworth's Luke, in the city given himself to dissolute courses. His brother's treatment of his father profoundly distressed young William; he determined to devote himself to attacking and, so far as he might, eradicating the causes of all such tragedies. Reading of the work of the Good Templars temperance organisation, he established a lodge in the village, and in a very few weeks sixty others had joined, attracted, doubtless, by the cause itself, others by the social life, others, it may be, by the element of ceremony. William, as was natural, earned great unpopularity with a certain element of village society; indeed, he would have considered he had failed if he had not earned unpopularity, for he was out to smite the enemy hip and thigh, giving no quarter and asking none. His own church, though nationally among the pioneers of the movement, was locally opposed to such modernistic notions and William came near to expulsion. However, he won through and in due time his work received a visible reward. As I have mentioned, he had leased a dower-house in the centre of the village because the kitchen was the one room large enough for meetings of the Lodge; but as members increased, they needed an even larger place to meet. He approached the lord of the manor in the adjoining parish of Nettlecombe, Sir Walter Trevelyan, who was an enthusiastic advocate of abstinence from alcohol and a patron and president of the United Kingdom Alliance. Sir Walter listened sympathetically, promised to help, and very soon built a spacious, well-proportioned hall on the edge of his property, within the village, for the use of the Roadwater Good Templars, requiring them only to give hospitality to other good causes when called upon. The Temperance Hall was opened with great excitement on a sunny July day in 1877 and William was presented with an illuminated address and an engraved and inlaid writing cabinet in recognition of his services. The hall served well and remained in pretty constant use not only for temperance gatherings but also for rallies, Sunday School anniversaries, Good Friday teas, and village concerts, for seventy years, and, though in sad disrepair, it still stands.

The portrait mentioned earlier was, if not unique, at least a highly unusual Victorian souvenir with a likeness stamped on a wafer-thin tongue of wood. I have no idea how or where it came into his possession but it represented the third cornerstone of his life, for it showed the People's William and bore the inscription \quotemark{Printed on wood from a tree felled by Mr Gladstone at Hawarden}. Grandfather did not, so far as I know, inherit the Liberal faith, but it was the inevitable political expression for a man of his convictions, and I imagine he came to it in the heady days of Mr Gladstone's first administration. Certainly, for the rest of his life, the times of triumph and the more numerous later days of disappointment, he boldly defended the faith, at considerable cost to his livelihood and standing. Political activity in the constituencies was more sporadic than nowadays, but at Parliamentary elections William with his ready speech, clear wit and command of language was the chairman for all Liberal meetings. Moreover, he was an assiduous canvasser, though admittedly he was ploughing stony soil among the predominantly Tory and Anglican farmers, and he often chuckled over a misadventure that befell him in a farmhouse on the hills near Withycombe. He had ridden over on his grey pony to try his eloquence on the farmer, but soon found that he could not budge him an inch. As he was about to leave, the farmer's two sons clattered in, grinning broadly and suddenly guffawing. William forebore to ask them the joke, and when he reached home and lit the stable lantern to rub down the pony the joke became clears the true-blue farmer's sons had anointed the grey pony with a blue-bag from head to foot.

The sign of his fourth cornerstone was a piece of flimsy paper about ten inches square, and ruled in a gridiron pattern, (or rather, some dozens of such papers); for this was the circuit \quotemark{plan} still found today in every Methodist home and showing, quarter by quarter, the names of the preachers appointed to conduct services in each of sixteen chapels scattered over the Brendon Hills, Watchet, and Luckwell Bridge. For William's Christian faith was the rock on which he built everything else, and Sunday after Sunday and many week nights as well, he would travel on his grey ponies all over this wide circuit, placing his eloquence at the service of his beloved Bible Christian Church, the smallest of the Methodist Connexions and the one in which the sense of family was most strong. William began preaching at the age of sixteen in a cottage in Bilbrook, was adjudged promising, pursued his studies and preached continually for sixty-four years, until very shortly before his death in 1929, travelling more than thirty thousand miles, and representing his circuit for decades in the wider field of the West of England.

His greatest moment came when he was quite a young man. He was appointed to preach in the mining village of Gupworthy, eight miles away, 1,100 feet up, and in exposed country.

In the morning, the rain came teeming down and, knowing that the afternoon congregation would be small, he felt inclined to wait for the weather to clear. But he seemed to hear a voice saying \quotemark{Go} and he and his father-in-law set off. The congregation was small indeed and the atmosphere cheerless, and he wondered whether he had heard aright. Toward teatime, however, the weather cleared, the sun broke through and a glorious evening began. A crowd of young miners came unexpectedly to the meeting room and as the service progressed William felt himself taken by a power outside himself. He spoke of the need for a man to make a visible decision and twenty-eight young men there and then took him at his word. In a matter of days the membership of this mining church rose from fourteen to forty-five, and it remained there until the closure of the mines and the dispersal of the community a few years later.

When William died, full of years, and having known his share of sorrows, tributes were paid to him by men and organisations throughout the West of England. But he would, I believe, have valued most highly the words written to his son by the squire whom he himself honoured: \quotemark{I know of no man in this district who has accomplished more good than your father}.
