\Chapter{The Bakehouse}{Glyn Court}

On the parcel of ground bought by my father there stood seven buildings, one of which was a bakehouse, a two-storied structure of sandstone, with a large window downstairs and a loft door above, both faced with brick quoins. For many years - possibly for half a century - the bakery business had been carried on over the road in conjunction with the post office by James Bellamy.

It was here that my mother came to work at the age of sixteen as a clerk and telegraphist. She remembered Mr Bellamy was a kindly, mild-mannered old gentleman with pince-nez. He was a widower with one daughter who, sadly, had inherited little of his mildness; her tongue gave him little peace, and often he would come in to dinner and go out again leaving it untasted. He was a lonely man, for not only had he no happiness and little society at home, but he had also joined a small and, at that time, particularist religious body, the Christadelphiaans, and henceforth had no contact with the village chapel; (the village had no Anglican church): and as he and his daughter were the only Christadelphians in the district, loneliness was inevitable.

His bakery was an annexe to the post office, and his oven one of the old type which was heated by burning faggots and bundles of twigs inside; then the dough was baked by the heat of the ashes, which were raked out before the next baking.

Mr Bellamy covered an extensive round with his pony and trap, travelling to outlying farms with a regular order to supplement the home-baked bread. He kept his own bakehouse scrupulously clean, and was fascinated and shocked by the lower standards he found in some farms. In one in particular he was greeted by \quotemark{None to-day, Mr Bellamy, I'm baking myself. - Here, you young rip, gi' me that!}, and his horrified gaze fell first on the grey, speckled dough, then on the hands kneading it, then on a child playing with a lump of the mixture among the ashes in the hearth, and lastly on the hand which swooped down to snatch the dough and thrust it, ashes and all, in with the next batch for the family.

In the orchard over the way he had started to build a house to which he meant to retire, but he died while still at work and his daughter sold the shell of the house to my father. The village had lost its bakery, so father, although he was still living in Roadwater, set up a new one, installed a modern coke-fired oven and rented it to a young baker from Williton, Henry Chilcott, together with a cottage for his wife and month-old son, Colin.

It may sound improbable, but right up to the 1950s every village of middling size in this district could support a bakery, and in days when the general taste had not been vitiated by packaged pap, the extent of a baker's trade depended on the quality of his goods, and of course his rounds over- lapped with his competitors'. Mr Chilcott and his son - later, Colin alone -ran the business unaided for forty years, and the excellence of their bread and the richness of their dough cakes and buns were proved by the length of the round they covered, to farms four or five miles away in different directions. In the early days Mr Chilcott ran a delivery van, and his cheerful manner and smart appearance - together with polished leather leggings - made him a welcome caller. Then, as the road tax was viciously increased, he was forced to change to a B.S.A. motor cycle and sidecar, but no doubt it did the work as effectively, even if in less comfort. I suspect, though, that in the 1930s the expense of running even the B.S.A. became too great for a small business, and a delivery bicycle was brought into service, and Colin would take his cheerful way round the village, up the river valley to Stamborough and Leighland four miles away, or over the hills to Fair Cross, Yarde and Stream, taking with him everywhere a song and a cheerful word for everyone. Sometimes, although ten years older than I, he would put up with my company on these rides, and on walks in the country, and would stop and show me the age-old crafts of boyhood, producing an ear-splitting whistle with a blade of grass and making a popgun out of an elder twig and a length of spring from an old lady's discarded corset!

The bakery was, I suppose, typical of hundreds, it measured about fourteen foot square, and on the left as you went in you saw the great oven, like the belly of a whale when the heavy door was raised by a massive lever. Under this was the proving oven, in which the dough was placed to rise.

Against the further wall stood the huge wooden \quotemark{trow}, in which Mr Chilcott would knead a hundredweight and more of dough at a time. The baking tins were stored on a rack, but greased and laid out on the floor each morning to receive the dough. From the ceiling hung two long hooks which carried the ten-foot \quotemark{peels} for drawing the bread from the oven. In one corner a flight of steps, without a handrail, led up to the loft, into which the flour was loaded through an outside door by a pulley and chain; but it had to be brought down again, painfully, by hand. Every day of the week, except Sundays, Mr Chilcott or Colin would be up at half-past five to open up the oven, stoked the night before, and to start work, and one of the pleasant memories of my boyhood - more pleasant for me, no doubt, than for those who worked at it - is of the musical clink-clink of baking-tins being deposited on the bakehouse floor.

When, indeed, I think of the bakehouse I think of music, for music ran strongly in the family, Mr Chilcott had, until coming to Washford, played the flute in a chapel band in Williton, but that activity, like so much else that was good, was crushed out of existence in 1914, and after that he took up his instrument only rarely. Colin, however, had taken to music as soon as he could speak for himself. At five years of age he was learning the piano, at eight he was learning the organ. Soon he was helping the organist on Sundays besides, and he went on making music in this way, and many others, until life left him.

The organ could not be the only outlet for the music in him, he needed more, and in the 1930s he and his father formed a small strict-tempo dance band: Colin played the piano, two other lads played the saxophone and the violin and his father kept time on the drum. Unlike more recent groups they did not set out to deafen the dancers, and perhaps for that reason the \quotemark{Night Revellers} in their uniform of blue tunic with yellow edging, were all the more welcome, and they built up a list of regular engagements till 1943 put an end to their activities. In the 1950s ill-health brought on by the innumerable particles of flour compelled Colin to give up the bakery and take other work; so another village industry closed; but his music went on. He married a lady who was a singer, and they made their home, still next door to the old bakery, a place of music and constant cheer.
