\Chapter{Brief Flirtation with Commerce}{Glyn Court}

The summer holidays came on, and we spent the whole of the time at Washford, It was during this period that we discovered for the first time the delights of buying stock for the shop, and hardly a week went by without our driving off to Bristol or Exeter to visit the wholesaler. Shoe repositories had changed very little in fifty years, and we had a very interesting time poking around in the dusty corners of Quicks and Ware's in Exeter, or Lindrea in Bristol. On these visits one was impressed by the old world charm and courtesy that seems to be a tradition of the boot and shoe trade. Some of the assistants seemed almost as old as the stock, and they were obviously no more enthusiastic about modernising their premises than Glyn's parents - \quotemark{enthusiastic} is hardly the word, the thought of changing anything would never have entered their heads. The office at Quicks was a dark brown enclosure wherein dwelt hard-working clerks sitting at high mahogany desks.

On receiving the first account we were pleasantly surprised to note that it was written out in longhand in flowing, cursive script. Ware's business is conducted in an old tanyard in the village of Woodbury near Exeter - a convenient stopping-point for me when going to visit my parents. From time to time they told us the adjacent stream overflows and ruins all their stock - a story with an all too familiar ring to our ears. Their accounts are headed with a Biblical reference from Proverbs. On looking up this reference, we found the words: A false balance is an abomination to the Lord but a just weight is his delight. We thought this a pleasing touch in the hard world of commerce. Alas! Times are changing rapidly and most of the trading is now done in modern estates where I greatly doubt that one would have the great fortune to discover, as I did at Quicks, heaps of ancient hobnailed boots, just like ours, pushed into a corner.

Another excursion was to Associated Wholesale Stationers in Exeter where Bert, an amiable character in a grey pullover and with a strong Devonshire accent dispensed stationery, bail-point pens, jig-saw puzzles, maps of the South West and novelties. A rather endearing feature of these display rooms is an almost total disregard for the seasons, and a very pronounced disregard for what we have come to understand by \quotemark{advertising}. At Associated Wholesale Stationers, for example, it was always Christmas, Christmas decorations and \quotemark{lines} being an important part of their trade.

In order to sell the decorations of course they must be displayed, a tiresome business which necessitates climbing up on to steps and sticking pins into the ceiling. Hardly surprising then that once the season was over the decorations stayed up, becoming progressively more tatty and dusty as the year wore on. Presumably the new season brought new lines, but from my experience, albeit limited, of small wholesalers I should say that they often take nearly as long to clear one item from their shelves as small retailers. Somehow, it does take away the glamour and novelty of Christmas to see last year's decorations still up in August - the normal trading month when preparations for the attack on the Christmas market seriously begin. The most popular visit that we made that summer was to the Christmas Trade Fair at Bristol. This was altogether a much more sophisticated business with a huge variety of fancy goods, crockery, handbags, novelties, ornaments and toys from every imaginable country. We took the two eldest children with us and we fell prey to a particularly charming representative on a Chinese stand. We bought a large amount of goods, ranging from hand-painted fans, ivory bracelets and cotton birds, to bowls and spoons for a ridiculously small sum of money - I think we spent £25. Toys of course attracted the children, and before we knew where we were we were acquiring teddy bears, plastic fire engines, badminton sets, buckets and spades, sailor hats and the like. Here again we were astounded at the quantity we could purchase for £25 or so.

This sort of buying was not to prove our undoing as there is always a sale for cheap, attractive goods of this sort, but we came seriously unstuck on the footwear. The shop was now beginning to look much brighter and more attractive, and we had been painting the walls a little at a time, covering shelves with Fablon and generally making the whole place look much lighter and more attractive. Our customers were delighted at the change, and lulled by their favourable comments into a false idea of success, we invested fairly heavily in a new stock of shoes. To anyone considering selling footwear, we could probably give some very helpful advice - mainly of a very negative nature. We had to trust our own instinct plus advice from salesman on what to buy. You immediately run up against all sorts of problems. First, just think of all the different sorts of footwear. Take men's to start with: you have brown shoes, black shoes, good quality, cheap ones, lace-ups, brogues, slip-on casuals, heavy duty work shoes, reinforced boots, cricket boots, canvas casuals, sandals, slippers - the list is endless. Say you select a well-known brand of lace-up brown shoes in a popular price. What sizes? Say six to eleven. You are almost bound to be left with the sixes and need two or three pairs of eights. Are you keeping the half sizes? No? Then on the first day a customer will ask for a half size. Do you fancy wellingtons? A simple task, I hear you say, just order one or two small ones and lots of popular sizes. Ah, but everyone wears wellingtons. And they come in light, medium and heavy, roughly speaking, with subdivisions. Very well then, you will keep a complete stock of wellingtons from infants' fours up to men's elevens, including a few coloured short pairs for ladies. So let us add them up - fours to thirteens, not forgetting that you need two to three pairs of some popular children's sized to make sure that you do not miss sale, say twenty-five pairs, and then from ones to large elevens, another twenty-five pairs of so, allowing for a modest choice between the cheapest and better quality. That is at least £50 worth of wellingtons -and in a bad week you may not sell any, so all your lovely money is tied up in rubber and plastic.

One could go on indefinitely thinking of types of shoe one might try to sell - plimsolls, white and black (here again many different qualities and styles), children's sandals (the thirteens always sell out and often cannot be replaced), ladies', gents' and children's slippers (a much easier thing to sell) and greatest problem of all, ladies' shoes. Suffice it to say that a small shopkeeper would be well advised to keep well away from any line with a hint of fashion to it - we unfortunately thought we knew better. Our shoes were excellent value from a popular wholesaler whose name you will never see advertised in the press or on television, and we were able to sell them for about thirty shillings a pair at a time when most stores were charging about £3 for well-known makes. However, we had not reckoned for the sophistication of the present day villager and with the reluctance of our customers to actually part with cash for the goods. Coming from a middle-class background myself I had had it well drummed into me that one always paid for goods, did not contract credit agreements and so forth, Glyn too was rather vague about such matters, although we had often heard his parents grumbling in a general way without taking much notice. Consequently we were totally unprepared for the carefully calculated exploitation of our apparent new-found wealth.

There were various gambits and ploys to all of which we fell innocent victims. One was to establish a relationship with us by buying goods over a period and paying for them, and then one day the customer had \quotemark{forgotten her purse} or \quotemark{come out in a hurry} but could she have a pair of shoes, wellingtons, or whatever it was. We of course say yes, the goods disappear and so does the customer. In the case of those who had to come into the Post Office in order to collect pensions we soon devised a system where we kept back five shillings per week of the Family Allowance until the debt was paid. Needless to say we did not dare charge interest for this service - we were only too glad to get the money.

William and Ada often used to tell many tales of the depression days when the stock was literally rotting on the shelves, and the whole area lay suffocated under a blanket of poverty. Customers would come in hopefully offering to exchange goods for a pair of shoes - William was notoriously kind hearted and in this way unwittingly acquired all manner of rubbish. One of these items, a large thermometer, lay around for years. Its owner apparently had said that he would redeem it when times improved. But mostly they simply took the goods, squeezing out a tear or two for the occasion, and depended on time to erase memory of the debt. As he used to say, \quotemark{They'd come in with the children, point to their bare feet and say ’Can’t 'e let me have a pair of boots for ’en, Mr Court?' How could I refuse?} Altogether William used to reckon, that he lost £500 in bad debts during this time - it was probably even more than that as many of the debts were never written down.

Quite recently my husband ran into a farmer who remembered a fairly typical incident which illustrates William’s rather vague approach to business matters. William had fitted out the farmer's family with boots and shoes and in due course the customer naturally expected to pay for the goods which he had received. He found it, however, extremely difficult to persuade the courteous and absentminded William to present him with the bill. The debt dragged on for a month or two and Mr J., honest fellow, worried at the delay, reminded him several times of his desire to pay. The conversation proceeded something like this:

\begin{quote}
\textbf{Honest John}: I'll be up Saturday to pay for they boots! \\
\textbf{On Saturday, William}: Let me see now, 'twas one pair, wadn' it. That'll be nineteen and six, mister. \\
\textbf{Honest John}: One pair be darned! Twas dree, sna. (it was three, don’t you know). I owe 'e dree pun, you! (I owe you three pounds).
\end{quote}

Another gambit which we learnt to distrust was to send the children up to the shop, hoping that we would entrust them with the desired articles. \quotemark{Mum says, have you got a pair of slippers size four?} We soon learnt to harden our hearts, and just sent one slipper out.

By about half way through August, we had sold quite a few pairs of the old footwear at knockdown prices, but there were still so many pairs lying around outside that we decided to have one big sale to try to get rid of them all. So we turned out the \quotemark{Toc H} to make room for the goods. This in itself sounds quite simple, but it necessitated a major clear out and another bout of exhausting physical labour. The building, a rather pleasant stone structure with a slate roof, stands just to the back of the house. It must have been built about a hundred years ago, and for a long period was a stable for the Post Office horse as the Post Office had always been in this part of the village. You will probably wonder at the name - Toc H - but the explanation is quite simple. During the 1930s when the stable was no longer in use, William devoted quite a lot of his time and energy to converting the loft into a committee room for the Toc H. The room downstairs became yet another place to put things for which they had no particular use. William replaced the old loft ladder with a nice little staircase and knocked out the bricked-up spaces and made new 'windows upstairs and downstairs. Always a lover of the picturesque, he used up some of his stained glass (provenance uncertain but probably bought in sales) to beautify the door which replaced the old stable entrance. Later on, when the idea of the museum suddenly sprang on me, this building was the obvious place to put it.

For the time being however the most urgent task was to clear out the two rooms and make then sufficiently presentable for the public to see. Both upstairs and down were crammed with rubbish of various sorts. Most of the things in the upper room were household goods which belonged to an aunt who had stored them there about twenty years earlier. She was still alive but we just hoped she had forgotten about it when we carted most of the rubbish out to the bonfire. There were rotten mats and masses of brown pictures, crumbling lino and we counted pieces of five beds - box springs, feather mattresses and heavy wooden and metal sides. These were more of a headache and as they were too large to be carried down the stairs, had to be thrown out of the window - the children enjoyed that part as much as their parents! Very little of interest turned up apart from a rather nice patch-work quilt. The room is light and pleasant and, once it had been emptied of its contents, amply repayed our efforts. When the happy day arrives, if we can stop the rain getting in through the old slate roof, it will be even better. Downstairs the work was heavier going, however. Oil drums were our chief enemy here and it was a problem to know what to do with them all. Later on I made an illicit trip to the dump, an ancient slate quarry of bottomless depth. The Bedford van was packed with an assortment of these rusty drums, and I hurled them one by one over a chain link fence which had been specifically erected to prevent such malpractices. They made a most satisfying noise when they reached the bottom a full ten seconds later. I also seem to remember bundles of pieces of marble from old washstands, too small to be of any use. They followed the oil drums. In one corner stood a very interesting French harmonium. We have never discovered where this came from, though the girls had a great deal of fun playing it. Grates and boxes, including the original ones in which the petrol pumps had been delivered forty years earlier, lay around in mouldering heaps everywhere. I thought we were never going to stop sweeping away cobwebs and dirt. When we had finally cleared the room up as well as we could, we had time to notice the floor, which is beautifully cobbled and tiled, but absolutely ruined by so many years of leaking oil and dirt. We laid down all the old mats which we had found on the floor and over seven or eight years, the worst of the oil has gradually worn away.

After a week of this frenzied cleaning, we had sorted things out sufficiently to hold the sale. Bearing in mind that we had no paid assistance and had never been involved in anything remotely resembling a sale before, we made quite a fair success of it. Of course, we had to advertise the venture, spending five whole pounds on a block in the \quotemark{Free Press} and the rest was done by word of mouth and home-made signs. We had no proper equipment for making notices- I turned out quantities of old cardboard boxes and shoe advertisements , in short, any piece of cardboard which could by fair means or foul pass muster as a signboard and painted \quotemark{Shoe Sale} all over them. There was no paint fit to use and we had forgotten to buy any so I used black leather dye which I had found in the shop and some ancient varnish from a rusty tin, slapped on with a fossilized brush whose bristles had long since matted into a solid block. We made as many signs as we could and fixed them on to whatever was handy including the telegraph pole at the bottom of the garden, using rusty nails or pieces of Post Office string. Considering the abysmal amateurishness of our efforts, and the general confusion which reigned while we were trying to make the preparations, the results were astounding. The holiday season was at its height, and a showery day was certain to bring in plenty of curious visitors. The rows of hob-nailed boots, having been duly polished by the cleaning lady (a mammoth task which took the whole of a day) proved tremendously popular with the local men, and as they were only marked at thirty shillings a pair, sold at a great rate. Boys' and mens' shoes sold as fast as we could put them out, some for as little as five or ten shillings a pair. The Grannies of the village \quotemark{sent up} and got themselves fitted out with good quality solid leather open-top bunion shoes for a pound or so. Downstairs in the Toc H, we put all the \quotemark{hopeless} shoes. These consisted of literally dozens of pairs of black leather walking boots such as were worn by children in the early years of the century. We doubted whether they had been much worn as late as 1921 when William built the shop, and I can only assume that he had brought with him all the remaining stock from Grandfather's business at Roadwater. There were elastic-sided boots, and football boots with high tops - they just would not sell even at 2/6 a pair. Do you remember those rather dashing two-coloured men's canvas shoes of the 1930s? The sort Bertie Wooster sometimes wears with plus-fours? We had some for one shilling. Afterwards of course we wished that we kept more of these shoes but unfortunately we had not yet conceived the idea of using all this material to start a museum, as explained in Chapter 19.

One interesting observation that came out of our discoveries was how much smaller feet must have been fifty or sixty years ago than they are now. There were quite a few ladies' shoes' of the period leading up to the first world war, and they were in tiny sizes, ones and twos. Even a child would not be able to get them on nowadays, as they are so narrow in fitting.

A very pleasant side effect of the sale was the number of people who came in with reminiscences of the \quotemark{old days} and stories of how they had had their first boots or bicycles from Mr Court or his father. Unfortunately the coming of the motor car destroyed the close relationship between shop and village which existed even up to the beginning of World War II, and many people did not realise that our family business still existed, let alone that it was struggling to renew its life. Since William's death in 1953 there had for obvious reasons been a gradual falling-off in the number of men calling at the shop. The old-fashioned way of life had somehow allowed time for a leisurely chat over every purchase. It was, after all, a serious business to buy a pair of boot which would last you for several years or even a lifetime. In Grandfather's day, when the profit margin on his work was so tiny and customers few and impoverished, it could be a life and death affair to spoil a pair of boots or aggravate a customer by poor workmanship. Something of this tradition lingered on into the forties and fifties. A typical encounter would go like this:

\begin{quote}
\textbf{Customer} (usually a man in his late fifties or sixties) to girl behind counter giving her a scornful look as if to say \quotemark{I id'n going to be served by the likes of you}: \textit{Mr Court about?} \\
\textbf{Girl}: \textit{I'll go and see}. (Longish pause while William is informed of the presence of Mr So and So, collects his wits and proceeds in leisurely fashion into the shop, puffing rather.) \\
\textbf{William}: \textit{Well, Mister, how is it then.} (Like most countrymen he had two languages and would speak pure dialect when occasion warranted.) \\
\textbf{Customer}: \textit{All right.}
\end{quote}

Then would follow a lengthy discussion on whatever was the burning issue of the moment, trouble over a footpath, a new shelter for the village \quotemark{rec} and allied matters. After some minutes of this:

\begin{quote}
\textbf{Customer}: \textit{Well, Mister, 'ave 'ee got a pair of boots to fit me then?} \\
\textbf{William}: \textit{Let me see, ‘tis a eight, idn’ it, Mister?}
\end{quote}


Then would follow the dignified ceremony of finding a suitable pair, laying down the small C.\&R. mat always used for trying on (this had been in use for thirty years and was never replaced) and usually the customer would be satisfied, even praising all the previous boots which it had been his privilege to obtain from the establishment. This last perhaps would not be completely disinterested, as the transaction would invariably end with William giving discount on the purchase thereby reducing even further his already modest profit.

In the sad years of widowhood Ada used to lament the changing times, and on these occasions she would often talk about the old days, finishing up with the very true observation \quotemark{Twas Dad’s personality sold the stuff}. Self-pity however was not the main purpose of her reminiscences, she just loved to remember all the happy days and have a little chuckle over all the ups and downs of their life.