\Chapter{Times Best Forgotten}{Clare Court}

That first summer we spent the whole of the school holidays wrestling with all the problems at Washford, and by the time they came to an end everyone, including the children, was feeling very exhausted and longing to get back to our own home. As if the shop and house were not enough to deal with, we had also inherited the enormous garden with lawns and flower beds and ornamental stonework, not to mention the orchard and vegetable patch. Somehow Glyn managed to keep the grass cut but everything else had to wait and a large part of the garden had more or less reverted to wilderness with neck-high brambles where the raspberries had once been. The birds had a wonderful time but the sight of so much neglect was becoming rather depressing and more than once I wondered how we could ever possibly restore the garden to something of its former order. Not only that, but it was far from a picnic running our daily life in a house which had been occupied by an old lady living alone for fourteen years. I have already mentioned the fearsome cooker, but another major problem was how to do the washing. Until quite recently Ada had used the old copper in the washhouse. The iron fire bars under the actual copper had long since worn away and it was difficult to keep the fire burning at all as the whole fire, coal, sticks and all, tended to fall right through at intervals, Occasionally, just for the fun of it, I had lit the fire and had a good boil up, and I must admit that there is a certain satisfaction in setting alight a roaring blaze beneath the copper and hearing the bubbles start to work! Rinsing was done in a large galvanised bath outside the door and wringing by hand. Ada had never possessed a wringer and was very proud of the fact that she could wring out more water than any of the other women in her family. After the washing was finished, all the water you no longer needed was tipped down the path, and eventually found its way into a drain - one of the many mysterious clay pipes that shoot under the garden in all directions leading to obscure bramble-clad outfalls in forgotten corners of fields.

The washhouse would have been quite an important meeting place in bygone years as it served three cottages. I often think how much we have lost by doing our washing all alone in our kitchens. In parts of Italy and France and Africa still one can see the women enjoying a lively gossip as they beat out their clothes on the banks of rivers or in those murky washing places in the back streets of towns. Perhaps the Laundrette is a good thing if it gets people out of their houses and into a place where they can still chat together and render small neighbourly services such as helping one another to fold the sheets.

There is one more amusing story about the washhouse. There was a tap quite near the door which William had fixed for general convenience and watering the garden, but one day arrived a man from the Council Offices to have a little look round, the object of the visit being to assess the property for rates. Before his arrival William disappeared \quotemark{around the back} with a spanner and shortly afterwards announced that he had removed the tap, as he was afraid the official might notice it and. put his rates up. Naturally I expected him to replace the tap as soon as the representative had departed, but no! He never got around to it and so for years every Monday Ada uncomplainingly carried buckets of water from the kitchen to the wash-house, a distance of twenty yards or so. She finally had the tap replaced about twelve years after it had been removed. I thought it was a very laborious way of doing the washing. Not only was washing difficult, but the house had never been wired properly for electricity. There were lights, but the wiring had perished and there were weird arrangements of monstrous ugliness, with three or four wires all leading out of one small plug. We had never managed to persuade Ada to regard electricity with respect and the miracle was that nothing blew up. There had not been any power in the house at all until recently, when Ada had three points installed in the main living rooms so renewing the wiring was the first job that had to be done, which meant that in addition to coping with all our other problems, we had to endure the inconvience of floorboards being taken up all over the house for several weeks. It was unfortunate but the house had been built a few years before electricity came to the village. The original lighting, the latest thing at the time, had been by acetylene gas from a large tank in the garden. The pipe still leads into the house but we do not get rid of it as it makes a useful rack for hanging up the shovel and dustpan. The ornamental gas brackets lingered on, although there is only one left now. There again, it had a use, as a towel hook in the bathroom. Apparently the acetylene used to run out at awkward moments and plunge the sorting office into total darkness just when the mail was being sorted. The arrival of electricity in the village was greeted as a great boon - though the cost was a continual annoyance to William and Ada. The difficulty has always been that the size of the house makes it very expensive to run. The main rooms are quite large and the shop and sorting office were also built to generous proportions. Consequently lighting and heating have always been rather a headache. I think that, having always lived in small dark cottages, William longed for more space to live and breathe, and larger windows to welcome the sun and air into the house. Having built the house facing the brisk northerly breezes from the Bristol Channel, not only did they discover the practical problems of maintaining the large spaces they had created, but I think they felt rather lost in them. They always lived in the smallest, sunniest room, and tended to fill the rest of the house up with very large, heavy furniture. As the windows were large, Ada was convinced that everyone passing the windows would see in and that she would lose what little privacy she had, so consequently all the casements were heavily draped with rather dark curtains which were never fully drawn back. This used to annoy William who constantly complained that he had built the house with large windows in order to get plenty of light!

Bearing all this in mind no one would be surprised to learn that I was glad to abandon the rather gloomy house for a space and return to normality. It was now four months since we had inherited the property and it was becoming obvious that we must face up to the problem of what we were going to do with it all. One thing was certain, it would be quite impossible to continue much longer running two homes on a schoolmaster's salary. The income from the shop might develop in time, but up to the present we had spent far more than we had received despite valiant do-it-yourself efforts. I was already beginning to think that the charm of do-it-yourself was limited. It was all very well up to a point, but rather beyond a joke when one was perpetually so rushed and exhausted that there was no longer time to enjoy an excursion to the hills or take the children to the beach. We did not even have time for a holiday apart from a few snatched days and were beginning to feel the strain.

To cut a long story short, a suitable teaching post in the area eventually materialised and after a year of confusion we were finally able to move ourselves, our children and all our possessions into the house. This was by no means the end of our problems but it did mean that we were now totally committed to Washford, its pleasures and its pains, its treasures and its trials. For a long time the house and I did not get on very well together - I was after all the intruder who wanted to change everything and the house disliked change. Even the garden seemed to resent my sudden appearance on the scene. After seven years however we have reached an understanding and now view each other with mutual respect.