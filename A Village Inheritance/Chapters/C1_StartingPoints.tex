\Chapter{Starting Points}{Claire Court}

The story really begins quite a few years before either of us was born - in fact one would be hard put to it to name a date or a place, to fix on one particular event, a birth, death, marriage or building of a house, as having an overwhelm)ng and supreme significance in one's own particular portion of history. For we are all, whether we like it or not, inextricably and for ever woven into an infinitely complex pattern of history involving many people and places unknown to us. The delicate web of family connections, for example, knows no end when once a diligent search is made. Social upheavals of major or minor importance such as the rise and fall of the iron ore industry on the Brendon Hills, the coming of the railway, the Bible Christian movement, the advent of the telephone, the Great Depression of the 1930s -all these have had tremendous repercussions one way and another on the life of our family and helped shape many of its members.

It is usual when beginning a work of this sort (one hesitates to use such a grand title as “autobiography”') to fill the first few pages with a detailed account of all the known members of the family, their dates of birth, what they did far a living, how many children they had and so forth; but not only will much of this information come to light in the course of our narration, it will fill a large part of the ensuing pages. Suffice it to say therefore that my husband and I were married in 1950 at the respective ages of twenty-six and twenty-two years, and we came from very different backgrounds.

My father and family were draughtsmen and engineers in the gas industry and I had grown up very happily and pleasantly by the seaside on the South Devon coast where my father had charge of a small and not very onerous gas company. There was always plenty of time to play tennis, go sailing, camping, take part in amateur dramatic productions, and generally enjoy life. On my mother's side of the family I come from Cumberland farmers in the neighbourhood of Penrith., .Although not rich by the standards of pre-1940, we were comfortably off and the life we led certainly seems pleasant and rather easy when compared with the frantic existence of 1975. I had never left my home, apart from occasional shopping trips and Sunday School outings, and this rural calm continued even throughout the war years. Horizons began to widen a little with the preparations for D-Day, when large numbers of American soldiers were posted in our area for several months. These were the first foreigners that I had ever met.  Then at the end of the war, I progressed to University College, Exeter, where (apart from having my mind broadened just a little more) I met my future husband.

Glyn (as he is always known, discarding the other two grandparental names of Albert and William lovingly bestowed on him by his parents) came from a very different background He has in his veins the blood of Somerset yeomen, rural craftsmen and bootmakers, with a strong dash of the Great Western Railway on his mother's side, the whole nicely seasoned with a generous sprinkling of Post Office ink. One could no doubt find all sorts of interesting analogise or even, if so inclined, and with time and interest to spare, theories to account for this and that. Heredity after all has it exponents and I would be the last one to decry its merits. But for the time being I merely give the information for what it is worth.

I find myself back once again at the starting point, when I have almost convinced myself is going to be our wedding day, 23rd September 1950; the day when two fairly opposite poles suddenly came together seems as good a moment as any to call the beginning. In fact, I am almost ready to convince myself of the validity of the argument and take that date, namely twenty-five years ago, as the beginning, focal point, hub or radius, or whatever one likes to call it, of this story. If I decide on this course, a flashback technique will be inevitable.

Looking back to that day in 1950, not so very long after the war, when skirts were calf length and girls looked pretty (in my husband's opinion), I think it would be perfectly true to say that I had not the slightest idea what I was taking on. In common with most young brides, I quite sincerely believed that I was beginning a life of unequivocal bliss and happiness. I mention this not to contradict the belief but merely to stress the fact that I have had a very good share of both of those enviable commodities, but also quite unimagined practical problems which few people would expect to have to deal with in a normal married life in such quiet backwaters of Devon and Somerset. 