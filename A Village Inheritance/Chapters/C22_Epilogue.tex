\Chapter{Epilogue}{Glyn Court}

I approach the conclusion of this book almost as reluctantly as I approached the task of writing my share of it, and I hope the reader will to some extent share my feeling. Yet with each line I write, the sense of incompleteness grows upon me. My wife and I have, in point of fact, written two or even three books under one cover: hers, with its salutary tale of commercial upheaval and domestic stress, is detailed and complete, the tale of an episode that is closed; but mine, or the book written jointly, is merely an inchoate fragment of the chronicle that might be written on this one family; it gives no more than a hint of the rich social life of these hills and valleys in the days of Queen Victoria; while as for the legends that hang about our ways and dwelling places, we have found no place for them at all, knowing that once started we could never make an end.

From this comes the consideration that while admitting that not all families have had the good fortune of a tradition of conserving the past, every family has a body of memories, both common and particular, every family has its oral records and traditions, no matter how vague or dispersed; and we hope that this book will encourage readers here and there to undertake the gathering and collation of these scattered records, so that the past lives of their families, of their neighbourhood and of their homes may live again in the present and enrich the future with the wisdom - and the follies - of men and women not wholly unlike ourselves. \\

Washford, West Somerset \\
6th April 1976
