\Chapter{Of Oaken Things}{Clare Court}

Having built the house William's first and most pressing task was to finish off the floors - a not inconsiderable labour considering the area to be covered. A quick examination of our floors will soon reveal that the entire ground surface of the downstairs rooms is paved with wooden blocks. The shop, sorting office and living room floors were made of deal, for the precious oak would not run to covering the whole of the ground floor. William made every single oak block himself, planing the wood by hand, sawing the planks up into the required sizes and then laying them in herringbone pattern. The blocks measure approximately 8" by 2" by 1" thick though very few conform exactly to these measurements. Some are of irregular size and shape, such as the curved ones at the bottom of the stairs. In the course of writing about the floors it occurred to me that it would be an interesting calculation to work out just how many blocks were needed to cover the entire floor space. We did a few multiplication sums and came up with the grand total of 1,600 oak and 3,000 deal blocks.

There is one rectangular patch in the former shop which was not so paved. It was under the counter and William evidently did not consider it worth his while to waste labour and wood on it. Another corner which was never finished is under the stairs - an irregular triangular which still patiently awaits the day of fulfilment and completion. I wonder why my father-in-law never finished it? Perhaps a customer called to collect a pair of boots while he was on his hands and knees under the stairs. Perhaps, as I have often done, he hit his head on the massy oak as he got up with stunning effect. Be that as it may he never returned to finish off the corner. It was not because of the shortage of blocks either – there are still some out in the shed waiting, patiently waiting.

Over the years this oak parquet has been lovingly polished to a rich and mellow golden brown which warmly welcomes the rays of sunlight as they filter through the stained glass in the hall door. Unfortunately the wood has shrunk slightly over the years leaving narrow cracks between each block. In the main traffic areas the pitch with which they were stuck down has also worn away so that it is possible to prise out the blocks if one wishes to remove small objects that have disappeared into the cracks. This, needless to say, was a great attraction to the children in their younger days and our eldest daughter at a tender age spent many happy times removing as much of the hall floor as her little hands could lift up - and a surprising amount it was for a two year old! The game was so fascinating that she became quite cunning over it and learnt how to lure her Granny into the other room, shut the door quickly and dash back to continue her task unaided. Fortunately we usually managed to rescue the blocks before they travelled too far away from their proper resting place. 

Every house has its own noises, its own particular creaks, bumps and gurgles and another rather endearing feature of the floor blocks is the faint rocking sound made by the cats as their soft feet tread over the floor. It is a friendly little noise, something between a tap and a touch, which has become one of the idiosyncratic features of our home's individuality. When the day comes, if it ever does, that we have the floor relaid and all made perfect and whole, we shall miss the sound very much. The same could be said of the unfinished corner under the stairs. For years, this annoyed me intensely but long habit has made me grow used to it. After all - one wonders - is perfection necessary to our happiness? I think not. As long as some things remain in their unfinished state, there is always a stimulus - firstly to one's creative and orderly instincts, which command that one day the task shall be completed. Secondly, and by no means of lesser importance, there are the human associations (in the case of the floor I might be tempted to say human and feline) which breathe life into otherwise uninteresting and inanimate objects.

I think perhaps that this is why some stately homes and museums seem so soulless and deficient in human interest. They have been polished, tidied and set in order beyond the normal imagining of those who once lived in the rooms or used the objects which are displayed. This is one reason why I feel that our own home and little museum with all their imperfections, possess a quality which is so frequently lacking in those which suffer from \quotemark{la manie de la propreté}. They show man as he really is and not as an ideal in all the vigour of his creative talent, which however great, is not entirely immune from attack by occasional bouts of lethargy, laziness or boredom. They also show him as a very busy person who simply does not have the time to finish all the tasks that he has begun.

An interesting feature of our hall, which is entirely panelled, floored and furnished with oak, is the staircase. My father-in-law had at some time rescued an unwanted mill-wheel from one of the ten or so disused mills on the stream - probably New Mills at Luxborough. What condition the wheel was in when he transported it from the site we shall never know, but the sight of so much good timber going to waste was obviously too much for William to bear, and he conceived the idea of using the paddles to make the stairs in his house. Years of use and submersion in water must have made this a laborious and difficult task, but eventually the stairs arose as it were from the mill pond, ascending in a pleasing though somewhat dangerous curve to the landing. Despite William's efforts he was unable to remove the mark of a giant iron bolt which had burnt itself into the timber. The bolt has left a geometrical black stain on one of the stairs. We had often wondered about the origin of this strange black mark, but it was not until a few years ago that a visitor to our house told us the story about the mill wheel which Glyn had forgotten literally for decades. The explanation of the mark suddenly became very obvious.

The newel posts and banisters were also hand-carved though they too did not take kindly to the introduction of central heating into the hall. Rather characteristically, William's concentration was distracted from finishing the banister off with uprights and this left rather a large open space which must have a strong feeling of insecurity to anyone standing on the landing. Time went on however and still the banisters were not completed - until suddenly the arrival of a baby, which quickly developed into a toddler, demanded urgently that something be done to fill the gaping void. William solved the problem - quickly and effectively - he brought into the house the side of a bicycle crate which he proceeded to rope to the banisters! This crude contrast apparently did not upset his aesthetic sense, for the crate remained there for years, in fact, until Glyn came back from the army in Burma in 1947. At this point William evidently considered the time had come to replace the crate with something better, and the top part of the landing was fitted out with slats. On close examination they do not quite come up to the standard of the rest, but fortunately the light is not too good at the top of the stairs. On the opposite side, however, the slats are of very rough wood hastily nailed in, I imagine, to stop Glyn as a baby falling through. Perhaps one day we shall find a carpenter who will make it a labour of love to match the craftsmanship of the hall entrance, and finish off the banisters in fitting harmony with the rest. Until that day comes, the four-footed members of the household know that here is one place in the house where sharpening of claws is not forbidden.

Not content with making all the woodwork which was needed for the house, skirting boards, windows, doors, panelling, flooring and more, William also turned his hand on more than one occasion to making furniture. By trade he was a carpenter and not a cabinet maker, and although, when necessity urged, he would produce a polished piece of work, furniture making was a secondary interest for him. When we first had a home of our own we had no furniture so Glyn's mother gave us one of the home made beds. We slept on it for years. These beds were, of course, of oak and consequently though their qualities of endurance were truly British they were hardly calculated to fill the heart of any housewife with delight. For one thing, they weighed a ton, or so it seemed if ever one desired to move them from one side of the room to the other, and for another, they were almost a danger to life and limb, as William had not realised that sharp carved knobs on bed posts were hardly practical - in fact downright dangerous. Both head and foot rails were chest high, and the slats across the base of the bed were pieces of solid timber. When we took the beds to pieces we noticed for the first time that each piece was inscribed with Roman numerals I II III IV. The sides were united to the base by means of iron bolts ten inches long! On top of the slats reposed heavy box springs, also with wooden sides to add to the already considerable weight of the frame of the bed. The crowning glory of the structure was a feather bed and over all reposed a white linen bedspread.

There were two such beds, and such was their gravity and seriousness that it would have been quite easy to imagine Matthew, Mark, Luke and John standing vigil at the corners. The best bed on which William and Ada slept boasted, in addition to its solidity, twin panels of carving at head and foot. Unfortunately one section of a foliage capital had broken off along the grain of the wood, rather spoiling the symmetry of the design, and as long as I can remember seeing it, this part was always tied on with a piece of post office string. Both these beds now repose in the bakehouse awaiting their call to a further life but the trump has not yet sounded for them. They are too good to be thrown out, but far too cumbersome to use. My plan is to work them into the banisters for in this way the carving will not be lost.

Apart from a chair of no great distinction, William's best piece of furniture is undoubtedly the three-cornered cupboard which stands in the corner of our front room. Ada was very fond of this cupboard as William had made it for her as her birthday present in the first year of their marriage. It is in the late Victorian style and is opened by one of those fascinating wooden rings carved from a single block of wood. It is a pity that the lower half (in which George Hoyle's model of the steam engine was always kept) never got finished, but William was skilled at disguising his little incompletions and it is only on close examination that you will notice that the wood is a panel from a tea chest, cleverly varnished.
